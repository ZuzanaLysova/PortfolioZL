%
\usemodule[ctx-thesis-v0.991]
\usemodule[bib.sty-v2.78]

\setupthesis[cs,mendelu,pef,none][ % jazyk,univerzita,fakulta,ústav/katedra/pracoviště ; language,university,faculty,department
  type={dp},                 % bp,dp,pp,zp,sp,pr,pt aj./etc.
  authorname={Zuzana},	     % jméno
  authorsurname={Lysová},        % příjmení
  authordegree={Bc.},	     % titul před jménem
  authorgender={F},          % pohlaví (holky mají F)
  supervisor={RNDr. Zuzana Špendel, Ph.D.},        % vedoucí práce
  title={Analýza, návrh a implementácia softwarovej platformy pre firmu Asseco Central Europe, a.s.},                           % název práce
  titleen={Analysis, design and implementation of software platform for Asseco Central Europe, a.s.}, 	           % název práce anglicky
  keywords={bla,blabla,bla,blabla},    %
  keywordsen={aa,xxx,vvv,aa,bbb}, %
%  acknowledgement={Děkuji své babičce, že mi napekla na cestu buchty.},	           % poděkování
  abstract={},		                           % český abstrakt
  abstracten={},		                   % anglický abstrakt
  location={Brno,Brně},	   % místo vydání (za čárkou 6. pád) ; location (second parameter is not necessary for English)
%  year={2021},		   % rok odevzdání práce (automaticky aktuální rok) ; year, the default is the current year
%  thesisassignmentform={img/file001.png,img/file002.png},  % seznam souborů se skenem zadání práce; file is thesis assignment
]

\startthesis
\startbodymatter

\kap{Úvod}

asdadadf
\TODO Napsat celou práci.

\kap{Cieľ}

\kap{Súčasný stav}

%\QUES Opravdu musím psát literární rešerši? 

\kap{Metodika}

\pkap{Popis EMMA}

Tady je text.

\pkap{Popis SAMO}

Tady je text.

\pkap{Popis metodiky, analýzy a návrhu}

Tady je text.

\pkap{Charakteristika vybraných služieb verejnej správy}

Tady je text.

\kap{Výsledky}

\pkap{Model požiadaviek}

Tady je text.

\pkap{Use case model}

Tady je text.

\pkap{Sekvenčný diagram}

Tady je text.

\pkap{Konceptuálny dátový model}

Tady je text, který obsahuje odkazy na obrázky~\in{}[o1] a \in{}[o2].

\pkap{Logický dátový model pre SAMO}

Tady je text.

\pkap{Výber služby, ktorá bude implementovaná}

Tady je text.

\pkap{Implementácia vybranej služby}

Tady je text.

\pkap{Návrh testovacích scenárov}

Tady je text.

\pkap{Dokumentácia prevedených testov}

Tady je text.

\obrazek{o1}{Toto je první obrázek.}{img/pomocnyobrazek.jpg}{scale=500}

\obrazek{o2}{Toto je druhý obrázek.}{img/pomocnyobrazek.jpg}{width=4cm}

\kap{Diskusia}

Tady je text.

\kap{Závěr}

Tady je text.

%%%%%%%%%%%%%%%%%%%%%%%%% \def\refname{}

\bbib
Tady budou citace, nastuduj si styl bib.sty v příslušné verzi.
\ebib

\stopbodymatter

%%%%%%%%%%%%%%%%%%%%%%%% Varianta, kdy seznamy jsou součástí práce a nejsou uvedeny v přílohách

\setupsectionblock[backmatter][before={\setuplist[kap][before={}]}]

\startbackmatter

\THESIScompletelistof{tables}
\THESIScompletelistof{figures}
\THESIScompletelistof{abbreviations}
\THESIScompletelistof{codes}

\stopbackmatter

%%%%%%%%%%%%%%%%%%%%%%%% Varianta, kdy seznamy nejsou součástí práce, ale jsou zařazeny do příloh.
%%%%%%%%%%%%%%%%%%%%%%%% Níže uvedeným čtyřem příkazům postačí odstranit znak procenta.
%%%%%%%%%%%%%%%%%%%%%%%% Naopak před výše uvedené čtyři příkazy je potřeba znak procenta vložit.

\startappendices

\cast{Přílohy}
%\THESIScompletelistof{tables}
%\THESIScompletelistof{figures}
%\THESIScompletelistof{abbreviations}
%\THESIScompletelistof{codes}

\stopappendices

\stopthesis

\endinput		

%%%% TODO %%%%%%%%%%%%%%%%%%%%%%%%%%%%%%
Tady si můžeš psát poznámky, které se neobjeví ve výstupu.
