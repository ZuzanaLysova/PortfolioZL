%
\usemodule[ctx-thesis-v0.991]
\usemodule[bib.sty-v2.78]
\setupinteraction[state=start, color=blue]
\overfullrule=3cc
\setupthesis[sk,mendelu,pef,none][ % jazyk,univerzita,fakulta,ústav/katedra/pracoviště ; language,university,faculty,department
  type={dp},                 % bp,dp,pp,zp,sp,pr,pt aj./etc.
  authorname={Zuzana},	     % jméno
  authorsurname={Lysová},        % příjmení
  authordegree={Bc.},	     % titul před jménem
  authorgender={F},          % pohlaví (holky mají F)
  supervisor={RNDr. Zuzana Špendel, Ph.D.},        % vedoucí práce
  title={Analýza, návrh a implementácia softwarovej platformy pre firmu Asseco Central Europe, a.s.},                           % název práce
  titleen={Analysis, design and implementation of software platform for Asseco Central Europe, a.s.}, 	           % název práce anglicky
  keywords={SAMO, EMMA, digitálna verejná správa, správa zamestnancov, softwarová platforma, digitálna integrácia},    %
  keywordsen={SAMO, EMMA, eGovernment, analysis, employee management, software platform, digital integration}, %
acknowledgement={Ďakujem firme Asseco Central Europe, a.s. za možnosť rozvíjať projekt EMMA v rámci diplomovej práce. Veľká vďaka patrí rodine a priateľom za nekonečnú podporu, kolegom za konzultácie a vedúcej práce za pripomienky a pomoc.},	           % poděkování
  abstract={Táto diplomová práca sa zameriava na analýzu, návrh a implementáciu softwarovej platformy firmy Asseco Central Europe, a.s. Ide o platformu EMMA, ktorej cieľom je integrácia služieb verejnej správy do systémov firiem. Práca podrobne analyzuje súčasný stav eGovernmentu v Českej republike aj zahraničí. Výstupom práce je dôkladná analýza, návrh a následná implementácia prezentačnej vrstvy platformy EMMA, ktorá je realizovaná pomocou platformy SAMO. Konkrétne sa práca zaoberá najmä modulom zamestnancov s cieľom zjednodušenia a optimalizácie.},		                           % český abstrakt
  abstracten={This thesis focuses on the analysis, design, and implementation of the EMMA software platform for Asseco Central Europe, a.s. The aim of the platform is to integrate public administration services into corporate systems. The work thoroughly analyzes the current state of eGovernment in the Czech Republic and internationally. The output of the thesis is a detailed analysis, design, and subsequent implementation of the presentation layer of the EMMA platform, which is realized using the SAMO platform. Specifically, the thesis primarily addresses the employee module with the goal of simplification and optimization.},		                   % anglický abstrakt
  location={Brno,Brne},	   % místo vydání (za čárkou 6. pád) ; location (second parameter is not necessary for English)
%  year={2021},		   % rok odevzdání práce (automaticky aktuální rok) ; year, the default is the current year
%  thesisassignmentform={img/file001.png,img/file002.png},  % seznam souborů se skenem zadání práce; file is thesis assignment
]

\startthesis
\startbodymatter

\kap{Úvod a cieľ}
V súčasnosti je množstvo zamestnancov zahltených rutinnými administratívnymi úkonmi. Firma Asseco Central Europe, a.s. prišla s nápadom vytvoriť integračnú platformu s pracovným názvom EMMA, ktorá by mala tieto úkony minimalizovať. Projekt je v štádiu riešenia a aktuálne existuje zjednodušená implementácia služby \uv{nástup zamestnanca do zamestnania}, ktorá je v katalógu služieb VS. 

Katalóg služieb verejnej správy Českej republiky obsahuje v dnešnej dobe takmer 8 tisíc služieb a vyše 34 tisíc úkonov.  Tieto služby a úkony sa týkajú bezmála 400 agend a ohlasuje ich približne 30 rôznych ohlasovateľov (ministerstvá, úrady a pod.). Väčšinu týchto úkonov je možné previesť online, prostredníctvom dátovej schránky. 

Projekt EMMA je riešením, ktoré umožni integráciu služieb VS do informačných systémov firiem a spojenie viacerých služieb verejnej správy. Už spomínaný príklad nástupu zamestnanca do zamestnania umožňuje pomocou jedného formuláru (poprípade tlačítka) nahlásiť túto udalosť príslušným úradom -- Českej správe sociálneho zabezpečenia a zdravotnej poisťovni.

Prvotným zámerom projektu EMMA je integrácia služieb VS, no do budúcna sa plánuje rozšíriť to aj o služby komerčnej sféry, napr. poistenie auta. 

V rámci tejto diplomovej práce je cieľom navrhnúť a implementovať účinné rozšírenie platformy EMMA na základe detailnej analýzy. V súlade s požiadavkami spoločnosti Asseco Central Europe, a.s. je hlavným cieľom rozšírenie prezentačnej vrstvy platformy s dôrazom na modul zamestnancov. Prezentačná vrstva je implementovaná pomocou platformy SAMO.

Prvým krokom bude dôkladná analýza súčasného stavu eGovernmentu\zlom a G2B2B služieb v Českej republike a vo svete. Táto analýza umožní získať hlbšie porozumenie možností rozvoja platformy EMMA. Zároveň bude súčasťou práce i analýza existujúceho stavu projektu EMMA, ktorý je v období začiatku tejto práce už rozpracovaný.

Ďalej bude práca zameraná na vypracovanie analýzy a analytických modelov pomocou modelovacieho jazyka UML, ktorý poskytne vizuálny a štruktúrovaný prehľad procesov a ich interakcií. 

Zároveň s analýzou bude kľúčové aj spracovanie návrhu vhodného riešenia. Výstupy budú slúžiť ako základ pre ďalšiu fázu práce, ktorá zahŕňa implementáciu vybraných rozšírení platformy.

V ďalšej fáze bude práca zameraná na samotnú implementáciu v súlade s architektúrou a princípmi platformy EMMA a SAMO. Tento proces zahrnie programovanie funkcionality, integráciu s existujúcimi komponentmi a testovanie implementovaných procesov podľa štandardov firmy Asseco CE.

Záverečná fáza práce zahrnie testovanie implementovaného riešenia pomocou pripravených testovacích scenárov.

Výstupom práce bude funkčné rozšírenie platformy SAMO EMMA, ktoré bude nielen odpovedať na aktuálne potreby trhu, ale aj poskytovať pevný základ pre ďalší rozvoj a inovácie v oblasti digitalizácie procesov. Diplomová práca tak prinesie hmatateľný prínos pre firmu Asseco Central Europe, a.s. a pre širšie spektrum jej klientov, ktorí sa spoliehajú na efektívne digitálne riešenia.


\kap{Súčasný stav}
V dnešnej dobe sa digitalizácia verejnej správy stala častou témou rôznych diskusií. Je to najmä preto, že predstavuje cestu k efektívnejšiemu, účinnejšiemu a transparentnejšiemu poskytovaniu služieb naprieč rozličnými odvetviami.\zlom V~tomto dynamickom kontexte sľubuje digitalizácia transformáciu tradičných modelov služieb na modely, ktoré sú viac prispôsobené súčasným potrebám občanov a~inštitúcií. \scr(Andersson, 2022)

Medzi hlavné výhody digitalizácie verejnej správy patrí zvýšenie efektivity práce, kvality služieb, spoľahlivosti a dôvery, a to vo vzťahu občana a verejnej správa i vládnych inštitúcií medzi sebou \scr(Terlizzi, 2021).

Pre správne chápanie fungovania eGovernmentu v Českej republike (i vo svete) je nutné pochopenie základných pojmov a koncepcií, ktoré sú popísané v nasledujúcich kapitolách.

\pkap{Architektura eGovernmentu ČR}
Termín eGovernment,  ktorý sa v tejto práci opakovane objavuje, označuje pojem popisujúci modernú digitálnu verejnú správu. Opiera sa o využitie digitálnej infraštruktúry pre efektívne vykonávanie právomocí daných inštitúcií. 

Táto infraštruktúra realizuje sadu služieb informačných technológií (ICT služieb), ktoré sú zdieľané, dôveryhodné, prepojené, bezpečné, automatizované, efektívne a ľahko používateľné pre užívateľov. 

Služby eGovernmentu sú určené občanom, firmám, podnikateľom i úradníkom. Synonymami pojmu eGovernment sú \uv{digitálny government} alebo \uv{digitálna verejná správa} \scr(Digitální a informační agentura, 2023).

Digitálna verejná správa používa rôzne poskytnuté a dostupné informácie, ktoré automatizovane spracuváva s cieľom obmedziť, respektíve znížt množstvo podania a objemu informácií zo straný užívateľov služieb VS.

Hlavným poslaním eGovernmentu je: \citat{Poskytovať klientom verejnej správy jednoduché a efektívne služby, ktoré im uľahčia dosiahnutie ich práv a nárokov, ako aj plnenie ich povinností a záväzkov vo vzťahu k verejnej správe.} \scr(Digitální a informační agentura, 2023)

Vízia eGovernmentu v ČR do konca horizontu Informačnej koncepcie ČR (viac popísaná v kapitole 2.2 Informační koncepce ČR) je: \citat{Česká republika je jednou z popredných krajín v užívateľskej prívetivosti verejnej správy vďaka svojmu klientsky orientovanému prístupu, modernému dizajnu úradných procesov a efektívnemu využívaniu digitálnych a nedigitálnych technológií.} \scr(Digitální a informační agentura, 2023)

\pkap[kap:ikcr]{Informační koncepce ČR}
Informační koncepce ČR (ďalej ako IKČR) rozpracováva vyššie spomenutú víziu do rôznych cieľov, ktoré realizujú jednotlivé orgány VS. Predstavuje komplexný plán na rozvoj informačných systémov verejnej správy, ktorý je prispôsobený potrebám a cieľom štátu.

To, či ciele boli naplnené alebo nie ukazuje stav plnenia zadefinovaných cieľov a pozícia v rebríčkoch ako je napríklad DESI (rozobraté v kapitole 2.5.1 Index digitálnej ekonomiky a spoločnosti).

Všetky povinné subjekty podľa zákona  č. 365/2000 Sb., o informačných systémoch, majú povinnosť viesť vlastné informačné koncepcie a vždy ich musia uviesť do súladu s Informačnou koncepciou ČR. Je to prakticky koncepcia rozvoja informačných systémov verejnej správy, ktorú spracováva Ministerstvo vnútra a schvaľuje vláda. Je vypracovaná na základe ustanovenia § 5a, Zákona č. 365/2000 Sb., o informačných systémoch verejnej správy. Týmto prístupom sa zabezpečuje jednotný rámec pre rozvoj a prevádzku informačných systémov a služieb eGovernmentu v celej krajine.

\blank
Medzi hlavné časti IKČR patria:

\startitemize
\item{\start\bf architektonické principy \stop eGovernmentu a elektronizácie VS,}
\item{\start \bf efektívny rozvoj \stop digitálnej verejnej správy a informačných systémov verejnej správy (ISVS),}
\item{\start\bf zásady \stop riadenia ICT vo verejnej správe,}
\item{základné koncepčné \start \bf povinnosti \stop pre budovanie, rozvoj a prevádzku\zlom ISVS a ich vzájomné prepojenie a pre budovanie spoločných služieb eGovernmentu.}
\stopitemize

IKČR je základný dokument, ktorý určuje dlhodobé ciele a strategické smerovanie ČR v oblasti informačných systémov a digitálnych služieb verejnej správy a všeobecné princípy obstarávania, tvorby, správy a prevádzky ISVS v ČR. Obsahuje predovšetkým:
\startitemize
\item{ciele a podporu oblasti eGovernmentu (zo strany informačných systémov verejnej správy),}
\item{zásady riadenia útvarov informatiky a riadenie životného cyklu ISVS,}
\item{architektonické principy pre návrh a rozvoj ISVS a ich služieb. \scr(Digitální a informační agentura, 2023)}
\stopitemize

\ppkap{Metódy riadenia ICT verejnej správy ČR}
Súčasťou a kľúčovým predpokladom naplnenia cieľov stanovených v IKČR je zavedenie efektívnej centrálnej koordinácie riadenia ICT. Zároveň je to aj podpora transformačných iniciatív, ktoré smerujú k digitalizácii VS a plnému digitálnemu governmentu.

\uv{Metódy riadenia ICT verejnej správy ČR} (ďalej ako MRICT) je dokument, ktorý stanovuje pravidlá prevádzkovania ICT kapacít, kompetencií štátnych podnikov, riadenia útvarov informatiky, centrálneho koordinovaného riadenia ICT podpory eGovernmentu a podobne. MRICT nadväzuje na zásady riadenia ICT, ktoré sú súčasťou IKČR, a predstavuje kľúčový nástroj na zabezpečenie súladu a efektívnosti pri procesoch digitalizácie VS \scr(Digitální a informační agentura, 2023).

\pkap{Katalóg služieb verejnej správy}
Katalóg služieb VS je súčasťou Registra práv a povinností (RPP) a obsahuje údaje o službách VS, úkonoch a dostupných kanáloch. RPP je jedným zo štyroch základných registrov, medzi ktoré ďalek patria Registr obyvatel (ROB), Registr osob (ROS) a Registr územní identifikace adres a nemovitostí (RÚIAN). Tieto základné registre sú hlavným zdrojom dát o právnych subjektoch a objektoch a o procesoch vykonávaných v rámci verejnej správy. \scr(Digitální a informační agentura, 2023)

\blank
Katalóg služieb VS sa dá vnímať z dvoch pohľadov:

\startitemize[a]
\item{ako klientská aplikácia, ktorá poskytuje údaje klientom}
\item{ako úradnícka aplikácia, ktorá je určená na zber a úpravu údajov}
\stopitemize

Funkcie katalógu služieb VS možno ozdeliť do štyroch kategórií:

\startitemize
\item{\start \bf automatizačné \stop – zber dát potrebných na automatizáciu}
\item{\start \bf informačné \stop – poskytovanie prehľadu o existujícich službách VS a spôsobu ich spracovania}
\item{\start \bf publikačné \stop – poskytovanie informácií, ktoré sú potrebné na korektné zobrazovanie služieb VS na portáloch VS (kategórie, radenie\dots)}
\item{\start \bf riadiace \stop – riadenie poskytovania a dodávky služieb VS (tvorba plánu digitalizácie, zodpovednosť za služby\dots)}
\stopitemize

Časti katalógu služieb VS sú služby vykonávané z úradnej moci, ale taktiež aj služby, ktoré iniciuje klient (subjekt práva). 


%Na vyplnenie katalógu služieb je nutné urobiť nasledujúce kroky:
%
%\startitemize[n]
%\item{Identifikovať služby VS a popísať ich atribúty v ohlasovaných agendách.}
%\item{Rozložit služby VS na jednotlivé úkony a popísať ich atribúty.}
%\item{Definovať spôsob, akým dochádza k interakciou medzi OVM a klientom a určit obslužný kanál.}
%\item{Určiť časové rámce a obslužné kanály pre vykonávanie digitálneho úko-nu a využívanie digitálnych služieb.}
%\stopitemize

Vzhľadom na to, že údaje v katalogu služeb VS sú referenčné, je nutné ich udržiavať aktuálne \scr(Digitální a informační agentura, 2023).

Obsahom katalógu služieb VS sú služby a úkony. V Českej republike je to v súčasnosti takmer 8 tisíc služieb a vyše 34 tisíc úkonov \scr(Digitální a informační agentura, 2023).

\start\bf Služba \stop reprezentuje funkciu (činnosť) úradu, ktorá je poskytovaná konkrétnym OVM (úradníkom) konkrétnemu príjemcovi služby podľa príslušného právneho predpisu. Prináša príjemcovi hodnotu -- buď vo forme benefitu alebo splnenia zákonnej povinnosti. Pri službe VS ide vždy o interakciu medzi OVM a klientom (a opačne). Každá služba sa skladá z minimálne jedného úkonu. \scr(Digitální a informační agentura, 2023).

\start \bf Úkon \stop je taktiež interakcia medzi klientom a OVM, no v tomto prípade ide len o jednu interakciu, ktorá vedie k ďalšiemu úkonu resp. k naplneniu výstupu služby, ak sa jedná o koncový úkon. Úkon sa teda dá definovať ako jeden krok, jedna časť služby VS \scr(Digitální a informační agentura, 2023).



\pkap{Digitální Česko}
Táto kapitola je venovaná iniciatíve Digitálne Česko. Ide o ucelenú víziu, ktorá je realizovaná na základe niekoľých koncepcií, plánov a stratégií, ktoré sú v súlade s potrebami ČR a politikou EÚ.

Projekt Digitálne Česko pokrýva tri základné piliere:
\startitemize
\item{\start \bf Česko v digitální Evropě \stop -- vládna koncepcia zameriavajúca sa na jednotný digitálny trh v Európe} 
\item{\start \bf Digitální ekonomika a společnost \stop -- strategický dokument, ktorého cieľom je koordinácia agend z oblastí digitálnej ekonomiky a spoločnosti naprieč verejnou správou, hospodárstvom, sociálnou či akademickou sférou (súvisí aj s kap. 3.5.1)}
\item{\start \bf Informační koncepce České republiky \stop (popísaná v kap. 3.2)}
\stopitemize

Vláda ČR považuje program Digitálne Česko za súbor stratégií, ktoré vytvárajú predpoklady pre dlhodobú prosperitu krajiny v ére digitálnej transformácie a revolúcie. (Úřad vlády ČR, 2024)


\pkap{Postavenie Českej republiky v oblasti digitalizácie}
Existuje viacero spôsobov hodnotenia a merania úrovne rozvinutosti krajín v oblasti digitalizácie a rozvoja eGovernmentu. Patrí medzi ne napríklad \start \it Index digitálnej ekonomiky a spoločnosti, Index rozvoja eGovernmentu, eGovernment Benchmark \stop a iné \scr(Terlizzi, 2021). Tieto nástroje umožňujú detailnejšie pochopenie postupov, ktoré krajiny implementujú na podporu eGovernmentu a identifikáciu oblastí, v~ktorých je potrebné zlepšenie.

V nasledujúcich kapitolách sú v~skratke popísané spomínané prieskumy\zlom a~ich posledné výsledky. Analýza výsledkov umožní lepšie pochopiť súčasný stav na národnej aj medzinárodnej úrovni.

\ppkap{Index digitálnej ekonomiky a spoločnosti}
Európska komisia sleduje a monitoruje pokrok členských štátov v digitálnej oblasti od roku 2014 a každý rok zverejňuje informácie o indexe digitálnej ekonomiky a spoločnosti (Digital Economy and Society Index, DESI). Tento index zoraďuje štáty podľa úrovne digitalizácie a zároveň posudzuje ich relatívny pokrok za uplynulých päť rokov vzhľadom na ich počiatočnú situáciu.

\blank
Oblasti, ktoré skúma DESI sú:

\startitemize
\item{\start\bf ľudský kapitál\stop -- internetové znalosti používateľov, pokročilé znalosti ľudí v IT oblasti}
\item{\start\bf konektivita\stop -- využitie a pokrytie internetového pripojenia a~jeho ceny}
\item{\start\bf integrácia digitálnych technológií \stop -- digitálne technológie pre firmy (cloud, umelá inteligencia\dots), e-commerce \footnote{e-commerce -- obchodné činnosti prevádzané na internete a pomocou ďalších elektronických prostriedkov}}
\item{\start\bf digitálne verejné služby \stop -- eGovernment, otvorené dáta \footnote{otvorené data (open data, vládne dáta) -- informácie verejného sektoru, ktoré sú bezplatne dostupné na akékoľvek účely}}
\stopitemize 

Európska komisia spolu s Radou prejednávajú rozhodnutie o politickom\zlom programe \start\it \uv{Cesta k digitálnej dekáde} \stop, ktorý stanovuje ciele na úrovni Európskej únie, dosiahnuteľné do roku 2030. Cieľom je zaistiť to, aby bola digitálna transformácia komplexná a udržateľná a aby prebehla vo všetkých odvetviach hospodárstva. Dosiahnutie cieľa programu závisí na všetkých členských krajinách a na ich spoločnom úsilí \scr(Európska komisia -- Metodika, 2022).

\obrazekH{obr:DESI}
{Index digitálnej ekonomiky a spoločnosti 2022 (Európska komisia -- Česko, 2022)}{images/DESI.png}{width=30cc}

\obrazekH{obr:DESI2}
{Index DESI 2022 -- relatívne výsledky v jednotlivých oblastiach (Európska komisia -- Česko, 2022)}{images/DESI2.png}{width=30cc}

Česká republika je podľa výsledkov DESI za rok 2022 na 19. mieste (z 27 členských štátov) (viď \in{obrázok}[obr:DESI]). V porovnaní s rokom 2021 sa Česká republika zlepšila v oblasti digitálnych verejných služieb a konektivite. Zhoršila sa v~integrácii digitálnych technológií.

Možným pozitívom a príležitosťou je, že za digitalizáciu verejnej správy v~ČR je od roku 2007 prvýkrát zodpovedná konkrétna osoba, miestopredseda Ivan Bartoš. ČR pokračuje v implementácii stratégie \uv{Digitálne Česko} z roku 2018 (aktualizovanej v roku 2020) \scr(Európska komisia -- Česko, 2022).

Na grafe z \in{obrázku}[obr:DESI2] možno vidieť okrem pozície ČR aj výsledky všetkých ostatných krajín vrátane lídrov v hodnotení -- Fínsko a tesne za ním Dánsko. Graf na \in{obrázku}[obr:DESI2] ukazuje porovnanie jednotlivých oblastí indexu DESI s~priemernými výsledkami krajín EU-27. \scr(Európska komisia -- Česko, 2022)

\ppkap{Index rozvoja eGovernmentu}
Ďalším významným prieskumom je hodnotenie eGovernmentu vykonávané Organizáciou Spojených Národov (OSN), ktoré poskytuje hodnotenie eGovernmentu  naprieč všetkými 193 členskými štátmi. Tento prieskum hodnotí krajiny na základe Indexu rozvoja e-governmentu (E-Government Development Index, EGDI), ktorý je kombináciou primárnych dát (zbieraných a vlastnených OSN) a sekundárnych dát (získaných od iných agentúr) \scr(United Nations, 2024).

\blank
EGDI sa získava váženým priemerom troch indexov, ktoré sa týkajú týchto oblastí:

\startitemize
\item{\start\bf online služby \stop \footnote{Online Services Index (OSI)} -- hodnotenie verejných portálov na základe piatich kritérií -- \start\it inštitucionálny rámec, poskytovanie služieb, poskytovanie obsahu, technológie a digitálna účasť občanov\stop}
\item{\start\bf telekomunikačná infraštruktúra  \stop \footnote{Telecommunications Infrastructure Index (TII)} -- hodnotí úroveň rozvoja infraštruktúry nevyhnutnej pre e-vládu, vrátane pripojenia na internet, infraštruktúry širokopásmového prístupu a mobilných sietí}
\item{\start\bf ľudský kapitál \stop \footnote{Human Capital Index (HCI)} -- hodnotí vzdelanie a úroveň zručností obyvateľstva krajiny, s dôrazom na faktory ako miera gramotnosti, zapojenie do vzdelávania a dostupnosť kvalifikovaných odborníkov v oblasti ICT} % informačných a komunikačných technológií}
\stopitemize 

Na základe hodnôt indexov EGDI je možné členské štáty OSN rozčleniť do štyroch kategórií: krajiny s veľmi vysokým indexom (0,75\az1,00), krajiny s vysokým indexom (0,50\az0,75), krajiny so stredným indexom (0,25\az0,50) a krajiny s~nízkym indexom (0,00\az0,25).

Podľa najnovšieho prieskumu z roku 2022 spadá do veľmi vysokého indexu 60 krajín (31~\%), do vysokého 73 (38~\%), do stredného 53 (27,5~\%) a 7 krajín (3,5~\%) má nízky index rozvoja eGovernmentu.

\obrazekB{obr:EGDI-mapa}
{Geografické rozloženie štyroch EGDI kategórií (United Nations, 2022)}{images/EGDI-mapa.png}{width=30cc}

Medzi najvyspelejšie krajiny v oblasti elektronického vládnutia podľa Indexu rozvoja e-governmentu (EGDI) sa, podobne ako v prípade DESI, radia Dánsko a Fínsko. Česká republika sa umiestňuje na 45. pozícii, avšak stále patrí do kategórie krajín s veľmi vysokým indexom EGDI.

Porovnanie EGDI hodnôt vybraných krajín je v \in{tabuľke}[EGDI]. Na \in{obrázku}[obr:EGDI-mapa] je zobrazené geografické rozloženie jednotlivých krajín a ich úrovní EGDI.

\setupTABLE[frame=on]
\setupTABLE[row][first][background=color, backgroundcolor=gray, style=bold]
\setupTABLE[column][1][width=10cc]
\setupTABLE[column][2][width=6cc]
\setupTABLE[column][3][width=4cc]
\setupTABLE[column][4][width=4cc]
\setupTABLE[column][5][width=4cc]
\setupTABLE[column][6][width=4cc]
\setupTABLE[r][each][align={middle,lohi}]

\Tabulka{EGDI}{Porovnanie EGDI vybraných krajín s ČR (United Nations, 2022)}{
\bTABLE
  \bTR
    \bTH Krajina \eTH
    \bTH EGDI poradie\eTH
    \bTH OSI \eTH
    \bTH HCI \eTH
    \bTH TII \eTH
    \bTH EGDI \eTH
  \eTR
  \bTR
    \bTD Dánsko \eTD
    \bTD 1 \eTD
    \bTD 0.9797 \eTD
    \bTD 0.9559 \eTD
    \bTD 0.9725 \eTD
    \bTD 0.9753 \eTD
  \eTR
  \bTR
    \bTD Fínsko \eTD
    \bTD 2 \eTD
    \bTD 0.9833 \eTD
    \bTD 0.9640 \eTD
    \bTD 0.9172 \eTD
    \bTD 0.9533 \eTD
  \eTR
  \bTR
    \bTD ... \eTD
    \bTD ... \eTD
    \bTD ... \eTD
    \bTD ... \eTD
    \bTD ... \eTD
    \bTD ... \eTD
  \eTR
  \bTR
    \bTD Česká republika \eTD
    \bTD 45 \eTD
    \bTD 0.6693 \eTD
    \bTD 0.9114 \eTD
    \bTD 0.8456 \eTD
    \bTD 0.8221 \eTD
  \eTR
  \bTR
    \bTD Ukrajina \eTD
    \bTD 46 \eTD
    \bTD 0.8148 \eTD
    \bTD 0.8669 \eTD
    \bTD 0.7270 \eTD
    \bTD 0.8029 \eTD
  \eTR
  \bTR
    \bTD Slovenská republika \eTD
    \bTD 47 \eTD
    \bTD 0.7260 \eTD
    \bTD 0.8436 \eTD
    \bTD 0.8328 \eTD
    \bTD 0.8008 \eTD
  \eTR
\eTABLE
}


\ppkap{ eGovernment Benchmark}
Posledným zo spomínaných prieskumov je eGovernment Benchmark. eGovernment Benchmark monitoruje pokrok v digitalizácii verejných služieb 35 európskych krajín, známych ako EU27+ (27 členských štátov Európskej únie spolu s Islandom, Nórskom, Švajčiarskom, Albánskom, Čiernou horou, Severným Macedónskom, Srbskom a Tureckom) \scr(van der Linden, 2022).

Prieskum eGovernment Benchmark sa zameriava na tieto štyri kľúčové oblasti:

\startitemize
\item{\start\bf orientácia na užívateľa \stop -- miera poskytovania online služieb, mobile-friendly služby, online podpora a spätná väzba}
\item{\start\bf transparentnosť  \stop -- informácie o tom, ako sú poskytované služby VS, spracovaní osobných údajov a pod.}
\item{\start\bf kľúčové faktory \stop -- dostupnosť technologických faktorov v súvislosti so službami VS}
\item{\start\bf cezhraničné služby \stop -- jednoduchosť používania služieb VS pre občanov zo zahraničia a mechanizmy podpory a spätnej väzby pre takýchto občanov}
\stopitemize 

Na základe týchto štyroch oblastí získavajú krajiny tzv.\,\uv{skóre eGovernment vyspelosti}, ktorého škála sa pohybuje na stupnici od 0 do 100. Vedúcimi krajinami boli podľa posledného prieskumu z roku 2022 Malta a Estónsko. Česká republika dosiahla 22. miesto \scr(van der Linden, 2022). 


\setupTABLE[frame=on]
\setupTABLE[row][first][background=color, backgroundcolor=lightgray, style=bold]
\setupTABLE[column][1][width=12cc]
\setupTABLE[column][2][width=6cc]
\setupTABLE[column][3][width=14cc]
\setupTABLE[r][each][align={middle,lohi}]
\Tabulka{benchmark}{Porovnanie skóre eGovernment Benchmark vybraných krajín s ČR (van der Linden, 2022)}{
\bTABLE
  \bTR
    \bTH Krajina \eTH
    \bTH poradie\eTH
    \bTH eGovernment maturity score \eTH
  \eTR
  \bTR
    \bTD Malta \eTD
    \bTD 1 \eTD
    \bTD 96 \eTD
  \eTR
  \bTR
    \bTD Estónsko \eTD
    \bTD 2 \eTD
    \bTD 90 \eTD
  \eTR
  \bTR
    \bTD ... \eTD
    \bTD ... \eTD
    \bTD ... \eTD
  \eTR
  \bTR
    \bTD Česká republika \eTD
    \bTD 22 \eTD
    \bTD 63 \eTD
  \eTR
  \bTR
    \bTD Bulharsko \eTD
    \bTD 23 \eTD
    \bTD 61 \eTD
  \eTR
  \bTR
    \bTD Taliansko \eTD
    \bTD 24 \eTD
    \bTD 61 \eTD
  \eTR
  \bTR
    \bTD Chorvátsko \eTD
    \bTD 25 \eTD
    \bTD 61 \eTD
  \eTR
  \bTR
    \bTD Slovenská republika \eTD
    \bTD 26 \eTD
    \bTD 60 \eTD
  \eTR
\eTABLE
}


\obrazekW{obr:benchmark-mapa}
{Geografické rozloženie eGovernment vyspelosti podĺa prieskumu eGovernment Benchmark (van der Linden, 2022)}{images/benchmark-mapa.png}{width=30cc}

Porovnanie jednotlivých indexov vybraných krajín je v \in{tabuľke}[benchmark]. \in{Obrázok}[obr:benchmark-mapa] ilustruje geografické rozloženie krajín a ich príslušné skóre eGovernment vyspelosti.

\ppkap{Zhrnutie výsledkov prieskumov}

Analýza Indexu digitálnej ekonomiky a spoločnosti (DESI), Indexu rozvoja eGovernmentu (EGDI) a eGovernment Benchmarku ukazuje, že Česká republika dosahuje pokrok v digitalizácii verejnej správy. Napriek tomu pri porovnaní s globálnym merítkom, predovšetkým so severskými krajinami a vedúcimi členmi Európskej únie, čelí výzvam spojeným s konektivitou, integráciou digitálnych technológií a poskytovaním digitálnych verejných služieb.

V kontexte susedných krajín si ale ČR vedie pomerne dobre. Severské krajiny a vyspelé členské štáty EÚ vynikajú v inováciách a ponúkaní efektívnych a užívateľsky prívetivých digitálnych služieb. Tento fakt môže ČR použiť ako model pre zlepšovanie svojich digitálnych služieb. Významná je tiež potreba zamerania sa na cezhraničné digitálne služby, kde ČR môže opäť čerpať z príkladov zo zahraničia.

Aktuálne sa ČR nachádza v strednej časti hodnotiacich rebríčkov digitalizácie, avšak iniciatíva \uv{Digitálne Česko} (bližšie popísaná v kapitole 2.4 Digitální Česko) má potenciál posunúť ČR na prednejšie pozície v týchto prieskumoch. Predstavuje sľubný krok smerom k zlepšeniu výkonnosti Českej republiky v~digitálnom prostredí, zvýšeniu jej konkurencieschopnosti a zlepšeniu poskytovania digitálnych služieb občanom.

Projekt EMMA, ako aj jeho integrácia do systémov ERP firiem, či sprostredkovanie prostredníctvom platformy SAMO, predstavuje významný krok vpred v rámci digitalizácie českého eGovernmentu. Zameranie na digitálne verejné služby a integráciu digitálnych technológií by mohlo, v prípade preniknutia platformy EMMA na trh, viesť k posunu Česka na vyššie pozície v spomínaných rebríčkoch.


%Posunom doby sa stále viac vecí spojených s verejnou správou dá riešiť online. Zamestnávatelia sa stretávajú na pravidelnej báze s rôznymi procesmi nutnými k chodu podniku a potrebujú mať všetko v súlade so zákonmi a pravidlami verejnej správy. Aktuálne to musia všetko riešiť buď osobne na úradoch alebo v tom lepšom prípade online na rôznych portáloch verejnej správy.
%
%Procesy, ktoré sa dajú v ČR vyriešiť online sú častokrát dostupné pomocou formulárov na portáloch ministerstiev. Príkladmi takýchto portálov sú portál Ministerstva práce a sociálnych vecí (https://www.mpsv.cz/), portál Českej správy sociálneho zabezpečenia (https://www.cssz.cz/), Portál občana (https://portal.gov.cz/) a podobne. Tieto portály poskytujú rôzne formuláre, ktoré umožnia občanom vybaviť rôzne požiadavky online. Ide pri tom o komunikáciu medzi klientom a orgánom verejnej moci (OVM). Cieľom platformy EMMA od firmy Asseco je, aby tieto služby boli zapojené do interných informačných systémov podnikov. To bude viesť k tomu, že zamestnávatelia všetko vybavia na jednom mieste a nebudú musieť vypĺňať množstvo formulárov na množstve rôznych portálov.
%
%Príkladom EMMA služby je nástup zamestnanca do zamestnania. Povinnosťou zamestnávateľa pri nástupe zamestnanca je oznámiť nástup príslušným úradom a poisťovniam. Existuje viac možností, ako to urobiť, no vždy to zahŕňa viacero oddelených činností. Jednou z možností je, že musí navštíviť ePortal ČSSZ (Česká správa sociálního zabezpečení), Portál zdravotních pojišťoven a v prípade zamestnávania cudzincov aj ePortal MPSV (Ministerstva práce a sociálních věcí). Služba EMMA umožní vyplniť údaje len raz v jednom formulári a spojí sa s atomickými službami úradov. 
%
%Táto služba spolu s ďalšími bude uvedená na trh po ukončení realizácie projektu, čo môže výrazne zjednodušiť biznis procesy podnikov. 

%\pkap{Súčasný stav digitálnej verejnej správy v zahraničí}
%
%\TODO TODO: zhrnutie predchádzajúcich kapitol - z pohľadu zahraničia

%V tejto kapitole a jej podkapitolách je zanalyzovaná a popísaná situácia v krajinách, ktoré sú buď blízko Českej republike (polohou, kultúrou, rozvinutosťou...) alebo krajiny, o ktorých je známe, že nejakým spôsobom v oblasti digitálnych služieb vynikajú.
%
%\ppkap{Slovensko}
%Prvou analyzovanou krajinou je Slovensko, ako najbližší sused Českej republiky. Na Slovensku existuje množstvo elektronických služieb verejnej správy, ktoré sú dostupné prostredníctvom Ústredného portálu verejnej správy (ÚPVS). Tento portál umožňuje centrálne vykonávať elektronickú úradnú komunikáciu s rôznymi orgánmi verejnej moci a pristupovať k spoločným modulom. %Jeho dizajn je ale pomerne zastaralý a neprehľadný (viď \in{obrázok}[obr:portal\_sk]).
%
%Na základe analýzy tohto portálu je možné tvrdiť, že Slovenská republika sa zatiaľ špecializuje viac na vzťah občan-štát, ako na vzťah občan-zamestnávateľ-štát. V popredí je oblasť e-financovania, hlavne situácie týkajúce sa daní. Ministerstvo financií Slovenskej republiky v spolupráci s daňovými úradmi zaviedlo štandardizovaný proces pre elektronické prenosy faktúr a to prostredníctvom Informačného systému elektronickej fakturácie (IS EFA), ktorý umožňuje odosielanie štruktúrovaných elektronických fakturačných údajov do slovenskej daňovej správy. Používanie tohto IS bude časom povinné pre inštitúcie štátnej správy, verejnej správy a taktiež pre podnikateľov, ktorí im dodávajú tovary a služby. Tento systém je v princípe podobný systému e-kasa, ktorý prepája všetky slovenské registračné pokladnice s online portálom slovenskej daňovej správy \scr(Kilinger, 2023; MFSR, 2024). Portály eKasa a eFaktúry už podliehajú Jednotným dizajn manuálom elektronických služieb (viac informácií v \in{kapitole}[dizajn-system]).
%
%\ppkap{Poľsko}
%
%
%\ppkap{Dánsko a Estónsko}
%Denmark’s outstanding performance in e-government is the UN’s \#1 ranking in the E-Government Development Index in 2018, 2020, and 2022, as well as the highest number of citizens using e-government services in the entire European Union, with 93\% of Danish internet users using digital public services in 2021. Meanwhile, according to the 2022 Digital Economy and Society Index by European Commission, Estonia is the best performing country in the sector of digital public services, and it is recognized as a leader in e-government, outperforming Central and Eastern European countries.
%The two countries have carefully and precisely defined their digitalization goals and have pursued them. To develop a wide range of efficient and well-integrated digital public services, they have supported digital innovation, enacted laws to promote the adoption of digital services, educated the public, and engaged in public-private partnerships. These developments have resulted in significant infrastructural solutions that benefit all citizens and businesses in their daily lives when interacting with the government. In the same way that Denmark has borger.dk, Estonia has eesti.ee as an integrated public service portal, both of which are available 24/7 and protected against high demand. This robust government service delivers a superior citizen experience by meeting the four core tenets of user needs by Aarron Walters: functional, reliable, usable, and pleasurable (Queue IT, 2023).

%\obrazekB{obr:portal\_sk}
%{Ústredný portál verejnej správy SR (Slovensko.sk, 2024)}{images/portal_sk.png}{width=30cc}

%\QUES Opravdu musím psát literární rešerši? 

%\pkap[dizajn-system]{Jednotným dizajn manuálom elektronických služieb}
%haha

\pkap{Projekt EMMA}
EMMA predstavuje jedinečnú integračnú G2B2B platformu a nadväzujúce služby, ktoré sú efektívne, rýchle a \uv{zabudovateľné} do každodenných procesov klientov verejnej správy, hlavne podnikov. Tieto služby sú navrhnuté a poskytované tak, aby sa dali čo najjednoduchšie integrovať do ERP systémov podnikov\footnote{ERP (Enterprise Resource Planning) systém -- interný informačný systém podniku slúžiaci na správu rôznych činností podniku (účtovníctvo, zásobovanie, personalistika\dots)}. Zároveň by mali tieto služby podporovať podnikové procesy a zaisťovať prostredníctvom zakomponovaných služieb možnosť plniť svoje povinnosti a~vymáhať si svoje práva vočí verejnej správe. \scr(Asseco, 2023)

Úlohou integračných platforiem (alebo integration platform-as-a-service,\zlom iPaaS) je prepojiť informácie z rôznych zdrojov (z aplikácií, procesov, služieb\dots) a pripraviť tak priestor pre rýchlejšie inovácie a automatizáciu. Množstvo podnikov sa prikláňa k riešeniam iPaaS, aby zjednotili a digitalizovali podnikové operácie a mohli používať pri procesoch moderné technológie a umelú inteligenciu. Táto služba môže pomocou konektorov a API rozhraní pomôcť spoločnostiam centralizovane a automatizovane vytvárať, spravovať a monitorovať integračné toky naprieč systémami. \scr(SAP, 2024)

Medzi existujúce \uv{integračné platformy ako služby} patria napríklad SAP Integration Suite, IBM® App Connect, Workato a iné.

EMMA je prvým a jedinečným riešením v Českej republike. Jej najväčšou výhodou oproti ostatným je, že je stavaná na integráciu služieb z katalógu služieb verejnej správy.

Hlavným cieľom projektu EMMA je vybudovanie EMMA ako súhrn služieb a~riešení v oblasti podpory komunikácie komerčného sektoru s verejnou správou a začlenenie do informačných systémov firiem (ERP/FM/CRM systémy).

Cieľovými skupinami sú najmä skupiny, ktoré potrebujú informačne podporiť komunikáciu subjektov s verejnou správou hlavne v opakujúcich sa, rutinných činnostiach.  Ide hlavne o činnosti spojené s vykazovaním, ohlasovaním (za zamestnancov alebo klietov). 

%Ako sa píše v článku od Anderssona z roku 2022, každá práca môže byť automatizovaná (respektíve digitalizovaná) len vtedy, ak môže byť vizuálne reprezentovaná a preložiteľná do algoritmických inštrukcií pre počítač. Spomínané činnosti reprezentovateľné sú, a tak môžu byť digitalizované resp. automatizované.

\blank
Medzi konkrétnych cieľových užívateľov patria napríklad personalisti, účtovní a daňoví pracovníci, banky, poisťovne a pod. Potenciálnych zákazníkov možno rozdeliť do dvoch skupín: 

\startitemize[a]
\item{zákazníci, ktorí nemajú žiaden plnohodnotný ERP systém,}
\item{zákazníci, ktorí zvažujú zmenu/upgrade používaného ERP systému.}
\stopitemize

Projekt nadväzuje na Architektonický princíp č. 11: eGovernment jako platforma (Embedded eGovernment) uvedený v IKČR. Architektonické princípy IKČR sú spomenuté v kapitole 2.2. V skratke princíp č. 11 hovorí o tom, že procesy a služby verejnej správy aj s potrebnými technickými nástrojmi musia byť navrhnuté tak, aby organizácie mohli tieto služby jednoducho integrovať do svojich ICT systémov, čo im uľahčí plnenie povinností a využívanie práv voči verejnej správe \scr(Digitální a informační agentura, 2023).

Okrem tohto princípu sa EMMA riadi aj ďalšími procesnými zásadami ustanovenými v IKČR, napr. Z6 Riadenie výkonnosti a kvality, Z7 Riadenie zodpovednosti za služby a systémy, Z11 Riadenie prínosov a hodnoty a ďalšie.

%\startitemize
%\item \start\bf Z6 Riadenie výkonnosti a kvality \stop --  meranie výkonnosti a kvality, princípy merateľnosti a spätnej väzby, pravidelné audity
%\item \start\bf Z7 Riadenie zodpovednosti za služby a systémy \stop -- každý proces a~služba musí mať svojho vlastníka a garanta
%\item  \start\bf Z8 Riadenie ICT služieb \stop -- IT podpora riadená katalógom ICT služieb pre interné a externé procesy
%\item \start\bf Z11 Riadenie prínosov a hodnoty \stop -- rozhodovanie založené na ekonomickej výhodnosti, zahŕňa analýzu nákladov, rizík a prínosov, nutnosť spracovania investičného zámeru
%\item \start\bf Z16 Využívanie otvoreného software a štandardov \stop -- preferencia otvoreného softvéru a štandardov, podpora udržateľnosti, rozvoja a bezpečnosti  \scr(Digitální a informační agentura, 2023)
%\stopitemize

Projekt EMMA môže okrem ekonomických prínosov priniesť aj neekonomické a to ako pre klientov, tak i pre ČR a EÚ. Dajú sa identifikovať napr. ako reputačný prínos ČR, prínos k riešeniu spoločenských výziev EÚ, rozvoj ľudského potenciálu a pod. Zámer projektu zároveň prispieva k naplneniu cieľov Národnej inovačnej stratégie.

Platforma EMMA poskytuje služby VS pomocou štandardizovaného API
%\footnote{API (Application Programme Interface) -- webové rozhranie, ktoré umožňuje komunikáciu medzi dvomi rôznymi aplikáciami}
, ktoré je jednoducho integrovateľné do ERP systémov. Podniky môžu využívaním platformy EMMA dosiahnuť zníženie administrátorskej záťaže podnikov. 

Dá sa povedať, že ide o G2B platformu na sprostredkovanie a zaistenie vložiteľnosti služieb VS do informačných systémov a zároveň B2B platforma na poskytovanie týchto služieb.

Príkladom služby EMMA je \uv{oznámenie o nástupu zamestnanca}. Pri tejto životnej situácii je podnik povinný informovať viaceré subjekty VS, konkrétne ČSSZ, zdravotné poisťovne a MPSV (v prípade zahraničného zamestnanca). Bližšie je tento proces popísaný na \in{obrázku}[obr:sekv1].

\obrazekH{obr:sekv1}
{Sekvenčný model funkčnosti Oznámení nástupu zaměstnace" (Asseco, 2023)}{images/SekvEMMA01.png}{width=32cc}

Obsahom platformy EMMA sú:

\startitemize
\item{interné služby na správu a prevádzku platformy,}
\item{nástroje na využívanie služby prostredníctvom Rozhrania na volanie služieb VS,}
\item{nástroje pre interoperabilitu VS ČR v legislativnom rámci Digital Service Act\footnote{Digital Service Act (Akt o digitálnych službách) -- súbor pravidiel platiacich v celej EÚ, ktorých cieľom je vytvoriť bezpečnejší digitálny priestor, v ktorom budú chránené základné práva všetkých užívateľov digitálnych služieb \scr(Európska komisia, 2022)} a Data Governance Act\footnote{Data Governance Act (Akt o správe dát) -- úsilie zvýšiť dôveru v zdieľanie dát a posilnenie mechanizmov pre zvýšenie dostupnosti dát \scr(Európska komisia, 2022)},}
\item{modul rozhrania pre G2B2B,}
\item{služby API pre integráciu.}
\stopitemize 

%Na \in{obrázku}[obr:emma-koncept] možno vidieť celkový koncept platformy EMMA. 

Na \in{obrázku}[obr:archimate] je zobrazený model aplikačnej architektúry, ktorý popisuje štruktúru, správanie a interakcie v aplikácii. Bol vytvorený pomocou štandardu Archimate, keďže bol pre tieto účely najvhodnejší. Obsahuje rôzne elementy ako procesy, služby, interakcie, rozhrania a podobne.

%\obrazekH{obr:emma-koncept}
%{EMMA - Koncept (Asseco,2023)}{images/EMMAkoncept.png}{width=32cc}

\pkap{Platforma SAMO}

SAMO je platforma firmy Asseco, pomocou ktorej je vybudované veľké množstvo aplikácií verejnej správy. V súčasnosti slúži aj ako prezentačná vsrtva projektu EMMA.

Názov SAMO vznikol skrátením slov Strategic Asset Management \& Operations system, čo v preklade znamená systém pre strategickú správu majetku. Vývoj jednotlivých modulov začal v roku 1991. Počas posledných vyše 30-tich rokov vývoja sa firme Asseco podarilo do rôznych riešení zapojiť množstvo skúseností a best practices. \scr(Asseco, 2024b)

Pôvodne bolo SAMO vyvinuté ako platforma zameraná hlavne na geografické informačné systémy, čo bol ideálny základ najmä pre spoločnosti spravujúce mestskú infraštruktúru a distribučné siete.

Platforma SAMO sa dá konceptuálne rozdeliť na dve hlavné časti – evidenčnú a priestorovú. Evidenčná časť sa zaoberá evidenciou majetku a jeho charakteristík, ako sú napr. posledné kontroly či opravy. Priestorová zložka obsahuje geometrické údaje, mapové informácie a podobne.

Postupne bol systém rozšírený o procesnú zložku, ktorá zahŕňa riadenie procesov ako sú napríklad hlásenie udalostí či plánovanie údržby. Vznikajú tak agendy na správu majetku, ktoré sa skladajú zo zoznamu entít, editačných formulárov, detailov, stavových diagramov a iných prvkov. 

Vzhľadom na to, že každý zákazník má špecifické potreby, SAMO sa neponúka ako finálny produkt, ale len ako flexibilná platforma zložená z rôznych komponent, ktoré sa skladajú podľa individuálnych požiadaviek zákazníka. Toto prispôsobenie a možnosť agilného vývoja projektov je konkurenčnou výhodou SAMO v oblasti verejnej správy. Vďaka modularite a možnosti znovupoužitia existujúcich metadát a komponent je SAMO vhodné najmä pre unikátne agendy evidenčného charakteru, ktoré majú nejakú GIS zložku, pričom GIS zložka ale nie je podmienkou vhodnosti použitia SAMO. 

Pri implementácii SAMO aplikácie je nutné meniť hlavne business zložku jednotlivých systémov (logiku akcií a procesov).

\blank
Platforma SAMO má 3 základné moduly:

\startitemize
\item \start\bf SAMO EAM \stop (Enterprise Asset Management) --  správa podnikového majetku
\item \start\bf SAMO AIS \stop (Agendový IS) -- procesy verejnej správy
\item  \start\bf SAMO LIDS/GIS \stop -- geografický informačný systém
\stopitemize

Platforma SAMO je základom pre široké spektrum aplikácií používaných v~rozličných sektoroch – od priemyslu a energetiky až po verejnú správu a koncepty inteligentných miest. V Českej republike sa na nej zakladajú projekty pre významné inštitúcie, ako sú Český banský úrad, Český rybársky zväz, Agentúra ochrany prírody a krajiny ČR, čo potvrdzuje jej flexibilitu a široké využitie.


V kontexte tejto práce je cieľom rozšírenie agendového systému (SAMO AIS) na správu zamestnancov a služieb verejnej správy.

SAMO AIS predstavuje špecifický modul na podporu procesov verejnej sprá\-vy. Zahŕňa zadávanie požiadavkov, vyhodnocovanie workflow, notifikácie, analýzu dát, pridávanie priestorových informácií atď. Tento modul je navrhnutý tak, aby bol kompatibilný s inými systémami verejnej správy, využíval otvorené dáta z~rôznych zdrojov (napr. registry, katastre) a zároveň poskytoval informácie podľa potrieb koncových užívateľov. Je vhodný pre miestne, ústredné, ale i federálne orgány akéhokoľvek druhu (od malých obcí až po ministerstvá). Cieľom je vybudovať a udržať efektívny eGovernment a prinášať hodnotu užívateľom a občanom. \scr(Asseco, 2024b)

Platforma SAMO je založená na modulárnej architektúre integrovaných softvérových riešení. Vychádza z princípov SOA (Servisne orientovaná architektúra), čo znamená, že je navrhnutá pre efektívnu vzájomnú spoluprácu nezávislých komponent. % \footnote{SOA -- Servisne orientovaná architektúra (Service Oriented Architecture) -- sada princípov a metodológií, ktorá odporúča stavbu aplikácií zo vzájomne nezávislých komponent}. %Považuje sa za platformu určenú na efektívne poskytovanie, nie za produkt samotný.

SAMO používa mikroservice prístup k vývoju softvéru. Každá mikroslužba je zameraná na konkrétnu funkčnosť a môže byť vyvíjaná, nasadená a spravovaná nezávisle od ostatných častí aplikácie. To umožňuje flexibilnejšie škálovanie, rýchlejšie nasadzovanie nových funkcionalít a jednoduchšiu údržbu. %SAMO je aplikácia poháňaná metadátami, ktoré obsahujú informácie o vzťahoch medzi časťami  dát, popis užívateľského rozhrania, pravidlá, formuláre a pod. \scr(Asseco, 2023) 

\blank
Systém kladie dôraz na integritu dát a procesov. Architektúru tvoria základ\-né vrstvy, ktoré sú na sebe technologicky nezávislé a ich komunikáciu zabezpečujú štandardy popísané API. Ide o tieto vrstvy:

\startitemize[n]
\item \start\bf prezentačná vrstva \stop -- skupina webových serverov, ktoré poskytujú služby prezentačnej vrstvy pre interných i externých pracovníkov
\item \start\bf aplikačná vrstva \stop -- skupina aplikačných serverov založených na platforme J2EE, na ktorých je implementovaná biznis logika
\item  \start\bf databázová vrstva \stop -- skupina databázových serverov s požadovanou výkonnosťou zabezpečujúcich služby dátového úložiska pre aplikačnú vrstvu
\stopitemize

Aplikačný interface, SAMO API Gateway, umožňuje definovať špecifické služby podľa účelu použitia. Na základe konfigurácie sú vytvárané kompozitné služby, ktoré minimalizujú klient\ip server volania. Technológie API Gateway obsahujú aj autentizačnú a autorizačnú vrstvu, ktorá zaisťuje riadenie prístupu k~službám a dátam, ktoré služby poskytujú.

Dáta sú spravované systémom SAMO. Aplikačná vrstva sprostredkovává prístup do relačnej databázy a často vyhľadávané dáta sú uložené redundantne aj v dokumentovej databáze ElasticSearch. 

Dátový model je definovaný metadatovým predpisom vrátane spôsobu uloženia, definície atribútov a väzieb medzi entitami. Popis modelu obsahuje taktiež informáciu o tom, či entita obsahuje priestorové dáta. Je zapojená aj historizácia zmien na záznamoch a ich vzťahoch.

Systém ponúka integračné rozhrania, ktoré je primárne určené na integráciu systému s externými informačnými systémami. Systém poskytuje funkcionalitu vstupného a výstupného SOAP interface, vstupného a výstupného REST interface, validácie dát, zabezpečenie komunikácie a dát atď. Integračné väzby sú riešené pomocou webových služieb nad integračnou platformou ESB (Enterprise service bus). 

Vývoj prebieha na nainštalovanom lokálnom prostredí, pričom databáza je ale serverová. Na lokále bežia 2 konzoly -- GTW a LIDS (metadáta) a pripája sa to na databázu, elastic search a user service na server. Po úpravách sa zmeny commitujú a pushujú pomocou Gitlabu a beží CI/CD (na server sa to dostane až keď je CI/CD ok, bez failu). Významný nástroj, ktorý je dôležitý na rozbehnutie lokálneho prostredia je utility localtron. 

Súborová štruktúra je rozdelená na 2 väčšie celky -- configuration a project. V časti project sú uložené rôzne parametre ako verzia, prístup do databáze, informácia o aktuálnom prostredí (vývojové, testovacie, produkčné a pod.). V configuration sú už samotné metadátové súbory a aplikačná logika. (Mega, 2024)

\obrazekH{obr:samo-koncept-arch}
{Konceptuálna aplikačná architektúra SAMO (Asseco, 2023)}{images/samo-konceptualni-aplikacni-architektura.png}{width=32cc}

\ppkap{Aplikačná architektúra}
SAMO sa z hľadiska aplikačnej architektúry skladá z aplikačných komponent, ktoré medzi sebou komunikujú cez predom dohodnuté komunikačné kanály založené na otvorených a všeobecne uznávaných štandardoch (hlavne\zlom REST, SOAP). Aplikačné komponenty sa skladajú z aplikačných vrstiev (užívateľské rozhranie, biznis logika, integračná vrstva, dátová vrstva a iné).

Riešenie je budované na princípoch a postupoch servisne orientovanej integrácie, kde jednotlivé zdieľané funkcionality sú vystavené v podobe služieb. Vďaka aplikačnej a technologickej architektúre systému sa aplikačná logika vykonávaná na serveri a klientská aplikačná logika vzájomne neovplyvňujú. Architektúra je zobrazená na \in{obrázku}[obr:samo-koncept-arch].

Aplikačná logika systému je písaná v jazyku JavaScript a využíva bohatú podporu funkcií platformy, ako je práca s entitami, väzbami, dokumentmi, klientskými dátami. Ďalej obsahuje podporu pre prácu s transakciami, indexáciu do ElasticSearch, komunikáciu s notifikačným modulom a mnoho ďalších funkcií. \scr(Mega, 2024)


%\TODO prepojenie SAMO a LIDS atd


%\TODO komponenta správa číselníkov

\ppkap{SAMO Dynamic Application}
Klientská čast, SAMO Dynamic Application, je ľahký webový klient, ktorý komunikuje so serverovými komponentami ako SAMO Gateway, LIDS Application Server a Security Server a pod. V rámci celkového riešenia SAMO sú zapojené aj ďalšie technológie ako Docker, ElasticSearch, PostgreSQL a NGINX, čo zvyšuje efektivitu a flexibilitu systému.

Dynamic app sa skladá z niekoľkých hlavných modulov. Medzi ne patrí takzvaný cockpit, ktorý predstavuje úvodnú obrazovku, úvodný rozcestník. Ďalšou komponentou sú tzv.\,pages, ktoré obsahujú zoznamy entít. Tieto zoznamy sú označované ako browse. 

Po rozkliknutí entity z browse sa zobrazí detail, ktorý môže obsahovať okrem hlavičkových dát aj sekcie obsahujúce ďalšie naviazané entity. Na detailoch je možnosť editácie prostredníctvom editačného detailu,  ktorý je poslednou hlavnou komponentou DA. Umožňuje modifikovanie hlavičkových informácií o entite, avšak nie dát v spomínaných sekciách. Vizuálne zobrazenie komponent je v kapitole 4.3.1.

%\TODO Skladá sa z dvoch hlavných podprojektov, ktorými sú emma-samo-configuration a emma-samo-project

\ppkap{LIDS Application Server}

LIDS časť systému SAMO je oveľa väčšia ako spomínaná SAMO Gateway. Riadi všetkú logiku systému -- správa dát, prístupy k aplikačnej logike a dátam, security, REST API a pod. SAMO Gateway slúži primárne na to, aby poskytovala metadáta pre Dynamic App. Preto je LIDS popísaný detailnejšie v samostatnej podkapitole.

LIDS aplikačný server je kontrolovaný metadátami, s ktorými pracujú jednotlivé časti systému. 

Hlavnou stavebnou jednotkou LIDS metadát sú tzv.\,feature types. Ide v podstate o nejaký typ objektu reálneho sveta (napr. ft\_osoba, ft\_adresa, ft\_zamestnanec\dots). Feature type definuje atribúty objektu, môže definovať aj geometriu, symboliku a iné vlastnosti.

Ďalšou časťou LIDS AS je tzv\, feature. Ten je inštanciou feature typu, a teda je to reprezentácia objektu reálneho sveta. Feature nesie informácie o tom, aký je to feature type, sémantické atribúty (id, name, type), jeho miesto v databáze (tzv.\,databázový kontajner), poprípade symboliku a typ geometrie.

Tieto metadáta sú uložené, prenášané a spravované vo forme niekoľkých XML dokumentov, pričom najdôležitejší je dokument s názvom model.xml. Všet\-ky tieto XML dokumenty majú pevne danú štruktúru popísanú v súboroch typu XSD (XML Schema Definiton). \scr(Asseco, 2024a)

%\startitemize
%\item{\start \bf model.xml \stop -- hlavný metadátový súbor, v ktorom sú uložené informácie o tzv.\,feature types (entita SAMO systému), ich atribútoch, číselníkoch a pod.}
%\item{\start \bf presentation.xml \stop -- definuje predvolenú symboliku projektu a pod.}
%%\item{\start \bf thematization.xml \stop - definuje dynamickú symboliku prvkov (na základe štandardu OpenGIS Symbology Encoding)}
%\item{\start \bf tool.xml \stop -- definuje panely nástrojov špecifických pre projekt}
%\item{\start \bf resource.xml \stop -- definuje napr. štýly čiar, symboly, fonty, ikony a pod.}
%\item{\start \bf option.xml \stop -- definuje voliteľné funkcie systému ako napr. kopírovanie prvkov, derivovanie atribútov, zobraziteľné atribúty\dots}
%\stopitemize 
%
%Okrem týchto hlavných XML súborov existuje aj množstvo ďalších. Všetky spomínané súbory majú pevne danú štruktúru popísanú v súboroch typu XSD (XML Schema Definiton). 

Základom pre budovanie aplikácie je vytvorenie dátového modelu, na ktorom sa celá aplikácia buduje a logika sa zapája až potom. 

LIDS dokáže na základe validného model.xml modifikovať databázu (vytvárať nové tabuľky, atribúty, meniť dátové typy a pod.). Spúšta sa to v administrátorskej konzole, kde sa porovnáva existujúca databáza s xml modelom a vygeneruje sa SQL skript s potrebnými príkazmi a po potvrdení sa to do pustí do databázy.

V administrátorskej konzoli existuje aj GUI, ktoré prehľadne zobrazuje všetky feature types, názov kontajneru (db tabuľky) daného ft, atribúty, väzby, stavový diagram (workflow, ak existuje), akcie (operácie, metódie, funkcie) a podobne. 

Okrem dátovej časti (model.xml) obsahuje LIDS-ová časť aj aplikačnú logiku. Ide o niekoľko javascript a json súborov definujúcich stavy, akcie nad entitov a~celková potrebná logika správania danej entity.

%Poslednou vrstvou je security vrstva, ktorá určuje prístupy k feature typom, atribútom, akciám alebo k riadkom browsu (napr. na základne lokality). Buď je to na základe security číselníku alebo druhá možnosť je nastavenie ownershipu (vlastníctva) na vybrané záznamy (napr. pri žiadostiach na základe funkcí a pod.). 

\ppkap{Security Server}
Na správu identifikačných údajov uživateľov SAMO je používaný systém Security Server. Prístup do správcovskej aplikácie je umožnený prostredníctvom webového rozhrania. Security Manager je aplikácia určená na definíciu užívateľov vrátane ich prístupových oprávnení. 

Aplikácia je štruktúrovaná do niekoľkých funkčných oblastí, vrátane správy užívateľov, ich rolí a oprávnení. SAMO Security Server využíva tzv.\,Role Based Access Control (RBAC) mechanizmus. Tento mechanizmus umožňuje to, že každý užívateľ pristupuje k systému v definovanej role. Ku každej roli je možné nastaviť ľubovoľný počet oprávnení a užívateľ môže byť uvedený i vo viacerých rolách. Konečné oprávnenie je vyhodnocované ako zjednotenie všetkých oprávnení. \scr(Asseco, 2024a)

Security rola určuje napr. práva na dlaždice, tlačítka a pod. Tieto role sa priradia vybraným skupinám a do skupín sa priradia užívatelia. To zabezpečí, že prihlásený užívateľ má umožnené v aplikácii vidieť a robiť len to, na čo má oprávnenie.

Užívatelia zvyčajne prichádzajú pomocou LDAP od zákazníka. Security skupiny určujú práva na feature types. Zvyčajne ide o skupinu read, edit a admin, (napr. ZAMESTNANCI-read, ZAMESTNANCI-edit, ZAMESTNANCI-admin), no je možné vytvárať aj špeciálne skupiny. Sú určené na to, aby boli užívatelia zaradení do už spomínanej skupiny oprávnení.

%Okrem agendového systému prináša aj štatistiky, funkcie pre publikáciu a integračnú vrstvu. Súčasťou je aj workflow engine a API gateway, čo podporuje flexibilitu systému. 

\pkap{Nástroje na optimalizáciu práce so SAMO}
\ppkap{EA2LIDS}
Nástroj EA2LIDS je vlastný nástroj firmy Asseco, ktorý slúži na generovanie dátového modelu (model.xml) z modelu v Enterprise Architect. Je to tzv.\,Model Driven Generation Technology (MDG). 

MDG technológie umožňujú rozširovať funkcionalitu programu Enterprise Architect (ďalej ako EA) prostredníctvom špeciálnych rozšírení \scr(Sparx, 2024). 

Existuje množstvo komerčných MDG technológií, ale je možné aj použitie vlastnej, ako je to aj v prípade EA2LIDS. Okrem existujúcich modelov v EA je možnosť rozšírenia základných štruktúr napríklad o tzv.\,tagged values, stereotypes, profiles, design patterns a podobne. 

V prípade EA2LIDS sa pracuje so špeciálnym typom diagramu nazývaným LIDS7 a príslušiacim toolboxom v EA (viď \in{obrázok}[obr:special-ea]). Pre správne využitie a funkčnosť nástroja je potreba dbať na správnosť modelov -- správne vybrané stereotypy, nadefinované dátové typy, správna menná konvencia, tagy atď. 

Po dokončení modelu je možnosť generovať súbor formátu XML. Po kontrole a prípadných ručných úpravách je výsledkom validný model.xml (zjednodušená schéma fungovania je zobrazená na \in{obrázku}[obr:ea2lids]). Tento model potom používa nástroj popísaný v nasledujúcej kapitole, AMK.

\obrazekH{obr:special-ea}
{Špeciálny toolbox a diagram v nástroji Enterprise architect (vlastné spracovanie)}{images/special-ea.png}{width=32cc}

\obrazekH{obr:ea2lids}
{EA2LIDS v praxi (vlastné spracovanie)}{images/ea2lids.png}{width=32cc}

\ppkap{AMK}
Platforma SAMO umožňuje vytvárať evidenčné agendové aplikácie, ktoré sa tvoria pomocou konfiguračných metadát. Do týchto metadát sa ukladajú informácie ako dátový model, štruktúra obrazoviek, formulárov atď.

Pre účely urýchlenia implementácie SAMO aplikácií bola vyvinutá sada nástrojov označená ako AMK (Application Modling Kit). AMK je koncept a definovaný spôsob práce, súbor metodík a nástrojov. 

AMK je určený pre agendové aplikácie, teda aplikácie, kde hlavnú úlohu hrá evedincia artefaktov a workflow nad evidenciou. Zaoberá sa spôsobom predávania informácií medzi analýzou a výrobou v štrukturovanej podobe tak, aby štruktúra vyhovovala analytikovi i implementácii.

Logika využitia AMK je na \in{obrázku}[obr:amk]. AMK možno popísať ako most spájajúci analýzu s implementáciou. Na \in{obrázku}[obr:amk-archi] je zobrazený model aplikačnej architektúry AMK.

Na generovanie potrebných súborov SAMO aplikácie (tzv.\,metadát) sa využíva nástroj FMPP (FreeMarker-based file PreProcessor). Ide o nástroj na predspracovanie textu, ktorý umožňuje dynamické generovanie obsahu pomocou šablón FreeMarker. Je schopný rekurzívne spracovať adresáre a podporuje generovanie statických webových stránok, zdrojového kódu, konfiguračných súborov a podobne. Dokáže integrovať s dátovými zdrojmi ako súbory typu CSV, XML alebo JSON. \scr(FMPP, 2018)

\obrazekH{obr:amk}
{AMK v praxi (vlastné spracovanie na základe informácií z interných dokumentov firmy Asseco)}{images/amk.png}{width=32cc}

\obrazekH{obr:amk-archi}
{AMK -- Časť aplikačnej logiky (vlastné spracovanie)}{images/amk-archi.png}{width=32cc}

\zlom
\pkap{Súčasný stav projektu}
V čase začiatku práce na tejto diplomovej práci už bola časť projektu EMMA hotová, ale  jeho ďalší aktívny vývoj a optimalizácia boli v spoločnosti dočasne odložené. Stále ale je množstvo príležitostí na vylepšovanie, ktoré by umožnili jeho komerčnú ponuku zákazníkom a rozšírenie medzi rozličné subjekty. Dôvodom pozastavenia je najmä prioritizácia iných projektov.

Cieľom platformy EMMA je sprostredkovávať služby verejnej správy podnikateľským subjektom. V aktuálnom stave nemá zákazníka a je pokladaná za \uv{výskumný projekt}.

Na diagrame nižšie (\in{obrázok}[obr:koncept]) je zobrazená enterprise architektúra platformy. V ľavej časti sú ilustrované orgány verejnej správy (úrady, obce a pod.), ktoré vystavujú služby umožňujúce prístup k údajom a informáciám, a tiež slúžia na splnenie povinností klientov verejnej správy.

Hoci existuje Register práv a povinností, ktorý obsahuje katalóg služieb verejnej správy s cieľom zmapovať všetky služby verejnej správy a verejne poskytovať informácie o týchto službách, často to nestačí. Dôvodom je, že samotné služby jednotlivých úradov nie sú konsolidované (nemajú jednotné API) a nie sú poskytované prostredníctvom jednej platformy.

Pre realizáciu služieb musí platforma EMMA obsahovať podporné komponenty ako je Katalóg služieb EMMA, Portál pre prístup klientom a rôzne SW komponenty na sprostredkovanie služieb (zbernice, API management, správa oprávnění a iné). 

Klienti EMMA (pravá časť diagramu z \in{obr.}[obr:koncept]), teda podnikateľské subjekty, môžu využívať služby EMMA pre jednoduché zapojenie do svojich ERP systémov a tým znížt množstvo ručne vykonávaných činností vďaka automatizácii.

Aktuálne je EMMA nasadená na cloudovom prostredí Azure. Konkrétne sú využívané služby ako Application Insights pre sledovanie behu aplikácie, Key vault na ukladanie hesiel, Smart detector alert rule na detekciu anomálií a iné. 

Na demonštráciu fungovania platformy potenciálnym zákazníkom sa používa aplikácia založená na platforme SAMO. Aj EMMA aj SAMO sú platformy spoločnosti Asseco. SAMO je základom pre evidenciu služieb EMMA.

Pôvodným zámerom bola len správa katalógu služieb verejnej správy, ohlasovateľov, poskytovateľov a podobne. Časom sa to ale začalo rozširovať a vznikol modul \uv{Žádosti} a konkrétne \uv{Ohlášení nástupu zaměstnance}. V súčasnom stave to spočíva vo vyplnení jednoduchého formulára s textovými, prípadne číselnými údajmi, ktorý sa vyplní potrebnými údajmi pre ČSSZ a ZP. Údaje sa týkajú zamestnanca, zamestnania aj zamestnávateľa. Následne sa pomocou API služieb odošle na príslušné úrady a overí sa.

Tento proces nedostatočný najmä kvôli nedostatku validácií. Všetky položky, ktoré sa do žiadosti vypĺňajú sú len textové položky, čo môže viesť k rôznym chybám pri spracovaní žiadosti. Zároveň tam nie sú vytvorené ani používané žiadne číselníky, ktoré by vedeli výrazne uľahčiť vypĺňanie formulárov a žiadostí a predišlo by sa spomínaným chybám.

Požiadavkom firmy Asseco je to, aby bolo možné spravovať prostredníctvom SAMO EMMA zamestnancov, osoby a vybrané číselníky. To umožní jednoduchú prácu s potenciálnymi ďalšími službami EMMA.  Tieto procesy výrazne zefektívnia a zjednodušia úkony smerované k verejnej správe. Dôsledkom bude, že komunikácia s úradmi bude prebiehať takzvane \uv{na jeden klik}.

Ďalším nedostatkom stávajúcej aplikácie je jej grafická stránka, konkrétne obrázky a ikony, ktoré boli prevzaté z iného projektu. Ukázali sa však ako nedostatočne reprezentatívne a adekvátne. Je teda nutné ich upraviť alebo vytvoriť úplne nové, aby lepšie vyhovovali potrebám a kontextu tejto aplikácie.


\obrazekB{obr:koncept}
{Enterprise architektúra platformy EMMA  (Asseco, 2023)}{images/koncept.png}{width=32cc}

\kap{Metodika}

Informácie používané pri vypracovaní diplomovej práce sú získavané najmä z interných dokumentácií a konzultácií so zamestnancami zapojenými do projektu. Pre hlbšie pochopenie témy je potrebná aj analýza eGovernmentu a súvisiacich procesov v Českej republike a v zahraničí. Existuje aj množstvo kľúčových pojmov, ktoré taktiež musia byť pochopené, napr. služba verejnej správy, katalóg služieb verejnej správy, agendový systém a pod. Tieto časti sú popísané v kapitole 2 Súčasný stav.

Používaná literatúra pochádza z dôveryhodných zdrojov, ktoré sú dostupné na Google Scholar alebo Web of Science. Zároveň sú používané aj rôzne dokumenty zverejnené národnými a nadnárodnými inštitúciami. Na overovanie a dopĺňanie niektorých informácií bola používaná pre uľahčenie práce umelá inteligencia.
%Na analýzu a tvorbu diagramov je používaný program Enterprise architect. Dizajnové prvky sú vytvárané v nástroji Figma. Hlavnou používanou technológiou je platforma SAMO a používaný databázový systém je PostgreSQL. 

V aktuálnom stave je síce projekt v rámci firmy pozastavený, ale stále existujú požiadavky a ciele, ktoré je nutné zapojiť. Po konzultáciách s vedením projektu vrámci spoločnosti Asseco je vybraté rozšírenie existujúcej aplikácie najmä o modul zamestnancov. Zároveň sa predpokladá aj úprava existujúceho riešenia, keďže pôvodné riešenie vykazuje nedostatky v dátovom modeli a celkovej implementácii, ktoré je potrebné poupraviť a vylepšiť. Tieto nedostatky sú popísané v kapitole 2.9 Súčasný stav projektu.

Pri tvorbe tohto projektu je uprednostňovaná agilná metodológia a pravidelné konzultácie so \uv{zákazníkom}, ktorým je v tomto prípade firma Asseco. Konzultácie prebiehali spočiatku na mesačnej báze, neskôr boli častejšie. Práca je rozdelená do viacerých pomyselných šprintov, zahrňujúcich všetky aspekty projektu -- od formálnej dokumentácie, cez analýzu, vývoj, až po testovanie.

Počiatky agilnej metodiky siahajú do 90. rokov 20. storočia. Dovtedy dominoval predovšetkým vodopádový model.

Pri vodopáde ide o podrobné zadefinovanie celého projektu, ktoré pozostáva z analýzy požiadaviek, návrhu, implementácie, testovania, nasadenia a prevádzky. Fázy nasledujú za sebou a po ukončení každej fázy je spracovaná dokumentácia a report. Víziou totho prístupu je dostať procesy pod úplnú kontrolu a~zamedziť vzniku chýb. Avšak, ukázalo sa, že tento prístup často vedie k zbytočnej byrokracii a nie je taký hladký, ako sa predpokladalo. Prelom nastal koncom 90. rokov, kedy sa začali formovať flexibilnejšie metodológie. Tento počin sa nazýva aj \uv{The Agile Manifesto}. \scr(Shore,2022)

\blank
Agilný prístup je veľmi adaptívny, s hlavnými princípmi zahŕňajúcimi:

\startitemize
\item zameranie na uspokojenie zákazníka, 
\item otvorenosť voči zmenám požiadaviek,
\item postupná dodávka software,
\item úzka spolupráca medzi obchodným tímom a vývojármi. \scr(Shore,2022) 
\stopitemize

V prvom desaťročí po zavedení agilných metodológií panovali pochybnosti o ich efektivite. Napriek tomu sa agilný vývoj ukázal ako úspešný a jeho popularita stále rastie, čo potvrdzuje aj stúpajúci záujem o agilné prístupy pri rôznych typoch projektov, vrátane tých v oblasti eGovernmentu. Dôvodom sú najmä časté zmeny požiadaviek, respektíve nejasné požiadavky. Práve kvôli častým zmenám požiadaviek v takýchto projektoch je agilný prístup čoraz viac uprednostňovaný.

Projekty spojené s digitalizáciou a eGovernmentom (ale aj iné) zlyhávajú najmä z dôvodu nedostatočnej komunikácie a nejasných požiadaviek. \scr(Looks, 2021) Aj to je dôvodom výberu agilnej metodiky vývoja pre túto prácu.

Na analýzu a návrh je používaný predovšetkým jazyk UML, avšak pre detailnejšie zobrazenie aplikačnej logiky bolo nutné využiť aj jazyk ArchiMate. Modely sú vytvárané pomocou nástroja Enterprise Architect. Tento nástroj je okrem zabudovaných možností rozšírený o špeciálny typ diagramu -- LIDS7, pomocou ktorého je vytváraný dátový model. Tento dátový model sa pomocou nástrojov EA2LIDS a AMK použije ako základ pre vyvíjanú aplikáciu a prvý krok implementačnej linky. 

Grafické prvky a návrhy aplikácie sú vytvárané v nástrojoch Figma a Canva.

Na implementáciu je použitá platforma SAMO. Teoretické základy a praktické porozumenie implementácie platformy SAMO, špecifického \uv{frameworku} firmy Asseco, sú podrobnejšie rozpracované v kapitole 2.7 Platforma SAMO. Písanie kódu prebieha pomocou nástroja Visual Studio Code.

Čo sa týka databázového systému, aplikácia využíva PostgreSQL. Ako databázové GUI je používaný software DBeaver.

Aplikácia je nasadená v cloudovom prostredí Azure. Testovanie aplikácie prebieha manuálne na základe predpripravených testovacích scenárov.

%\TODO EBSI (https://ec.europa.eu/digital-building-blocks/sites/display/EBSI/Home) (https://assecoce.sharepoint.com/:w:/r/teams/EMMAPEGOV-Analza/Shared%20Documents/Anal%C3%BDza/80%20-%20V%C3%BDstupy/Etapa1%20-%20N%C3%A1vrh%20%C5%99e%C5%A1en%C3%AD/E1.4%20Enterprise%20architektura/EMMA_E1_Enterprise_architektura.docx?d=wa8731ee84cc84c9b9e26b2ac55ed06bf&csf=1&web=1&e=DRJrfT)

%
%\obrazek{obr:amk-situace2}
%{Schéma AMK v praxi  2 (vlastné spracovanie na základe informácií z interných dokumentov firmy Asseco)}{images/amk-situace2.png}{width=30cc}

%\pkap{Popis metodiky, analýzy a návrhu}

%\TODO


%\kap{Poznámky z konzultácie}
%- kompetenčný model agendy VS
%
%- centrálne zdielane služby
%
%- porovnanie voči svetu - u nás (v čr) sú prenesená pusobnost a spravna pusobnost obcí
%stavebné řízení - obec povoluje schvalovanie stavieb, ale obec môže byť dotknutá riadením (takže je vlastne schvalovatel aj účastník řízení)
%
%- informačný koncepce, národný arch plán a rámec
%
%- pôvodný zámer EMMA (vize a tak)
%
%- právo na digitálne služby -> malo by priniest zlepšienie egov
%
%- Jirka ohladom SAMO

\kap{Výsledky}
Výsledky sú realizované podľa princípov metodiky popísaných v kapitole 3 Metodika. Na začiatku spracovania je potrebné definovanie problému a dôkladná analýza. Nasledujúce časti podrobne opisujú postup od úvodnej analýzy až po testovanie implementovaných riešení. Teoretické základy a poznatky pre vytvorenie tejto práce sú obsiahnuté v kapitole 2 Súčasný stav.

%K výslednému produktu tejto práce viedlo viacero krokov. V nasledujúcom zozname sú v skratke zadefinované:
%\startitemize[n]
%\item{preskúmanie súčasného stavu eGovernmentu a nutných znalostí k vývoju aplikácie určenej pre komunikáciu s verejnou správou}
%\item{analýza súčasného stavu projektu EMMA implementovaného prostredníctvom SAMO aplikácie}
%\item{pochopenie fungovania platformy SAMO a jeho technologického zázemia}
%\item{na základe dohody s firmou rozšírenie existujúcej aplikácie o modul zamestnancov a zapojenie vybraných služieb VS}
%\startitemize[a]
%\item{potrebná analýza a modely (use case, model požiadaviek a pod.)}
%\item{KDM a LDM, ktoré sú potrebné pre generovanie metadát aplikácie}
%\item{návrh agendovej aplikácie a jeho zápis do štrukturovanej podoby kvôli AMK}
%\item{použitie AMK generátoru, ktorý vykoná extrakciu dát z AMK DB a vygeneruje SAMO konfiguračné soubory (metadáta)}
%\item{úprava vygenerovaných metadát do žiadanej podoby, zapojenie aplikačnej logiky}
%\stopitemize
%\item{vytvorenie testovacích scenárov}
%\stopitemize

%\pkap{Popis súčasného stavu EMMA}
%Platforma EMMA bude sprostredkovávať služby verejnej správy podnikateľským subjektom. V aktuálnom stave nemá zákazníka a je pokladaná za "výskumný projekt".
%
%Na diagrame nižšie je zobrazená enterprise architektúra platformy. V ľavej časti je zobrazená verejná správa (úrady, obce a pod.), ktoré vystavujú služby slúžiace na poskytovanie údajov a informácií a zároveň na splnenie poivnností klientov VS. Aj keď existuje Register práv a povinností, ktorý obsahuje katalóg služieb verejnej správy, s cieľom zmapovať všetky služby verejnej správy a verejne poskytovať informácie o týchto službách. Každopádne samotné služby jednotlivých úradov nie sú konsolidované (nemajú jednotné API) a nie sú poskytované prostredníctvom jednej platformy.
%
%Pre realizáciu služieb musí platforma EMMA obsahovať podporné komponenty ako je Katalóg služieb EMMA, Portál pre prístup klientom, SW komponenty na sprostredkovanie služieb (zbernice, API management, správa oprávnění a iné). 
%
%Klienti EMMA (pravá časť diagramu), podnikateľské subjekty, môžu využiť služby EMMA pre jednoduché zapojenie do svojich ERP systémov a tým znížt ručne vykonávané činnosti pomocou automatizácie
%
%\obrazekB{obr:koncept}
%{Enterprise architektúra platformy EMMA  (Asseco, 2023)}{images/koncept.png}{width=32cc}
%
%Aktuálne je EMMA nasadená na cloudovom prostredí Azure. Konkrétne sú využívané služby ako Application Insights (pre sledovanie behu aplikácie), Key vault (ukladanie hesiel), Smart detector alert rule (detekovanie anomálií) a iné. 
%
%Pre jednoduchšie ukázanie fungovania platformy pre potenciálneho zákazníka sa využíva aplikácia postavená na platforme SAMO. Aj EMMA aj SAMO sú platformy spoločnosti Asseco. SAMO je základom pre evidenciu služieb EMMA. Pôvodným zámerom bola len správa katalógu služieb verejnej správy, ohlasovateľov, poskytovateľov a podobne. Časom sa to ale začalo rozširovať a vznikol modul "Žádosti" a konkrétne "Ohlášení nástupu zaměstnance". V súčasnom stave to spočíva vo vyplnení jednoduchého formulára s textovými, prípadne číselnými údajmi, ktorý sa vyplní potrebnými údajmi pre ČSSZ, MPSV a ZP. Údaje sa týkajú zamestnanca, zamestnania aj zamestnávateľa. Následne sa pomocou API služieb odošle na príslušné úrady a overí sa. Bližšie je tento proces popísaný na \in{obrázku}[obr:sekv1].


%\obrazek{obr:sekv1}
%{Sekvenčný model funkčnosti "Oznámení nástupu zaměstnace" (Asseco, 2023)}{images/SekvEMMA01.png}{width=32cc}
%
%Tento proces je ale pomerne zdĺhavý, keďže pri každom nástupe zamestnanca sa musí všetko vypĺňať. Ak by šlo o inú službu, niečo, čo sa týka už existujúcich zamestnancov, proces by sa ešte viac predĺžil a duplikoval.
%
%Preto je cieľom tejto práce vyvinúť modul zamestnancov a zamestnávateľov, ktorý tieto procesy urýchli. Komunikácia s úradmi tak bude takzvane "na jeden klik".
%
%Čo sa týka výzoru aplikácie, pôvodné obrázky a ikony boli prebraté z iného projektu, a tak neboli úplne adekvátne a bolo nutné ich upraviť/vytvoriť nové.

\pkap{Analýza}
Proces analýzy je neoddeliteľnou súčasťou každého projektu a zohráva kľúčovú rolu v jeho úspešnom realizovaní. Analýza pomáha identifikovať požiadavky, zvyšuje efektivitu vývoja a umožňuje hľadať najvhodnejšie riešenia v~súlade s potrebami a cieľmi organizácie a zákazníka. (Tradinos, 2023)

V nasledujúcich podkapitolách sa nachádzajú rôzne modely v jazyku UML, ktoré približujú požiadavky, prípady použitia a dátovú štruktúru aplikácie.

\ppkap{Model požiadaviek}
Na základe analýzy vznikol model požiadaviek (\in{viď obrázok}[obr:f-req] v časti Prílohy). Nachádza sa tam 15 požiadaviek, pričom táto práca je zameraná najmä na rozšírenie aplikácie o požiadavku modulu zamestnancov. To ale obnáša aj ostatné požiadavky ako napríklad správa oprávnení, prihlasovanie, registrácia, notifikácie a podobne.


\ppkap{Use case model}
Use case diagram alebo diagram prípadov použitia sa používa na zobrazenie funkčnosti systému alebo jeho časti. Zároveň znázorňuje funkčné požiadavky a ich interakcie s externými agentmi, tzv.\,aktérmi \scr(GeeksForGeeks, 2024).

\blank
V modeloch sú používaní štyria aktéri:

\startitemize
\item{systém}
\item{správca systému -- administrátor na strane sprostredkovateľa}
\item{zamestnávateľ -- administrátor na strane klienta}
\item{poverený zamestnanec -- zamestnanec s prístupom do systému (účtovník, personalista\dots)}
\stopitemize 

Prvý model (\in{obrázok}[obr:uc1]) zobrazuje prípady použitia zo strany administrátora systému.  Správca systému je schopný zadávať nových zamestnávateľov (klientov) do systému a zakladať im užívateľské kontá. Ďalej má umožnenú správu služieb a oprávnení, reporting a personalizáciu.

\obrazekH{obr:uc1}
{Use case model zo strany správcu (vlastné spracovanie)}{images/UC Správca.png}{width=32cc}

%\obrazekH{obr:uc3}
%{Use case model registrácie povereného zamestnanca (vlastné spracovanie)}{images/UC registrácia.png}{width=32cc}

\obrazekH{obr:uc2}
{Use case model zo strany povereného zamestnanca (vlastné spracovanie)}{images/UC-Zamestnanec.png}{width=32cc}

Na druhom use case (\in{obrázku}[obr:uc2]) modeli sú zobrazené use casy z pohľadu povereného zamestnanca a zamestnávateľa. Obaja aktéri majú prístup do systému pod prihlasovacími údajmi. Ďalej majú umožnenú správu zamestnancov a vykonávanie procesných úkonov spojených so službami verejnej správy. Zároveň je im umožnené aj spravovať číselníky špecifické pre vybraného klienta.

Use case týkajúci sa len zamestnávateľa je proces založenia zamestnávateľa do SAMO EMMA. Akonáhle je vytvorený administrátorský účet klienta (správcom systému), môže do systému registrovať ďalších poverených zamestnancov.

Vybrané use casy a ich scenáre sú podrobnejšie popísané pomocou diagramov aktivít v nasledujúcej kapitole.


\ppkap{Diagram aktivít}
Diagram aktivít slúži na zobrazenie toku činností v systéme a popisuje use case a jednotlivé use case scenáre.

Diagramy aktivít hlavných činností, ktorými sa zaoberá tento projekt sú vyobrazené v časti Prílohy na \in{obrázkoch}[obr:acti3] až \in[obr:acti-overeni]. Slovný popis jednotlivých aktivít sa nachádza v nasledujúcich odstavcoch.

Na formuláciu slovného popisu bola použitá umelá inteligencia, konkrétne ChatGPT 4. Vygenerovaný text bol následne mierne upravený.

\zlom
\start
\setupindenting[no]
\start \bf Prihlásiť sa do systému\stop
\stop

V prvej popisovanej aktivite (\in{obrázok}[obr:acti3]) ide o proces prihlásenia sa do systému z pohľadu užívateľa (zamestnanca, zamestnávateľa, správcu). Proces sa začína tým, že zamestnanec otvára prihlasovaciu stránku, na čo systém reaguje jej zobrazením. Nasleduje zadanie mena a hesla zamestnancom, ktorých správnosť systém overí. Ak sú prihlasovacie údaje správne, systém prihlási používateľa a presmeruje ho na domovskú stránku. V prípade, že údaje správne nie sú, systém informuje o chybe prostredníctvom chybovej hlášky o nesprávnych prihlasovacích údajoch. Aktivita končí úspešným/neúspešným prihlásením.


\blank
\start
\setupindenting[no]
\start\bf Vytvoriť žiadosť o registráciu povereného zamestnanca\stop
\stop

Ďalší diagram (\in{obrázok}[obr:acti1]) zobrazuje proces vytvorenia žiadosti o registráciu povereného zamestnanca z pohľadu zamestnávateľa. Proces začína tým, že zamestnávateľ vytvorí fyzickú osobu pre zodpovedného zamestnanca. Systém potom zobrazí detail tejto fyzickej osoby. 

Následne zamestnávateľ vytvorí nový záznam spojenia (kontaktu), kde zadá email a označí spojenie ako primárne. Systém uloží primárny email osoby (derivovaný atribút). Ak je spojenie vytvorené správne, zamestnávateľ spustí akciu \uv{Žiadosť o vytvorenie registrácie} na detaile osoby. Systém skontroluje, či má osoba vyplnený primárny email. Ak áno, proces pokračuje a systém odošle osobe email s aktivačným kódom a odkazom na registráciu. Ak osoba nemá vyplnený primárny email, systém vypíše chybovú hlášku: \uv{U vybranej osoby nie je uvedený email ako primárne spojenie}.

Aktivita sa končí buď chybovou hláškou alebo úspešným odoslaním emailu registrovanému zamestnancovi.

\blank
\start
\setupindenting[no]
\start\bf Registrovať sa do systému\stop
\stop


Tento diagram (\in{obrázok}[obr:acti2]) popisuje proces samotnej registrácie z pohľadu zamestnanca. Proces začína vyhotovením žiadosti o registráciu povereného zamestnanca popísaným v predchádzajúcej sekcii. Po úspešnom zaslaní registračného emailu sa začína samotná aktivita registrácie.

Zamestnanec klikne na odkaz v emaili, čím otvorí registračnú stránku. Na tejto stránke zamestnanec zadá požadované údaje: meno, priezvisko, a aktivač\-ný kód (ktorý mu bol vygenerovaný v predchádzajúcej aktivite). Systém overí\zlom správnosť všetkých údajov, a ak je kombinácia meno-priezvisko-aktivačný kód správ-na, zamestnanec pokračuje na ďalší krok registrácie. Ak systém zistí, že kombinácia nie je správna, vypíše chybovú hlášku.

Po správnom zadání všetkých údajov druhého kroku, a teda login a dva krát zopakované dostatočne silné heslo, systém overí jedinečnosť loginu, dostatočnosť a správnosť hesla. Ak údaje spĺňajú kritériá, poverený zamestnanec prechádza na tretí krok registrácie -- potvrdenie súhlasu s GDPR. Systém nakoniec overí zaškrtnutie tohto súhlasu.

Po kliknutí na tlačidlo \uv{dokončiť} systém uloží údaje o úspešnej registrácii a odošle užívateľovi potvrdzovací email o úspešnej registrácii spolu s odkazom na prihlásenie. Týmto sa aktivita končí.

\blank
\start
\setupindenting[no]
\start\bf Zadať nového zamestnanca do systému\stop
\stop

Aktivita zadania nového zamestnanca (\in{obrázok}[obr:acti-nastup]) začína, keď poverený zamestnanec prejde na sekciu nových zamestnancov a klikne na tlačidlo pridať, označené symbolom \uv{+}. Systém následne otvorí päťkrokový formulár na pridanie nového zamestnanca. Poverený zamestnanec vyplní osobné údaje, kontaktné informácie, adresu a údaje o zamestnaní. Po vyplnení formuláru, systém validuje správnosť a úplnosť zadaných údajov.

Ak údaje nie sú správne alebo úplné, systém vypíše upozornenia na chyby. Ak sú údaje správne a úplné, zamestnanec klikne na tlačidlo \uv{vytvoriť}. Systém potom vytvorí príslušného zamestnanca a jeho spojenie, adresu a žiadosť typu \uv{oznámenie o nástupe zamestnanca}. Nakoniec, systém spustí funkciu \uv{zaregistrujNastupZamestnance}, ktorá žiadosti pridelí unikátny identifikátor (UUID). Proces sa končí úspešným pridáním zamestnanca a jeho naviazaných entít.

\blank
\start
\setupindenting[no]
\start\bf  Registrovať nástup zamestnanca\stop
\stop

Proces, ktorý je na \in{obrázku}[obr:acti-overeni] nasleduje po úspešnej aktivite popísanej v predchádzajúcom odstavci, a teda tým, že autorizovaný zamestnanec založí nového zamestnanca a súčasne s tým aj novú žiadosť. Po založení nového zamestnanca a žiadosti, systém zobrazí detail žiadosti spolu s tlačítkom \uv{overiť žiadosť}. Autorizovaný zamestnanec následne klikne na toto tlačítko na overenie žiadosti. 

Systém potom spustí funkciu s názvom \uv{ctiVysledekZaregistrujNastupZamestnance} s parametrom UUID žiadosti. Táto funkcia vykoná procesy potrebné na overenie žiadosti na príslušních orgánoch a vráti výsledok, ktorý môže byť buď úspešné overenie, alebo chyba. To sa prejaví prepnutím stavových atribútov žiadosti. 

Fungovanie funkcií zaregistrujNastupZamestnance a CtiVysledekZaregistrujNastupZamestnance je detailnejšie zobrazené na sekvenčnom diagrame na \in{obrázku}[obr:sekv1]


\ppkap{Konceptuálny dátový model}
Konceptuálny dátový model predstavuje pri modelovaní SAMO aplikácie prvý krok k vybudovaniu finálneho dátového modelu. Ide o zjednodušenú ukáž\-ku existujúcich entít a väzieb medzi nimi. KDM projektu je na \in{obrázku}[obr:kdm]. Nachádza sa v ňom desať entít.

Prvou entitou je nadriadená entita žiadosť, ktorá môže mať niekoľko potomkov, na základe typu žiadosti. Aktuálne má len jedného potomka, a to žiadosť týkajúca sa nástupu zamestnanca do zamestnania. 

Každá žiadosť musí mať väzbu na zamestnanca a zamestnávateľa, pričom násobnosť je 1\,:\,N smerom k žiadosti.

Entita zamestnanec predstavuje fyzickú osobu zamestnanú u nejakého zamestnávateľa. To vysvetľujú väzby zamestnávateľ\ip zamestnanec (1\,:\,N) a fyzická osoba\ip zamestnávateľ (1\,:\,N). 

Zamestnávateľ môže mať N zamestnancov a je naviazaný na práve jednu právnickú osobu. 

Právnická i fyzická osoba majú nad sebou abstraktnú entitu osoba. Dalo by sa to do budúcna rozšíriť o ďalšie typy osôb, ako je podnikajúca fyzická osoba, OVM a podobne. 

Posledné entity sa týkajú kontaktných údajov a adries fyzických a právnických osôb. Medzi osobami a spojením/adresou je väzba 1\,:\,N, čo znamená, že každá osoba môže mať v systéme viacero spojení a adries, ale každé spojenie a~adresa patria práve jednej osobe.


\obrazekH{obr:kdm}
{Konceptuálny dátový model - temp (vlastné spracovanie)}{images/kdm.png}{width=32cc}

\zlom
\ppkap{Logický dátový model pre SAMO}
Logický dátový model je vytváraný pomocou extenzie LIDS7 (spomínaný v~kapitole 3.8.1). Zobrazuje feature types (SAMO entity), atribúty a vzťahy medzi jednotlivými feature typami a odkazy na číselníky.

Na prvom modeli (\in{obrázok}[obr:ldm-sluzby]) je namodelovaný súčasný stav aplikácie SAMO EMMA. Ide primárne o správu katalógu služieb. Tejto časti sa praktická časť projektu venuje len okrajovo.

\obrazekH{obr:ldm-sluzby}
{Logický dátový model -- služby a katalóg služieb (vlastné spracovanie podľa aktuálneho riešenia)}{images/ldm-sluzby.png}{width=32cc}

Časť aplikácie, ktorou sa zaoberá táto práca je rozdelená kvôli prehľadnosti do dvoch modelov. Prvý z nich (\in{obrázok}[obr:ldm-os]) sa zaoberá správou osôb, ich kontaktov a adries. Druhá časť (\in{obrázok}[obr:ldm-zam]) je venovaná oblasti zamestnania, to znamená evidencii zamestnancov, zamestnávateľov a žiadostí.

\obrazekH{obr:ldm-os}
{Logický dátový model - osoby, adresy a spojenia (vlastné spracovanie)}{images/LDMos.png}{width=32cc}

\obrazekH{obr:ldm-zam}
{Logický dátový model - zamestnávatelia a zamestnanci (vlastné spracovanie)}{images/LDMz.png}{width=32cc}


Entity môžu obsahovať tri typy atribútov:

\startitemize
\item {\start\bf fyzický (F) \stop -- klasický atribút}
\item {\start\bf väzobný (V) \stop -- atribút, ktorý je prepojením na inú entitu, s ktorou má vybraná entita väzbu}
\item {\start\bf derivovaný (D) \stop -- odvodený atribút na základe výpočtu alebo databázovej funkcie, ktorý sa aktualizuje pri zmene atribútu, ktorý ho definuje}
\stopitemize

Všetky entity obsahujú okrem popísaných atribútov aj niekoľko systémových atribútov, ktoré sú v \in{tabuľke}[tab:system-att].


V nasledujúcich podkapitolách sú popísané vybrané číselníky, entity a ich atribúty a väzby. Ide hlavne o entity, ktoré boli v rámci projektu vytvárané, resp. najviac rozvíjané, teda osoba, zamestnanec, zamestnávateľ, adresa a~spojenie.

Všetky tabuľkách v časti Prílohy (konkrétne Príloha D) sú popísané jednotlivé atribúty všetkých feature typov, ich dátové typy a väzby.

\blank
\start
\setupindenting[no]
\start\bf Číselníky\stop
\stop

Pre správne fungovanie a validnosť údajov je potrebné vytvoriť a naplniť číselníky. Vzhľadom na to, že v aplikácii sa bude komunikovať s verejnou správou, kde musia získavať presné údaje, textové polia pri vyplňovaní niektorých údajov nestačia. 

Číselníky, ktoré boli novo vytvorené a naplnené sú napríklad cl\_zdravotniPojistovna, cl\_druhCinnosti. Tieto číselníky sú naplnené podľa verejných číselníkov Českej správy sociálního zabezpečení \scr(ČSSZ -- Číselníky, 2024). Tým sa zaistí, že údaje v žiadostiach budú správne. 

Ďalšími číselníkmi sú číselníky týkajúce sa adries, nateraz len českých adries. Konkrétne štát, kraj, okres, obec, časť obce, ulice a psč. Týmto sa predíde zadávaniu neplatných adries. Tieto údaje boli získané z čiselníkov vyšších a nižších územných celkov z RÚIAN-u \scr(RÚIAN, 2024).

Čiastočne sú databázovo naplnené číselníky štátneho občianstva a štátu krajinami susediacimi s ČR a ČR. V aplikácii ale bude možnosť ich rozšíriť v prípade, že firma zamestnáva aj ľudí zo vzdialenejších štátov.

Bola nutná aj úprava číselníka cl\_pohlavi, kedže v ňom chýbal stĺpec "kod", ktorý je potrebný v žiadostiach VS. Bol naplnený jednoduchým SQL skriptom do podoby 1-Žena-Z, 2-Muž-M, 3-Neurčeno-N.

Keďže okrem komunikácie s VS systém slúži aj na samotnú evidenciu zamestnancov, boli doplnené aj rôzne číselníky spojené priamo so zamestnaním. Ide o~číselníky konfigurovateľné priamo v aplikácii, keďže každá firma to môže mať rozdielne. Medzi tieto číselníky patrí cl\_divize, cl\_typUvazku, cl\_pozice. Konfigurácia prebieha prostredníctvom modulu \uv{Správa číselníků}.

Okrem spomínaných číselníkov existuje aj špeciálny číselník cl_anoNe, ktorý sa používa na vyjadrenie boolean hodnoty v SAMO aplikáciách. 

\blank
\start
\setupindenting[no]
\start\bf Osoba \stop
\stop

Feature type osoba je abstraktná entita, ktorá je rodičom dvoch ďalších entít, fyzickej a právnickej osoby. 

Tieto podriadené entity využívajú viacero číselníkov ako sú číselník pohlavia, štátneho občianstva a číselník cl_anoNe.

\blank
Odvodenie derivovaných atribútov fyzickej osoby: % je v úsekoch kódu na \in{obrázku}[obr:derFO].


%\obrazekH{obr:derFO}
%{Odvodenie derivovaných atribútov FO (vlastné spracovanie)}{images/derFO.png}{width=32cc}

\startitemize
\item {\start\bf PRIMARNIMAIL \stop -- na základe väzby as_fyzickaOsoba_spojeni sa doplní atribút at_spojeni_email (ak také spojenie existuje a je označené ako primárne)}
\item {\start\bf UPLNEJMENOOSOBY \stop -- reťazec, ktorý vznikne spojením dvoch atribútov FO, "prijmeni + ' ' + jmeno"}
\item {\start\bf VEK \stop -- vypočíta sa pomocou preddefinovanej databázovej funkcie\zlom date_part('year', age(at_fyzickaOsoba_datumNarozeni)) }
\stopitemize

\blank
\start
\setupindenting[no]
\start\bf Spojenie \stop
\stop

Táto entita slúži na ukladanie kontaktov osoby. Je naviazaná na právnickú i~fyzickú osobu a umožňuje ukladať kontakty ako email, telefónne číslo, dátová schránka a podobne. Osoba môže mať práve jedno spojenie označené ako primárne.

\blank
\start
\setupindenting[no]
\start\bf Adresa \stop
\stop

Entita adresa slúži na ukladanie adries osôb. Je, podobne ako spojenie, naviazaná na právnickú i~fyzickú osobu. Využíva viacero číselníkov, ktoré boli plnené pomocou dát z RÚIAN-u. Adresa môže byť trvalá a korešpondenčná. Osoby môžu mať práve 2 platné adresy -- jednu korešpondenčnú a jednu trvalú.

Okrem osôb je adresa naviazaná aj na adresné miesto. Táto entita by mala zmysel, ak by bol systém napojený na databázu RÚIAN, kedy by bral realtime dáta o adresných miestach z tohto registru.

\blank
\start
\setupindenting[no]
\start\bf Zamestnanec \stop
\stop

Ďalšou entitou je entita zamestnanec. Táto entita je naviazaná na zamestnávateľa a na fyzickú osobu. Slúži na evidenciu údajov o~zamestnaní, ktoré sú relevantné pre zamestnávateľa alebo úrady.

\blank
\start
\setupindenting[no]
\start\bf Zamestnávateľ \stop
\stop

Poslednou vybranou entitou je entita zamestnávateľ. Táto entita je naviazaná na spomínaného zamestnanca a právnickú osobu. Slúži na evidenciu údajov o~zamestnávateľovi, ktoré sú potrebné pre komunikáciu s VS.

\ppkap{Stavové diagramy}
Stavový diagram popisuje životný cyklus entity. Ukazuje jednotlivé stavy a~možné medzistavové prechody. V rámci tejto práce sa môžu meniť stavy dvom entitám -- zamestnanec a žiadosť. Stavové diagramy sú na \in{obrázkoch}[obr:zamestnanec-stav] a \in[obr:zadost-stav].

\blank
\start
\setupindenting[no]
\start\bf Zamestnanec \stop
\stop

Zamestnanec môže nadobudnúť štyri rôzne stavy. Sú to stavy "nový", "řádný", "výpověď" a "ukončený".

Počiatočný stav je "nový". Po úspešnom oznámení o nástupe zamestnanca sa tento stav aplikačne prepne do "řádný". V tomto stave je entita väčšinu existencie. 

V prípade, že zamestnanec dá výpoveď, zodpovedný zamestnanec pomocou tlačidla na detaile zamestnanca prepne jeho stav do stavu "výpověď". 

Posledným stavom je stav "ukončený", do ktorého sa entita dostáva po ubehnutí výpovednej lehoty. V prípade, že sa zamestnanec rozhodne výpoveď stiahnuť, môže sa prepnúť späť do stavu "řádný" pomocou tlačidla. 

O stave zamestnancov hovorí číselníkový atribút cl_stavZamestnance. Číselník bol naplnený ručne pomocou jednoduchého SQL príkazu.

\blank
\start
\setupindenting[no]
\start\bf Žiadosť \stop
\stop

Druhou stavovou entitou je entita žiadosť. Táto entita môže nadobudnúť tri rôzne stavy, a to "neoveřeno", "chyba" a "ověřeno". 

Počiatočný stav je "neověřeno". Po kliknutí na akciu overenia žiadosti na detaile vybranej žiadosti sa overí žiadosť na rôznych zaujatých orgánoch VS. 

Ak overenie prebehne bez chyby, prepína sa do stavu "ověřeno". V prípade akejkoľvek chyby sa prepne do stavu "chyba". 

Žiadosť môže nadobúdnúť viac stavov súčasne, keďže sa to vyhodnocuje vzhľadom k rôznym orgánom. Môže byť napríklad v stave "oveřeno" vo vzťahu k ČSSZ a v stave "chyba" vo vzťahu k ZP. 

Je to riadené pomocou číselníkových atribútov rôznych orgánov, napr. cl_stavZadostiCSSZ, cl_stavZadostiZP a podobne. Celkový stav žiadosti sa dá vyhodnocovať na základe jednotlivých stavov. 


\obrazekH{obr:zamestnanec-stav}
{Stavový diagram entity zamestnanec (vlastné spracovanie)}{images/stav1.png}{width=32cc} 


\obrazekH{obr:zadost-stav}
{Stavový diagram entity žiadosť (vlastné spracovanie)}{images/stav2.png}{width=28cc} 


\zlom
\pkap{Návrh a dizajn}
Farebná paleta a obrázok na úvodnej obrazovke boli predom určené projektovým tímom firmy Asseco.

Čo sa farebnej palety týka, primárnou farbou je modrá a sekundárne farby sú odtiene šedej (viď \in{obrázok}[obr:color]). Jednotlivé dlaždice agend môžu mať na sebe ešte farebný prúžok, ktorý je možno farebne prispôsobovať podľa požiadaviek zákazníka a nemusí rešpektovať farebnú paletu (viď dlaždice v spodnej časti \in{obrázku}[obr:dlazdice]).

\obrazekH{obr:color}
{Farebná paleta (vlastné spracovanie pomocou coolors.co)}{images/color.png}{width=30cc} 

Pri vytváraní ikoniek a obrázkov bolo dbané na štandardy a best practices firmy Asseco. Ikony sú navrhnuté v súlade s pravidlami Google Material Design. 

Ako možno vidieť na \in{obrázku}[obr:dlazdice], existujú dva rôzne typy dlaždíc. Jednou z typov dlaždíc je úvodná, pod ktorou sa skrývajú ďalšie dlaždice. Na týchto dlaždiciach sa nachádzajú obrázky reprezentujúce danú agendu, oblasť. 

Druhá úroveň dlaždíc už odkazuje priamo na zoznam, tzv.\,browse. Tieto dlaždice obsahujú názov, ikonku danej entity a počet záznamov skrývajúcich sa pod danou dlaždicou. 

Dlaždicové obrázky mali svoje pravidlá použitých farieb a hrúbky čiary 3~px. Ich úlohou je čo najviac vystihovať danú agendu. Ukladané sú vo formáte png. 

Ikony sú taktiež prispôsobované tomu, čo reprezentujú. Sú ukladané vo formáte svg bez použitia výplne alebo farby čiary, aby bola ich farba plne prispôsobiteľná v aplikácii.

Všetky dizajnové prvky sú kreslené pomocou nástroja Figma. Boli vytvorené/upravované tak, aby mala celá aplikácia jednotný vizuálny štýl. 

\obrazek{obr:dlazdice}
{Príklad SAMO dlaždíce (vlastné spracovanie)}{images/exist-obrazovky/dlazdice.png}{width=28cc} 

Čo sa týka štruktúry aplikácie, SAMO aplikácie majú danú nasledujúcu štruktúru.  Úvodná obrazovka je vždy jednoduchá login-page. V prípade SAMO EMMA to vyzerá tak, ako je na \in{obrázku}[obr:login]. Obrázok v pozadí je vybraný firmou.

Nasledujúce prvky aplikácie sú zobrazené pomocou drátených modelov vytvorených v nástroji Figma. Drátené modely sa nachádzajú v časi Prílohy. Tieto modely slúžia ako vzor pre finálny vzhľad aplikácie.

Po prihlásení sa užívateľ dostane na úvodný rozcestník, tzv.\,dashboard (\in{obrázok}[obr:drat-dashboard]). Dashboard môže byť rôzne logicky delený do dlaždíc. Pod každou dlaždicou sa nachádza ďalšia úroveň a to dlaždice s ikonou a počtom entít v danej agende (\in{obrázok}[obr:drat-dashboard2]). Tieto dlaždice už vedú na samotný zoznam entít, tzv.\,browse (\in{obrázok}[obr:drat-browse-detail]).


%\obrazekH{obr:drat-edit}
%{Drátený model editačného formulára typu samo-entity-properties-form (vlastné spracovanie)}{images/wire/edit-detail.png}{width=17cc}
%
%\obrazekH{obr:drat-stepper}
%{Drátený model editačného formulára typu samo-stepper-module (vlastné spracovanie)}{images/wire/stepper.png}{width=17cc}

%\obrazekH{obr:dashboard1}
%{Obrazovka -- prvá úroveň dlaždíc (vlastné spracovanie)}{images/exist-obrazovky/dlazdice1.png}{width=32cc} 
%
%\obrazekH{obr:dashboard2}
%{Obrazovka -- druhá úroveň dlaždíc (vlastné spracovanie)}{images/exist-obrazovky/dlazdice2.png}{width=32cc} 
%
%\obrazekH{obr:browse}
%{Obrazovka -- browse a detail (vlastné spracovanie)}{images/exist-obrazovky/browse.png}{width=32cc} 

V browse je užívateľovi umožnené vytvárať nové záznamy pomocou formulára ako je na \in{obrázku}[obr:drat-edit], poprípade [obr:drat-stepper]. Zároveň je možné prezerať si detaily záznamov (viď pravá časť \in{obrázku}[obr:drat-browse-detail], poprípade ich editovať (podobnými formulármi ako pri vytváraní). V aplikácii je na každej úrovni v ľavej časti sekundárna navigácia.

Okrem týchto drátených modelov je tu aj drátený model modulu na správu číselníkov. Na \in{obrázku}[obr:drat-spravacl] je vidieť rôzne číselníky, ktoré možno upravovať. Na obrazovku, ktorá je na \in{obrázku}[obr:drat-konkretnycl] sa dostane užívateľ výberom jedného z~číselníkov. Sú tam zobrazené konkrétne záznamy v danom číselníku s možnosťou editácie a pridávania novej hodnoty.

\blank
Pre lepšie priblíženie štruktúry aplikácie je na \in{obrázku}[obr:site-map] zobrazený hierarchicky usporiadaný zoznam všetkých stránok, takzvaná mapa webu alebo sitemap.

\obrazekH{obr:site-map}
{Mapa webu (vlastné spracovanie)}{images/sitemap.png}{width=32cc} 


%\obrazekH{obr:edit}
%{Obrazovka -- editačný formulár (vlastné spracovanie)}{images/exist-obrazovky/edit.png}{width=32cc} 

%\pkap{Výber služby, ktorá bude implementovaná}
%\pkap{Charakteristika vybraných služieb verejnej správy}
%Na to, aby bolo možné vybrať vhodné služby VS k analýze a ďalším úkonom je potreba analyzovať rôzne životné situácie spojené s verejnou správou. V nasledujúcej tabuľke (viď \in{tabuľka}[situacie]) je zobrazený zoznam takýchto situácií spolu so službou a úradom, ktoré do riešenia danej situácie vstupujú.
%
%
%  \setupTABLE[column][1][width=12cc]
%  \setupTABLE[column][2][width=8cc]
%  \setupTABLE[column][3][width=16cc]
%  %\setupTABLE[column][4][width=0.25\textwidth]
%  \setupTABLE[r][each][align={middle,lohi}]
%
%  \Tabulka{situacie}{Životné situácie spojené so zamestnávateľmi a verejnou správou}{
%    \bTABLE
%      % Table header
%      \bTR
%        \bTH Životná situácia \eTH
%        \bTH Úrad \eTH
%        \bTH Služba VS \eTH
%        %\bTH Služby v RPP \eTH
%      \eTR
%
%      % Table rows
%	\bTR 
%	    \bTD [nr=3] Nástup zamestnanca \eTD 
%	    \bTD ČSSZ \eTD 
%	    \bTD Oznámenie o nástupe do zamestnania (Přihlášky, odhlášky zamestnancov k nemocenskému poisteniu) \eTD  
%	\eTR
%	\bTR 
%	    \bTD ZP \eTD 
%	    \bTD Hromadné oznámenie zamestnávateľa (HOZ)  \eTD  
%	\eTR
%	\bTR 
%	    \bTD MPSV/ÚP \eTD 
%	    \bTD Informace o nástupe občana cudzinca, ktorý nepotrebuje/potrebuje pracovné oprávnenie do zamestnania  \eTD  
%	\eTR
%	\bTR 
%	    \bTD [nr=4] Zamestnávanie osôb so zdravotným postihnutím\eTD 
%	    \bTD MPSV  \eTD  
%	    \bTD Evidencia náhradného plnenia \eTD 
%	\eTR
%	\bTR 
%	    \bTD MPSV  \eTD  
%	    \bTD Ohlásenie plnenia povinného podielu osôb so zdravotným postihnutím (OZP) \eTD 
%	\eTR
%	\bTR 
%	    \bTD MPSV  \eTD  
%	    \bTD Žiadosť o príspevok na zdriadenie pracovného miesta pre OZP \eTD 
%	\eTR
%	\bTR 
%	    \bTD MPSV  \eTD  
%	    \bTD Žiadosť o príspevok na Nový podnikateľský program\eTD 
%	\eTR
%	\bTR 
%	    \bTD [nr=2] Voľné miesta\eTD 
%	    \bTD MPSV \eTD 
%	    \bTD Oznámenie voľných pracovných miest ÚP ČR \eTD  
%	\eTR
%	\bTR 
%	    \bTD MPSV \eTD 
%	    \bTD Oznámenie popisu pracovnej pozície pre Jobmatch do evidencie ÚP ČR neregistrovaným uživateľom  \eTD  
%	\eTR
%	\bTR 
%	    \bTD Row 1, Col 1 \eTD 
%	    \bTD Row 1, Col 2 \eTD 
%	    \bTD Row 1, Col 3 \eTD  
%	\eTR
%    \eTABLE
%  }
%
%
%
%\TODO
%TODO
%
%\pkap{Implementácia vybranej služby}
%
%\TODO
%TODO

\pkap{Implementácia}
Po dôkladnej analýze a návrhu prichádza fáza implementácie. Prvým krokom je inštalácia lokálneho prostredia. To zahŕňa naklonovanie potrebných repozitárov -- configuration a project. Následne je nutná registrácia balíčkov a samotná inštalácia projektu z env zložky projektu. Po úspešnej inštalácii je možné aplikáciu lokálne spustiť pomocou nástroja Localtron.

Po inštalácii lokálneho prostredia a kontrole správnosti logického dátového modelu v EA, sa môže pomocou nástroja EA2LIDS vygenerovať model.xml, ktorý je základom pre fungovanie SAMO aplikácie. 

Akonáhle model.xml prejde úpravami a je správny a kompletný, spustí sa skript AMK a vytvoria sa potrebné metadátové súbory ako napr. dashboard, základné pages, detaily, editačné formuláre atď. Vzhľadom na to, že je používaná agilná metodika, navrhnuté modely sa môžu meniť.

Vygenerované súbory sa nachádzajú v adresári emma-gen a pri prípadnom ďalšom generovaní sa prepíšu opäť podľa xml modelu. Preto je nutné akékoľvek zmeny v súboroch zaznamenávať v súboroch adresára emma-int. Týmito zmenami sú myslené rôzne validačné kontroly, zmeny defaultných formulárov a detailov a podobne. V nasledujúcich kapitolách sú rozobrané jednotlivé komponenty aplikácie.

\ppkap{Vzhľad a štruktúra}
Prvou obrazovkou, ktorú užívateľ uvidí je login page (viď \in{obrázok}[obr:login]). Po prihlásení sa dostane na úvodnú obrazovku, tzv.\,cockpit. Finálna podoba cockpitu je na \in{obrázku}[obr:fin-dashboard]. Je to rozdelené do sekcií "Zaměstnání", "Hlavní činnosti" a "Ostatní". Rozdelenie do sekcií a vzhľad obrázkov dlaždíc je odsúhlasený zákazníkom (firmou Asseco).

\obrazekH{obr:login}
{Obrazovka - login (vlastné spracovanie)}{images/exist-obrazovky/0-login.png}{width=32cc} 

\obrazekH{obr:fin-dashboard}
{Úvodná obrazovka -- dashboard (vlastné spracovanie)}{images/fin/dashboard.png}{width=32cc}

Sekcia zamestnania obsahuje všetky funkcie spojené so zamestnaním -- evidenciu zamestnancov, žiadostí a informácie o zamestnávateľoch (resp. konkrétnemu zamestnávateľovi, ktorý je zákazníkom projektu). 

V sekcii hlavných činností sú funkcie spojené s katalógom služieb a celkovo s evidenciou služieb verejnej správy v systéme. 

Posledná sekcia je určená na ďalšie agendy systému, ako je napríklad správa číselníkov a správa osôb.

\ppkap{Žiadosti}

Pod dlaždicou Žádosti sa aktuálne nachádza len jedna dlaždica, a to Ohlášení nástupu zaměstnance. Pri ďalšom rozvoji aplikácie tu pribudnú ďalšie žiadosti spojené s verejnou správou, napríklad hromadné oznámenie zamestnancov, odchod zamestnanca, služby spojené so zamestnávaním osôb so zdravotným postihnutím a podobne.

Po kliku na dlaždicu Ohlášení nástupu zaměstnance sa zobrazí zoznam všetkých žiadostí prihláseného zamestnávateľa. so základnými informáciami, konkrétne Jméno, Příjmení, Datum vytvoření, Stav žádosti ČSSZ a Stav žádosti Zdravotní pojišťovna. Pri rozkliknutí detailu sa zobrazia ďalšie informácie o žiadosti. Ukážka obrazovky je na \in{obrázku}[obr:zadosti-screen]. Je možné vytvoriť novú žiadosť pomocou plus v hornej časti. Formulár obsahuje všetky údaje k úspešnému oznámeniu a neobsahuje žiadne číselníky ani validácie. Tento formulár ale vznikol ešte pred začatím tejto práce a do budúcna sa bude žiadosť vytvárať cez dlaždicu \uv{Noví zaměstnanci}. Tento proces je popísaný nižšie, v časti Zamestnanci.

\obrazekH{obr:zadosti-screen}
{Obrazovka -- žiadosti (vlastné spracovanie)}{images/screen-neovereno.png}{width=32cc}

\ppkap{Zamestnanci}

Dôležitou súčasťou systému a hlavným cieľom tejto práce je modul pre evidenciu zamestnancov. Pod dlaždicou ZAMĚSTNANCI je to ešte ďalej rozdelené na zamestnancov a nových zamestnancov. Je to z dôvodu nutnosti splnenia niekoľkých úkonov na nových zamestnancoch, napr. už spomínaná služba \uv{Oznámení o nástupu zaměstnance}. Filtrovanie sa deje na základe atribútu cl_stavZamestnance.

Pri naberaní nových zamestnancov budú v stave "nový" a budú práve pod touto dlaždicou. Prehľadnejšie tak bude vidieť, ktorí zamestnanci už majú toto oznámenie hotové alebo nie. V momente, keď sa podarí úspešne nahlásiť nástup zamestnanca, prepne sa do stavu "řádný" a presunie sa pod dlaždicu zamestnanci.

V prípade výpovede zamestnanca prepne zodpovedný pracovník tohto zamestnanca do stavu "výpověď" a bude možné na ňom previesť akciu o ukončení pracovného pomeru vzhľadom k úradom. Stavový diagram je na \in{obrázku}[obr:zamestnanec-stav].

Úkony, ktoré sú možné pod dlažicou zamestnancov sú zakladanie nového zamestnanca, prezeranie zoznamu zamestnancov a ich detailov vrátanie naviazaných entít ako je adresa, kontakt a žiadosť. Zároveň je možnosť editácie existujúcich záznamov a prikladanie príloh k jednotlivým záznamom, ako napríklad scan dokladov, pracovná zmluva, potvrdenie o štúdiu študenta a podobne.

Čo sa týka evidencie, v zozname sú zobrazené základné informácie o zamestnancoch a to meno, dátum narodenia, divízia, pozícia a typ úväzku. Ostatné, podrobnejšie informácie sú v detailoch jednotlivých záznamov (viď \in{obr.}[obr:zam-b-d]).

\obrazekH{obr:zam-b-d}
{Zamestnanci -- browse a detail (vlastné spracovanie)}{images/zamestnanci-browse-detail.png}{width=32cc} 

Dlaždica nových zamestnancov taktiež umožňuje zakladať nových zamestnancov. Celý proces je zobrazený aj v diagrame aktivít na \in{obrázku}[obr:acti-nastup]. 

Ako bolo spomenuté aj v kapitole 4.2, v  SAMO existujú dva typy editačných formulárov. Prvým z nich je klasický editačný formulár, ktorý založí práve jednu entitu. Tento typ sa používa napríklad na žiadostiach, kedy sa po vyplnení formulára založí entita ft\_zadost, ktorá obsahuje vyplnené atribúty. Pri tomto type formuláru nie je potreba žiadna business logika (žiaden javascript) a všetko sa deje na úrovni metadát. 

Druhý typ formulára je akčný editačný formulár, ktorý dokáže založiť viacero entít naraz, poprípade vykonať ďalšie akcie. Pri tomto type formuláru je nutné použiť aj aplikačnú logiku, ktorá sa píše pomocou Javascriptu. 

V prípade zamestnancov bola použitá komponenta "samo-stepper-module". Stepper organizuje vstupy do viacerých postupných krokov. Každý krok môže pozostávať z viacerých sekcií, ktoré sú reprezentované vybranými webovými modulmi a entitami. V nasledujúcich odstavcoch si rozoberieme jednotlivé kro\-ky steppera zapojeného na zamestnancoch.

\blank
V prvom kroku má zodpovedný zamestnanec, ktorý zakladá nového zamestnanca 2 možnosti -- zakladá úplne novú osobu, ktorá v systéme ešte nijak nefiguruje (\in{obrázok}[obr:step1]) alebo zakladá zamestnanca z fyzickej osoby, ktorá už v~systéme bola založená (\in{obrázok}[obr:step1b]).

\obrazekH{obr:step1}
{Založenie zamestnanca krok~1 -- zakladanie novej osoby (vlastné spracovanie)}
{images/stepper/step1.png}
{width=25cc} 

\obrazekH{obr:step1b}
{Založenie zamestnanca krok~1 -- výber existujúcej osoby (vlastné spracovanie)}
{images/stepper/step1b.png}
{width=24cc} 

Krok číslo 2 slúži na vyplnenie kontaktných údajov zamestnanca. Ako email sa predvyplní štandardný email, ktorý je zvyklosťou firmy, napr. meno.priezvisko@firma.cz, ak nejaká zvyklosť u firmy existuje. Vo formulári sa zobrazia aj veľké písmená a diakritika, no ukladací skript to uloží ako validný email (malé písmená a bez diakritiky). Táto informácia je aj v zátvorke pri labeli emailu (viď \in{obrázok}[obr:step2]).

\obrazekH{obr:step2}
{Založenie zamestnanca krok~2 -- zadanie kontaktných údajov (vlastné spracovanie)}
{images/stepper/step2.png}
{width=24cc} 

V treťom kroku sa zadáva adresa zamestnanca, ktorá sa uloží zamestnancovi ako trvalá. Tretí krok je na \in{obrázku}[obr:step3].

\obrazekH{obr:step3}
{Založenie zamestnanca krok~3 -- zadanie adresy (vlastné spracovanie)}
{images/stepper/step3.png}
{width=24cc} 

Štvrtý krok obsahuje údaje týkajúce sa zamestnania, ako sú napr. zamestnávateľ, druh pracovného pomeru, pozícia a podobne. Všetky údaje možno vidieť na \in{obrázku}[obr:step4].

\obrazekH{obr:step4}
{Založenie zamestnanca krok~4 -- zadanie údajov o zamestnaní (vlastné spracovanie)}
{images/stepper/step4.png}
{width=24cc} 

Posledný, piaty krok slúži ako zhrnutie všetkých údajov žiadosti typu \uv{oznámenie o nástupe zamestnanca}. Všetky údaje sa predvypĺňajú na základe predchádzajúcich krokov. Všetky políčka sú neaktívne (disabled) a slúžia len pre kontrolu. V prípade, že niečo nesedí, je nutné upraviť dáta v niektorom z predchádzajúcich krokov steppera. Piaty krok je na \in{obrázku}[obr:step5].


\obrazekH{obr:step5}
{Založenie zamestnanca krok~5 -- kontrola údajov žiadosti (vlastné spracovanie)}
{images/stepper/step5.png}
{width=24cc} 

Po kliknutí na \uv{vytvoriť} sa vytvoria 4 entity -- zamestnanec, spojenie, adresa a žiadosť. Zároveň sa spustí proces, ktorý odošle hlásenie o nástupe zamestnanca na ČSSZ a ZP. Tento proces je event trigger, ktorý sa spustí pri vytvorení entity žiadosť. Tento trigger upraví dáta žiadosti do požadovaného formátu, ktorý je vstupom POST requestu zaregistrujNastupZamestnance.

Testovanie requestov je možné aj pomocou existujúceho Soap UI. Aplikačne sa všetky vyplnené údaje formulára musia zvalidovať a upraviť, aby nenastávali chyby. Zoznam vybraných nutných validácií a úprav:

\startitemize
\item{\start\bf datumNarozeni  \stop musí byť upravený do formátu RRRR-MM-DD} 
\item{\start\bf rodneCislo  \stop musí spĺňať kritériá pre správne rodné číslo, a to validný dátum, deliteľnosť 11 a dĺžku 9 alebo 10 číslic} 
\item{\start\bf pohlavi \stop musí byť odoslané ako jeden znak podľa číselníka pohlaví, konkrétne M alebo Ž} 
\item{\start\bf email \stop musí mať správny emailový formát}  
\item{\start\bf telefon \stop musí mať správny formát telefónneho čísla} 
\item{\start\bf datovaSchranka \stop musí byť validné ID dátovej schránky}  
\item{\start\bf druhCinnosti \stop musí byť 1 znak -- číslica alebo písmeno podľa číselníka druhu činností}  
\item{\start\bf datumNastupu \stop musí byť dátum vo formáte RRRR-MM-DD a v minulosti, respektíve max. dnes}  
\item{\start\bf zdravotniPojistovna \stop musí byť vybraná z číselníku zdravotných poisťovní a odosiela sa jej kód}  
\item{\start\bf statniObcanstvi a mistoVykonuCinnosti \stop -- nateraz je k dispozícii len Česká republika, konkrétne to musí byť v tvare CZ}  
\item{\start\bf okres a cisloOkresu \stop musia byť z číselníka okresov a číslo musí byť správne napárované k okresu (zaistené aplikačne)}  
\stopitemize


%\obrazekH{obr:post}
%{Telo POST requestu zaregistrujNastupZamestnance (vlastné spracovanie)}
%{images/post_req.png}
%{width=32cc} 

Pri správnych hodnotách a dátových typoch je žiadosť úspešne založená (viď \in{obrázok}[obr:zadost-pred]) a je možné prejsť na nasledujúci krok a to overenie žiadosti. Táto akcia sa spúšťa tlačidlom na detaile žiadosti, na ktorú je možné sa dostať z dlaždice alebo z detailu zamestnanca. 

Kliknutím na tlačidlo \uv{Ověř stav žádosti} sa spustí akcia CtiVysledekZaregistrujNastupZamestnance. Ide o GET metódu, ktorá má vstupný parameter UUID žiadosti a vracia 2 objekty -- stavCSSZ a stavZP. Na základe výsledku tejto akcie sa menia atribúty žiadosti cl_stavZadostiCssz a cl_stavZadostiZp, poprípade chybové stavy (viď \in{obrázok}[obr:zadost-po]).

\obrazekH{obr:zadost-pred}
{Časť detailu založenej žiadosti (vlastné spracovanie)}
{images/stepper/zadost-pred.png}
{width=24cc}

\obrazekH{obr:zadost-po}
{Časť detailu overenej žiadosti (vlastné spracovanie)}
{images/stepper/zadost-po.png}
{width=24cc}  


\ppkap{Správa číselníkov}

Ďalšou dôležitou dlaždicou je správa číselníkov. Niektoré číselníky boli databázovo naplnené a zaindexované. Ide o číselníky, ktoré sú rovnaké naprieč spoločnosťami ako napríklad pohlavie, kraje, obce a pod. České adresné miesta boli importované z RUIANu.

Niektoré sú natoľko špecifické, že si to bude každá firma zadávať podľa seba. Preto je v časti Správa číselníkov možnosť zadávať špecifické číselníkové hodnoty na číselníky ako napríklad divízia, typ úväzku, pozícia a podovne.

Aktuálne sú v aplikácii dve oblasti číselníkov, no je to plne prispôsobiteľné podľa potrieb zákazníka. Na \in{obrázku}[obr:fin-cl1] je úroveň dlaždíc s výberom oblasti. V ľavej časti možno vidieť aj navigačnú lištu pre jednoduchšie ovládanie. 

V ďalšej úrovni (viď \in{obrázok}[obr:fin-cl2] sa zobrazí zoznam konfigurovateľných číselníkov. V poslednej úrovni je už zoznam konkrétnych hodnôt v číselníku, ktoré možno pomocou editačnej ceruzky editovať alebo pomocou plus pridávať nové hodnoty. To prebieha pomocou klasického editačného formuláru, ktorý bol spomínaný vyššie.

\obrazekH{obr:fin-cl1}
{Obrazovka -- správa číselníkov (vlastné spracovanie)}{images/fin/cl1.png}{width=32cc} 

\obrazekH{obr:fin-cl2}
{Obrazovka -- správa číselníkov -- zoznam číselníkov (vlastné spracovanie)}{images/fin/cl2.png}{width=32cc} 

\obrazekH{obr:fin-cl3}
{Obrazovka -- správa číselníkov -- zoznam číselníkových hodnôt (vlastné spracovanie)}{images/fin/cl3.png}{width=32cc} 


\ppkap{Zamestnávatelia a osoby}

Dlaždice zamestnávateľov a osôb sú najmä evidenčného charakteru. Pod zamestnávateľom sú informácie o konkrétnej firme, ktorá je zákazníkom produktu SAMO EMMA. Obsahuje všetky potrebné atribúty nutné pre komunikáciu s verejnou správou. Umožňuje zobraziť jednak základné údaje o zamestnávateľovi, ale aj jeho spojenia, adresy a prílohy. Pri osobách možno evidovať taktiež všetky spomenuté údaje.


\pkap{Zabezpečenie a bezpečnosť}

Aplikácia obsahuje citlivé údaje o zamestnancoch a musí spĺňať prísne práv\-ne požiadavky na ochranu osobných údajov. Tieto zákony sú navrhnuté na ochranu súkromia jednotlivcov a zaisťovanie zodpovedného zaobchádzania\zlom s ich osobnými informáciami.

Najdôležitejší právny predpis v rámci EÚ je Všeobecné nariadenie o ochrane údajov (GDPR). GDPR vyžaduje, aby organizácie mali právny základ pre spracúvanie osobných údajov, poskytli transparentnosť o tom, ako sa údaje používajú, a zabezpečili práva subjektov údajov, ako je právo na prístup, opravu a vymazanie ich osobných údajov.

Zamestnanci a osoby, ktoré sa do systému dostanú musia so zákonom GDPR súhlasiť, čo sa spravidla deje pri podpise zmluvy. Keďže je systém plný rôznych citlivých údajov, je nutné ho mať dostatočne zabezpečený.

Aplikácia SAMO EMMA je dostupná len pomocou VPN vybranej firmy, v~prípade tejto práce pod VPN firmy Asseco, CE. Tento bezpečnostný mechanizmus chráni citlivé firemné údaje pred neoprávneným prístupom tým, že vytvára bezpečný \uv{tunel} medzi zariadením používateľa a firemnou sieťou. VPN používa kombináciu šifrovacích protokolov a autentizačných metód.

Ďalším krokom zabezpečenia je užívateľské meno a heslo, ktoré musí spĺňať určité podmienky. Konkrétne heslo musí mať minimalně 8 znakov a musí obsahovať veľké, malé písmená a číselný znak. Diakritika a špeciálne znaky sú zakázané. Práve vďaka zabezpečeniu pomocou VPN nie je systém zabezpečený certifikátom alebo inou, bezpečnejšou metódou, ale len menom a heslom.

Čo sa týka zabezpečenia funkcií EMMA, ktoré vytvárajú žiadosť o nástupe zamestnanca a overujú túto žiadosť, zabezpečené je to pomocou Azure Blockchain Service.

\pkap{Testovanie}

Testovanie aplikácie prebieha pomocou manuálnych testov. Je to štandardný proces testovania SAMO aplikácií v rámci spoločnosti Asseco, CE. 

Testovanie softvéru je dôležitou fázou projektov a prináša niekoľko dôležitých výhod. Pomocou testovania sa overuje, či aplikácia funguje podľa očakávaní. Prínosom testovania je aj identifikácia chýb a nedostatkov v raných fázach vývoja, čo znižuje náklady na ich neskoršie opravy. Okrem toho znižuje riziká súvisiace s kvalitou, bezpečnosťou a výkonom softvéru. \scr(IEEE,2024)

Manuálne testovanie prebieha pomocou testovacích scenárov. Scenáre sú vytvorené podľa štandardov firmy Asseco. Sú popísané tabuľkami s tromi stĺpcami -- krok, akcia a očakávaný výsledok. Všetky tabuľky sa nachádzajú v časti Prílohy, konkrétne tabuľky \in[tab:test1] až \in[tab:test6].

\kap{Diskusia a záver}
\pkap{Zhrnutie výsledkov}
V tejto práci bola dôkladne analyzovaná súčasná situácia eGovernmentu v~Česku a na medzinárodnej úrovni. Boli odhalené nedostatky a ukázali sa možnosti rozvoja. Hlavnými výzvami, ktorým ČR čelí v oblasti digitalizácie sú spojené s konektivitou, integráciou digitálnych technológií a poskytovaním digitálnych verejných služieb.

Keďže je projekt EMMA stále vo fáze vývoja, zatiaľ je ťažké určiť dopad a prínos tohto projektu pre český eGovernment a spoločnosť. Je ale možné konštatovať, že platforma EMMA má v sebe obrovský potenciál a môže byť výrazným krokom vpred v oblasti digitalizácie. 

Či už ide o samotnú platformu EMMA alebo jej použitie na platforme SAMO, projekt dokáže pomôcť firme Asseco Central Europe, a.s. zlepšiť svoju pozíciu na trhu. 

Hlavným modulom vyvíjaným v rámci tejto práce bol modul zamestnancov v SAMO. To výrazne uľahčí a zefektívni prezentáciu služieb EMMA potenciálnym zákazníkom a zvýši atraktivitu ako platformy EMMA, tak aj platformy SAMO. 

Všetky poznatky a produkty tejto práce môžu slúžiť na zlepšenie pozície na trhu a zvýšenie konkurencieschopnosti firmy Asseco a zlepšenie procesov digitalizácie v ČR.

\pkap{Možné vylepšenia}
V rámci ďalšieho rozvoja existuje množstvo oblastí, ktoré by sa dali vylepšiť.

Čo sa týka platformy EMMA, existuje veľké množstvo služieb verejnej sprá\-vy, ktoré by sa dali navrhnúť a implementovať podobne ako služba \uv{nástup zamestnanca}. Tým sa zväčší rozsah projektu a zároveň jeho atraktivita na trhu. Príkladom služieb môže byť napríklad opačný proces, a teda ukončenie pomeru zamestnanca, hromadné oznámenie zamestnávateľa, služby spojené so zamestnávaním zdravotne ťažko postihnutých zamestnancov a podobne. Následne je možné a aj plánované rozšíriť služby EMMA aj o služby komerčného sektoru.

Je pomerne náročné prezentovať projekt bez prezentačnej vrstvy, a preto je modul zamestnancov a ďalšie moduly v SAMO dôležitá súčasť projektu. Po vytvorení ďalších služieb EMMA by bolo možné jednoduchšie zaujať potenciálnych zákazníkov. 

Platforma SAMO ponúka množstvo možností na rozšírenie, ako napríklad reporty, generovanie dokumentov pomocou údajov zo systému, rozšírenie o ďalšie moduly potrebné k správe zamestnancov a podobne.


\pkap{Záver}
Na základe predom definovaného cieľa práce boli prevedené všetky kroky k~jeho naplneniu. Hlavným cieľom bolo rozšírenie stávajúceho riešenia projektu EMMA podľa požiadaviek firmy Asseco Central Europe, a.s. a zadania práce.

Súčasťou týchto krokov bola dôkladná literárna rešerš. V rámci tejto rešerše bola vykonaná analýza súčasného stavu eGovernmentu na základe vybraných prieskumov. Zároveň boli popísané základné pojmy a procesy spojené s digitalizáciou a digitálnymi službami. V poslednej časti bol analyzovaný súčasný stav projektu EMMA a teoretický základ technológií používaných v rámci tejto práce.

Následne nasledovala fáza analýzy a návrhu. V rámci tejto fázy boli vytvorené modely pomocou jazyka UML, ktoré graficky ilustrovali žiadané funkcionality neskôr implementovaného riešenia. Na základe analýzy bol vytvorený návrh pomocou drátených modelov.

Po analýze a návrhu sa prešlo na fázu implementácie. Šlo o rozšírenie aplikácie SAMO EMMA. Implementácia prebiehala pomocou platformy SAMO, interného \uv{frameworku} spoločnosti Asseco, CE. 

Posledným krokom bolo testovanie implementovaného riešenia. Testovanie prebehlo pomocou pripravených testovacích scenárov. 

Všetky procesy prevedené v rámci tejto práce boli v súlade s metodikou. Výsledkom je funkčný modul zamestnancov a ďalšie doplnkové moduly v SAMO EMMA, ktoré sú napojené na služby EMMA, konkrétne na službu Oznámení nástupu zaměstnace.
%%%%%%%%%%%%%%%%%%%%%%%%% \def\refname{}

\bbib

\publE{
\autor{Anderrsson, C.} \autor{Hallin, A.} \autor{Ivory, C.}
\nazev{Unpacking the digitalisation of public services: Configuring work during automation in local government}
\rok{2022}
\issn{0740624X}
\doi{10.1016/j.giq.2021.101662}
\www{https://www.sciencedirect.com/science/article/pii/S0740624X21000988}
\online{2023-11-19}
\nazevdok{Government Information Quarterly}
\cast{roč. \,39}
}

\publW{
\autor{Ardhaninggar, N.}
\nazev{E-Government Success Stories: Learning from Denmark and Estonia}
\rok{2023}
\www{https://moderndiplomacy.eu/author/nurulardhaninggar/}
\online{2024-01-31}
\nazevdok{moderndiplomacy.eu}
%\podnazev{Informační koncepce ČR}
}

\publX{
\autorkorp{Asseco Central Europe, a.s.}
\online{2023-11-27}
\www{interný SharePoint}
\nazev{SAMO conceptual application architecture}
\rok{2023}
}

\publX{
\autorkorp{Asseco Central Europe, a.s.}
\online{2024-02-09}
\www{interný dokument}
\nazev{SAMO Implementation Guide Version 9.4}
\rok{2024a}
}

\publX{
\autorkorp{Asseco Central Europe, a.s.}
\online{2024-03-03}
\www{https://www.samo-asseco.com/}
\nazev{SAMO – Platform for asset management solutions}
\rok{2024b}
}

\publX{
\autorkorp{Asseco Central Europe, a.s.}
\online{2023-11-26}
\www{interný dokument}
\nazev{Závěrečná zpráva o realizaci výsledků výzkumu a vývoje: VaV softwarové platformy embedded government (EMMA)}
\rok{2023}
}

\publE{
\autor{Barone, L. a kol.}
\nazev{State-of-play report on digital public administration and interoperability}
\rok{2023}
\isbn{978-92-68-08101-3}
\doi{10.2799/686251}
\www{https://op.europa.eu/en/publication-detail/-/publication/e2cf65a7-6719-11ee-9220-01aa75ed71a1/language-en}
\online{2024-1-12}
\nazevdok{Directorate-General for Informatics}
\cast{NO-04-23-973-EN-N}
}

\publX{
 \online{2024-04-17}
 \autor{COOLORS}
 \nazev{Image Picker}
 \www{https://coolors.co/image-picker}
 \rok{2024}
}

\publX{
 \online{2024-04-20}
 \autor{Česká správa sociálního zabezpečení}
 \nazev{web}
 \www{https://www.cssz.cz/web/cz}
 \rok{2024}
}

\publX{
 \online{\zlom2024-04-20}
 \autor{Česká správa sociálního zabezpečení}
 \nazev{Číselníky}
 \www{https://www.cssz.cz/ciselniky}
 \rok{2024}
}

\publX{
 \online{2024-04-17}
 \autor{Český úřad zeměměřický a katastrální}
 \nazev{RÚIAN (Registr územní identifikace, adres a nemovitostí)}
\podnazev{Číselníky ISÚI}
 \www{https://www.cuzk.cz/ruian/Poskytovani-udaju-ISUI-RUIAN-VDP/Ciselniky-ISUI.aspx}
 \rok{2024}
}

\publW{
\autorkorp{Digitální a informační agentura}
\nazev{Architektura eGovernmentu ČR}
\rok{2023}
\www{https://archi.gov.cz/start}
\online{2023-11-26}
\nazevdok{Národní architektonický plán}
\podnazev{Informační koncepce ČR}
}

\publW{
\autorkorp{Digitální a informační agentura}
\nazev{Architektura eGovernmentu ČR}
\rok{2023}
\www{https://archi.gov.cz/start}
\online{2023-11-26}
\nazevdok{Národní architektonický plán}
\podnazev{Katalog služeb veřejné správy}
}

\publW{
\autorkorp{Digitální a informační agentura}
\nazev{Architektura eGovernmentu ČR}
\rok{2023}
\www{https://archi.gov.cz/start}
\online{2023-11-26}
\nazevdok{Národní architektonický plán}
\podnazev{Slovník pojmů eGovernmentu}
}

\publW{
\autorkorp{Digitální a informační agentura}
\nazev{Architektura eGovernmentu ČR}
\rok{2023}
\www{https://archi.gov.cz/nap:zakladni_registry}
\online{2024-04-08}
\nazevdok{Národní architektonický plán}
\podnazev{Základní registry}
}


\publW{
 \online{2023-11-19}
 \autorkorp{Evropská komise.}
 \nazev{Balíček aktu o digitálních službách}
%\podnazev{Metodika}
 \www{https://digital-strategy.ec.europa.eu/cs/policies/digital-services-act-package}
 \nazevdok{Shaping Europe’s digital future}
 \rok{2022}
}

\publW{
 \online{2023-11-19}
 \autorkorp{Evropská komise.}
 \nazev{Index digitální ekonomiky a společnosti (DESI) 2022}
\podnazev{Česko}
 \www{https://digital-strategy.ec.europa.eu/en/policies/desi-czech-republic}
 \nazevdok{Shaping Europe’s digital future}
 \rok{2022}
}

\publW{
 \online{2023-11-19}
 \autorkorp{Evropská komise.}
 \nazev{Index digitální ekonomiky a společnosti (DESI) 2022}
\podnazev{Metodika}
 \www{https://digital-strategy.ec.europa.eu/cs/policies/desi}
 \nazevdok{Shaping Europe’s digital future}
 \rok{2022}
}

\publW{
 \autorkorp{FreeMarker}
 \nazev{FMPP - FreeMarker-based file PreProcessor}
 \online{2024-03-31}
 \www{https://fmpp.sourceforge.net/index.html}
 \nazevdok{Source Forge}
 \rok{2018}
}

\publX{
 \online{2024-04-17}
 \autor{GeeksForGeeks}
 \nazev{Unified Modeling Language (UML) Diagrams}
 \www{https://www.geeksforgeeks.org/unified-modeling-language-uml-introduction/}
 \rok{2024}
}

\publW{
 \autorkorp{IEEE}
 \nazev{The Importance of Software Testing}
\podnazev{Unveiling the Significance and Strategies of Comprehensive Software Testing}
 \online{2024-05-08}
 \www{https://www.computer.org/resources/importance-of-software-testing}
 \nazevdok{IEEE COMPUTER SOCIETY}
 \rok{2024}
}

%\publA{
%\autor{OECD}
%\nazev{ Government at a Glance}
%\nakl{OECD Publishing}
%\vyd{2023}
%\rok{2023}
%\isbn{978-92-64-85180-1}
%\rozsah{60}
%}

%\publE{
% \autor{Karunia, L. a kol.}
% \nazev{Analysis of the Factors that Affect the Implementation of EGovernment in Indonesia}
% \nazevdok{International Journal of Membrane Science and Technology}
% \cast{Vol.\,10, No.\,3}
% \rok{2023}
% \umist{46}{54}
% %\issn{0896-3207}
%\doi{https://doi.org/10.1063/5.0118820}
%}

\publD{%
 \autor{Josey, A. a kol.}
 \nazev{An Introduction to the ArchiMate® 3.0 Specification}
 \nazevdok{The Open Group}
\vyd{W168}
\nakl{The Open Group}{USA}
\rok{2016}
%\www{http://www.enterprise-architecting.com/eaex/EA\%20-\%20An\%20Introduction\%20to\%20Archimate.pdf}
 \umist{5}{15}
}

\publW{
\autor{Kilinger, A.}
\nazev{Obligatory Slovakian Information System (IS EFA) for exchanging B2G and B2B E-Invoice}
 \online{2024-01-29}
 \www{https://blog.seeburger.com/new-obligatory-slovakian-information-system-is-efa-for-b2g-and-b2b-e-invoicing/}
\issn{978-92-64-85180-1}
 \nazevdok{SEEBURGER}
 \rok{2023}
}

\publE{
\autor{Looks, H. a kol.} 
\nazev{Towards a Process Model for Agile Transformation in E-government Projects}
\rok{2021}
\issn{2468-4376}
\doi{10.29333/jisem/9571}
\www{https://www.jisem-journal.com/article/towards-a-process-model-for-agile-transformation-in-e-government-projects-9571}
\online{2024-03-24}
\nazevdok{Journal of Information Systems Engineering & Management}
\cast{6(1)}
}


%\publE{
%\autor{Kisielnicki, J.} \autor{Misiak, A.}
%\nazev{Effectiveness of Agile Compared to Waterfall Implementation Methods in IT Projects}
%\rok{2017}
%\issn{2300-5661}
%\doi{10.1515/fman-2017-0021}
%\www{https://sciendo.com/article/10.1515/fman-2017-0021}
%\online{2023-11-19}
%\nazevdok{Foundations of Management}
%\cast{roč. \,9}
%}

%\publW{
% \autorkorp{Ministerstvo financií Slovenskej republiky}
% \nazev{Informačný systém elektronickej fakturácie - BETA}
% \online{2024-01-29}
% \www{https://web-einvoice-demo.mypaas.vnet.sk/}
% \nazevdok{e-Faktúra}
% \rok{©~2024}
%}

%\publE{
%\autor{Mishra, A.} \autor{Alzoubi, Y.} 
%\nazev{Structured software development versus agile software development: a comparative analysis}
%\rok{2022}
%\issn{0975-6809}
%\doi{10.1007/s13198-023-01958-5}
%\www{https://www.sciencedirect.com/science/article/pii/S0740624X21000988}
%\online{2023-11-19}
%\nazevdok{International Journal of System Assurance Engineering and Management}
%\cast{roč. 14, č. 4}
%}

\publX{
\autor{Mega, K.}
\nazev{osobné zdelenie}
\rok{27.3.2024}
}

\publA{
 \autor{OECD}
 \nazev{Government at a Glance}
 \nakl{Paris}{OECD Publishing}
 %\vyd{2023}
 \rok{2023}
 \xisbn{978-92-64-85180-1}
 \rozsah{234\stran}
}

\publX{
 \autorkorp{SAP}
 \nazev{What is integration platform as a service (iPaaS)?}
 \online{2024-03-03}
 \www{https://www.sap.com/products/technology-platform/integration-suite/what-is-ipaas.html}
 \rok{©~2024}
}

\publD{%
 \autorkorp{Shore, J. a kol.}
 \nazev{The art of agile development}
 \nazevdok{Theory in practice}
\nakl{O'Reill}{Boston}
\vyd{2}
\rok{2022}
\isbn{9781492080695}
 \umist{3}{11}
}

\publW{
 \autorkorp{Sparx Systems}
 \nazev{Model Driven Generation (MDG) Technologies}
 \online{2024-03-31}
 \www{https://sparxsystems.com/resources/mdg_tech/}
 \nazevdok{Sparx Systems}
 \rok{©~2000 - 2024}
}

\publE{
\autor{Terlizzi, A.} 
\nazev{The Digitalization of the Public Sector: A Systematic Literature Review}
\rok{2021}
\xissn{ 1722-1137}
\doi{10.1483/100372}
\www{https://www.researchgate.net/publication/351069477_The_Digitalization_of_the_Public_Sector_A_Systematic_Literature_Review}
\online{2024-05-04}
\nazevdok{Research Gate}
\cast{16(1)}
}

\publW{
 \autorkorp{Tradinos}
 \nazev{The Importance of System Analysis in Software Development}
 \online{2024-05-08}
 \www{https://www.linkedin.com/pulse/importance-system-analysis-software-development-tradinos/}
 \nazevdok{Linkedin}
 \rok{2023}
}

\publD{%
 \autorkorp{United Nations}
 \nazev{E-Government Survey 2022}
 \nazevdok{The Future of Digital Government}
\nakl{UN}{New York}
\rok{2022}
\isbn{978-92-1-123213-4}
 \umist{32}{51}
}

%https://desapublications.un.org/sites/default/files/publications/2022-09/Web%20version%20E-Government%202022.pdf

\publW{
 \autorkorp{Úřad vlády ČR}
 \nazev{Tři pilíře Digitálního Česka}
 \online{2024-03-01}
 \www{https://digitalnicesko.gov.cz/vize/}
 \nazevdok{Digitální Česko}
 \rok{\zlom©~2024}
}

%\publE{
%\autor{van der Linden, N. a kol.}
%\nazev{eGovernment Benchmark 2023: Insight Report}
%\rok{2023}
%\isbn{978-92-68-05653-0}
%\doi{10.2759/474056}
%\www{https://espanadigital.gob.es/sites/espanadigital/files/2023-10/1_eGovernment_Benchmark_2023__Insight_Report_tmnnsE9rmVDxpAZ8IJECpnUZGLA_98708.pdf}
%\online{2024-2-7}
%\nazevdok{Connecting Digital Governments}
%\cast{KK-BH-23-001-EN-N}
%}

\publE{
\autor{van der Linden, N. a kol.}
\nazev{eGovernment Benchmark 2022: Insight Report}
\rok{2022}
\isbn{ 978-92-76-49793-6}
\doi{10.2759/488218}
\www{https://prod.ucwe.capgemini.com/wp-content/uploads/2022/07/eGovernment-Benchmark-2022-1.-Insight-Report.pdf}
\online{2024-2-7}
\nazevdok{Connecting Digital Governments}
\cast{KK-08-22-084-EN-N}
}

\publA{
 \autor{Wang, K.C.}
 \nazev{Embedded and Real-Time Operating Systems}
 \nakl{Cham}{Springer International Publishing}
 \vyd{1}
 \rok{2017}
 \isbn{978-3-319-51517-5}
 \rozsah{481\stran}
}

\publX{
\autor{Winkler, J.}
\nazev{osobné zdelenie}
\rok{25.2.2024}
}

\ebib

\stopbodymatter

%%%%%%%%%%%%%%%%%%%%%%%% Varianta, kdy seznamy jsou součástí práce a nejsou uvedeny v přílohách

\setupsectionblock[backmatter][before={\setuplist[kap][before={}]}]

\startbackmatter

\THESIScompletelistof{tables}
\THESIScompletelistof{figures}
%\THESIScompletelistof{abbreviations}
%\THESIScompletelistof{codes}

\stopbackmatter

%%%%%%%%%%%%%%%%%%%%%%%% Varianta, kdy seznamy nejsou součástí práce, ale jsou zařazeny do příloh.
%%%%%%%%%%%%%%%%%%%%%%%% Níže uvedeným čtyřem příkazům postačí odstranit znak procenta.
%%%%%%%%%%%%%%%%%%%%%%%% Naopak před výše uvedené čtyři příkazy je potřeba znak procenta vložit.
%
\startappendices

\cast{PRÍLOHY}

\setuplabeltext[appendix=~]

\kap{Aplikačná vrstva EMMA}

\obrazekH{obr:archimate}
{EMMA - Aplikačná vrstva (vlastné spracovanie na základe informácií z interných dokumentov firmy Asseco)}{images/archimate.png}{width=30cc}

\kap{Model požiadaviek}

\obrazekH{obr:f-req}
{Model požiadaviek (vlastné spracovanie na základe informácií z interných dokumentov firmy Asseco a analýzy)}{images/f_req2.png}{width=32cc}

\kap{Diagramy aktivít}

\obrazekH{obr:acti3}
{Diagram aktivít pre UC Prihlásiť sa do systému (vlastné spracovanie)}{images/ActivityPrihlasenie.png}{width=30cc}

\obrazekH{obr:acti1}
{Diagram aktivít pre UC Vytvoriť žiadosť o registráciu povereného zamestnanca (vlastné spracovanie)}{images/ActivityRegistracia_zadost.png}{width=30cc}

\obrazekH{obr:acti2}
{Diagram aktivít pre UC Registrovať sa do systému (vlastné spracovanie)}{images/ActivityRegistracia.png}{width=30cc}

\obrazekH{obr:acti-cl}
{Diagram aktivít pre UC Spravovať číselníky (vlastné spracovanie)}{images/ActivityCl.png}{width=30cc}

\obrazekH{obr:acti-nastup}
{Diagram aktivít pre UC Zadať nového zamestnanca do systému (vlastné spracovanie)}{images/ActivityNastup.png}{width=30cc}

\obrazekH{obr:acti-overeni}
{Diagram aktivít pre UC Registrovať nástup zamestnanca (vlastné spracovanie)}{images/ActivityOvereni.png}{width=30cc}


\kap{Entity, atribúty a väzby}

\setupTABLE[frame=on]
\setupTABLE[row][first][background=color, backgroundcolor=lightgray, style=bold]
\setupTABLE[column][1][width=12cc]
\setupTABLE[column][2][width=12cc]
\setupTABLE[column][3][width=8cc]
\setupTABLE[r][each][align={middle,lohi}]
\Tabulka{tab:system-att}{Prehľad systémových atribútov (vlastné spracovanie)}{
\bTABLE
  \bTR
    \bTH Kód \eTH
    \bTH Názov \eTH
    %\bTH Krátky popis \eTH
    \bTH Povinný? \eTH
  \eTR
  \bTR
    \bTD createDate \eTD
    \bTD Datum vytvoření \eTD
    %\bTD \eTD
    \bTD Áno \eTD
  \eTR
  \bTR
    \bTD createdBy \eTD
    \bTD Vytvořeno kým \eTD
    %\bTD \eTD
    \bTD Áno \eTD
  \eTR
  \bTR
    \bTD featureInfo \eTD
    \bTD Uživatelský identifikátor \eTD
    %\bTD \eTD
    \bTD Nie \eTD
  \eTR
  \bTR
    \bTD ftid \eTD
    \bTD Feature Type \eTD
    %\bTD \eTD
    \bTD Nie \eTD
  \eTR
  \bTR
    \bTD id \eTD
    \bTD Feature ID \eTD
    %\bTD \eTD
    \bTD Áno \eTD
  \eTR
  \bTR
    \bTD longTransactionId \eTD
    \bTD Long_trans_ID \eTD
    %\bTD \eTD
    \bTD Nie \eTD
  \eTR
  \bTR
    \bTD sid \eTD
    \bTD Semantic ID \eTD
    %\bTD \eTD
    \bTD Nie \eTD
  \eTR
  \bTR
    \bTD symbologyTokens \eTD
    \bTD Symbology tokens \eTD
    %\bTD \eTD
    \bTD Nie \eTD
  \eTR
  \bTR
    \bTD updateDate \eTD
    \bTD Datum modifikace \eTD
    %\bTD \eTD
    \bTD Nie \eTD
  \eTR
  \bTR
    \bTD updatedBy \eTD
    \bTD Modifikováno kým \eTD
    %\bTD \eTD
    \bTD Nie \eTD
  \eTR
  \bTR
    \bTD validFrom \eTD
    \bTD Platnost OD \eTD
    %\bTD \eTD
    \bTD Nie \eTD
  \eTR
  \bTR
    \bTD validTo \eTD
    \bTD Platnost DO \eTD
    %\bTD \eTD
    \bTD Nie \eTD
  \eTR
\eTABLE
}


\setupTABLE[frame=on]
\setupTABLE[row][first][background=color, backgroundcolor=lightgray, style=bold]
\setupTABLE[column][1][width=9cc]
\setupTABLE[column][2][width=9cc]
\setupTABLE[column][3][width=6cc]
\setupTABLE[column][4][width=5cc]
\setupTABLE[column][5][width=3cc]
\setupTABLE[r][each][align={middle,lohi}]
\Tabulka{tab:PO1}{Prehľad atribútov právnickej osoby (vlastné spracovanie)}{
\bTABLE
  \bTR
    \bTH Názov \eTH
    \bTH DB Name \eTH
    \bTH Dátový typ \eTH
    \bTH Povinný? \eTH
    \bTH Typ \eTH
  \eTR
  \bTR
    \bTD Dátum vzniku \eTD
    \bTD DATUMVZNIKU \eTD
    \bTD Date \eTD
    \bTD Nie \eTD
    \bTD F \eTD
  \eTR
    \bTR
    \bTD Dátum zániku \eTD
    \bTD DATUMZANIKU \eTD
    \bTD Date \eTD
    \bTD Nie \eTD
    \bTD F \eTD
  \eTR
  \bTR
    \bTD DIČ \eTD
    \bTD DIC \eTD
    \bTD String (100) \eTD
    \bTD Nie \eTD
    \bTD F \eTD
  \eTR
  \bTR
    \bTD IČO \eTD
    \bTD ICO \eTD
    \bTD String (100) \eTD
    \bTD Áno \eTD
    \bTD F \eTD
  \eTR
  \bTR
    \bTD Název PO \eTD
    \bTD NAZEVPO \eTD
    \bTD String (100) \eTD
    \bTD Áno \eTD
    \bTD F \eTD
  \eTR
\eTABLE
}


\setupTABLE[frame=on]
\setupTABLE[row][first][background=color, backgroundcolor=lightgray, style=bold]
\setupTABLE[column][1][width=9cc]
\setupTABLE[column][2][width=9cc]
\setupTABLE[column][3][width=6cc]
\setupTABLE[column][4][width=5cc]
\setupTABLE[column][5][width=3cc]
\setupTABLE[r][each][align={middle,lohi}]
\Tabulka{tab:FO1}{Prehľad atribútov fyzickej osoby (vlastné spracovanie)}{
\bTABLE
  \bTR
    \bTH Názov \eTH
    \bTH DB Name \eTH
    \bTH Dátový typ \eTH
    \bTH Povinný? \eTH
    \bTH Typ \eTH
  \eTR
  \bTR
    \bTD Souhlas s GDPR? \eTD
    \bTD AN_SOUHLASGDPR \eTD
    \bTD cl_anoNe \eTD
    \bTD Áno \eTD
    \bTD F \eTD
  \eTR
  \bTR
    \bTD Student? \eTD
    \bTD AN_STUDENT \eTD
    \bTD cl_anoNe \eTD
    \bTD Áno \eTD
    \bTD F \eTD
  \eTR
  \bTR
    \bTD ZTP? \eTD
    \bTD AN_ZTP \eTD
    \bTD cl_anoNe \eTD
    \bTD Áno \eTD
    \bTD F \eTD
  \eTR
  \bTR
    \bTD ZTP/P \eTD
    \bTD AN_ZTPP \eTD
    \bTD cl_anoNe \eTD
    \bTD Áno \eTD
    \bTD F \eTD
  \eTR
  \bTR
    \bTD Souhlas s GDPR - časové razítko \eTD
    \bTD RAZITKOGDPR \eTD
    \bTD Date \& time \eTD
    \bTD Nie \eTD
    \bTD F \eTD
  \eTR
  \bTR
    \bTD Číslo občanského průkazu \eTD
    \bTD CISLOOP \eTD
    \bTD String (50) \eTD
    \bTD Nie \eTD
    \bTD F \eTD
  \eTR
  \bTR
    \bTD Pohlaví \eTD
    \bTD CL_POHLAVI \eTD
    \bTD cl_pohlavi  \eTD
    \bTD Áno \eTD
    \bTD F \eTD
  \eTR
  \bTR
    \bTD Státní občanství \eTD
    \bTD CL_STATNIOBCANSTVI  \eTD
    \bTD cl_statniObcanstvi \eTD
    \bTD Nie \eTD
    \bTD F \eTD
  \eTR
  \bTR
    \bTD Datum aktivace učtu \eTD
    \bTD DATUMAKTIVACE \eTD
    \bTD Date \& time \eTD
    \bTD Nie \eTD
    \bTD F \eTD
  \eTR
  \bTR
    \bTD Datum narození \eTD
    \bTD DATUMNAROZENI \eTD
    \bTD Date  \eTD
    \bTD Áno \eTD
    \bTD F \eTD
  \eTR
  \bTR
    \bTD Jméno \eTD
    \bTD JMENO \eTD
    \bTD String (50) \eTD
    \bTD Áno \eTD
    \bTD F \eTD
  \eTR
  \bTR
    \bTD Aktivační klíč \eTD
    \bTD KLICAKTIVACE \eTD
    \bTD String (50) \eTD
    \bTD Nie \eTD
    \bTD F \eTD
  \eTR
  \bTR
    \bTD Místo narození \eTD
    \bTD MISTONAROZENI \eTD
    \bTD String (100) \eTD
    \bTD Nie \eTD
    \bTD F \eTD
  \eTR
  \bTR
    \bTD Poznámka \eTD
    \bTD POZNAMKA \eTD
    \bTD String (4000) \eTD
    \bTD Nie \eTD
    \bTD F \eTD
  \eTR
  \bTR
    \bTD Příjmení \eTD
    \bTD PRIJMENI \eTD
    \bTD String (50) \eTD
    \bTD Áno \eTD
    \bTD F \eTD
  \eTR
  \bTR
    \bTD Kontaktní email \eTD
    \bTD PRIMALNIMAIL \eTD
    \bTD String (150) \eTD
    \bTD Nie \eTD
    \bTD D \eTD
  \eTR
  \bTR
    \bTD Rodné číslo \eTD
    \bTD RODNECISLO \eTD
    \bTD String (10) \eTD
    \bTD Nie \eTD
    \bTD F \eTD
  \eTR
  \bTR
    \bTD Stav aktivace účtu \eTD
    \bTD STAVAKTIVACE \eTD
    \bTD String (50) \eTD
    \bTD Nie \eTD
    \bTD F \eTD
  \eTR
  \bTR
    \bTD Titul před jménem \eTD
    \bTD TITULPREDJMENEM \eTD
    \bTD String (32) \eTD
    \bTD Nie \eTD
    \bTD F \eTD
  \eTR
  \bTR
    \bTD Titul za jménem \eTD
    \bTD TITULZAJMENEM \eTD
    \bTD String (32) \eTD
    \bTD Nie \eTD
    \bTD F \eTD
  \eTR
  \bTR
    \bTD Úplné jméno osoby  \eTD
    \bTD UPLNEJMENOOSOBY \eTD
    \bTD String (150) \eTD
    \bTD Nie \eTD
    \bTD D \eTD
  \eTR
  \bTR
    \bTD ID uživatele  \eTD
    \bTD USERID \eTD
    \bTD Number (8, 0) \eTD
    \bTD Nie \eTD
    \bTD F \eTD
  \eTR
  \bTR
    \bTD Věk \eTD
    \bTD VEK \eTD
    \bTD Number (8, 0) \eTD
    \bTD Nie \eTD
    \bTD D \eTD
  \eTR
\eTABLE
}


\setupTABLE[frame=on]
\setupTABLE[row][first][background=color, backgroundcolor=lightgray, style=bold]
\setupTABLE[column][1][width=15cc]
\setupTABLE[column][2][width=12cc]
\setupTABLE[column][3][width=5cc]
\setupTABLE[r][each][align={middle,lohi}]
\Tabulka{tab:FOPO}{Prehľad väzieb fyzickej i právnickej osoby (vlastné spracovanie)}{
\bTABLE
  \bTR
    \bTH ID \eTH
    \bTH Popis \eTH
    \bTH Násobnosť \eTH
  \eTR
  \bTR
    \bTD as_osoba_spojeni \eTD
    \bTD osoba -> spojeni \eTD
    \bTD 1\,:\,N \eTD
  \eTR
   \bTR
    \bTD as_osoba_zamestnavatel \eTD
    \bTD osoba -> zamestnavatel \eTD
    \bTD 1\,:\,N \eTD
  \eTR  
  \bTR
    \bTD as_osoba_adresa \eTD
    \bTD osoba -> adresa \eTD
     \bTD 1\,:\,N \eTD
  \eTR  
  \bTR
    \bTD as_osoba_zadost \eTD
    \bTD osoba -> zadost \eTD
    \bTD 1\,:\,N \eTD
  \eTR
\eTABLE
}



\setupTABLE[frame=on]
\setupTABLE[row][first][background=color, backgroundcolor=lightgray, style=bold]
\setupTABLE[column][1][width=9cc]
\setupTABLE[column][2][width=9cc]
\setupTABLE[column][3][width=6cc]
\setupTABLE[column][4][width=5cc]
\setupTABLE[column][5][width=3cc]
\setupTABLE[r][each][align={middle,lohi}]
\Tabulka{tab:spojeni1}{Prehľad atribútov entity spojení (vlastné spracovanie)}{
\bTABLE
  \bTR
    \bTH Názov \eTH
    \bTH DB Name \eTH
    \bTH Dátový typ \eTH
    \bTH Povinný? \eTH
    \bTH Typ \eTH
  \eTR
  \bTR
    \bTD Primární? \eTD
    \bTD PRIMARNI \eTD
    \bTD cl_anoNe \eTD
    \bTD Nie \eTD
    \bTD F \eTD
  \eTR
    \bTR
    \bTD Číslo bankovního účtu \eTD
    \bTD CISLOBANKOVNIHOUCTU \eTD
    \bTD String (100) \eTD
    \bTD Nie \eTD
    \bTD F \eTD
  \eTR
  \bTR
    \bTD Datová schránka \eTD
    \bTD DATOVASCHRANKA \eTD
    \bTD String (100) \eTD
    \bTD Nie \eTD
    \bTD F \eTD
  \eTR
  \bTR
    \bTD Email \eTD
    \bTD EMAIL \eTD
    \bTD String (100) \eTD
    \bTD Nie \eTD
    \bTD F \eTD
  \eTR
  \bTR
    \bTD Fyzická osoba \eTD
    \bTD FR_FYZICKAOSOBA \eTD
    \bTD as_fyzickaOsoba_spojeni \eTD
    \bTD Nie \eTD
    \bTD V \eTD
  \eTR
  \bTR
    \bTD Právnická osoba \eTD
    \bTD FR_PRAVNICKAOSOBA \eTD
    \bTD as_pravnickaOsoba_spojeni \eTD
    \bTD Nie \eTD
    \bTD V \eTD
  \eTR
  \bTR
    \bTD Jiné \eTD
    \bTD JINE \eTD
    \bTD String (100) \eTD
    \bTD Nie \eTD
    \bTD F \eTD
  \eTR
  \bTR
    \bTD Platnost do \eTD
    \bTD PLATNOSTDO \eTD
    \bTD Date \eTD
    \bTD Nie \eTD
    \bTD F \eTD
  \eTR
  \bTR
    \bTD Platnost od \eTD
    \bTD PLATNOSTOD \eTD
    \bTD Date \eTD
    \bTD Nie \eTD
    \bTD F \eTD
  \eTR
  \bTR
    \bTD Telefon \eTD
    \bTD TELEFON \eTD
    \bTD String (100) \eTD
    \bTD Nie \eTD
    \bTD F \eTD
  \eTR
\eTABLE
}


\setupTABLE[frame=on]
\setupTABLE[row][first][background=color, backgroundcolor=lightgray, style=bold]
\setupTABLE[column][1][width=15cc]
\setupTABLE[column][2][width=12cc]
\setupTABLE[column][3][width=5cc]
\setupTABLE[r][each][align={middle,lohi}]
\Tabulka{tab:spojeni2}{Prehľad väzieb entity spojení (vlastné spracovanie)}{
\bTABLE
  \bTR
    \bTH ID \eTH
    \bTH Popis \eTH
    \bTH Násobnosť \eTH
  \eTR
  \bTR
    \bTD as_pravnickaOsoba_spojeni \eTD
    \bTD pravnickaOsoba -> spojeni \eTD
    \bTD 1\,:\,N \eTD
  \eTR
   \bTR
    \bTD as_fyzickaOsoba_spojeni \eTD
    \bTD fyzickaOsoba -> spojeni \eTD
    \bTD 1\,:\,N \eTD
  \eTR  
\eTABLE
}



\setupTABLE[frame=on]
\setupTABLE[row][first][background=color, backgroundcolor=lightgray, style=bold]
\setupTABLE[column][1][width=9cc]
\setupTABLE[column][2][width=9cc]
\setupTABLE[column][3][width=6cc]
\setupTABLE[column][4][width=5cc]
\setupTABLE[column][5][width=3cc]
\setupTABLE[r][each][align={middle,lohi}]
\Tabulka{tab:adresa1}{Prehľad atribútov entity adresa (vlastné spracovanie)}{
\bTABLE
  \bTR
    \bTH Názov \eTH
    \bTH DB Name \eTH
    \bTH Dátový typ \eTH
    \bTH Povinný? \eTH
    \bTH Typ \eTH
  \eTR
  \bTR
    \bTD Číslo evidenční \eTD
    \bTD CISLOEVIDENCNI \eTD
    \bTD Number (28, 0) \eTD
    \bTD Nie \eTD
    \bTD F \eTD
  \eTR
    \bTR
    \bTD Číslo orientační \eTD
    \bTD CISLOORIENTACNI \eTD
    \bTD String (100) \eTD
    \bTD Nie \eTD
    \bTD F \eTD
  \eTR
  \bTR
    \bTD Číslo popisné \eTD
    \bTD CISLOPOPISNE \eTD
    \bTD Number (28, 0) \eTD
    \bTD Nie \eTD
    \bTD F \eTD
  \eTR
  \bTR
    \bTD Část obce \eTD
    \bTD CL_CASTOBCE \eTD
    \bTD cl_castObce \eTD
    \bTD Nie \eTD
    \bTD F \eTD
  \eTR
  \bTR
    \bTD Druh adresy \eTD
    \bTD CL_DRUHADRESY \eTD
    \bTD  cl_druhAdresy \eTD
    \bTD Áno \eTD
    \bTD F \eTD
  \eTR
  \bTR
    \bTD Kraj \eTD
    \bTD CL_KRAJ \eTD
    \bTD cl_kraj \eTD
    \bTD Nie \eTD
    \bTD F \eTD
  \eTR
  \bTR
    \bTD Obec \eTD
    \bTD CL_OBEC \eTD
    \bTD cl_obec \eTD
    \bTD Nie \eTD
    \bTD F \eTD
  \eTR
  \bTR
    \bTD Stát \eTD
    \bTD CL_STAT \eTD
    \bTD cl_stat \eTD
    \bTD Nie \eTD
    \bTD F \eTD
  \eTR
  \bTR
    \bTD Ulice \eTD
    \bTD CL_ULICE \eTD
    \bTD cl_ulice \eTD
    \bTD Nie \eTD
    \bTD F \eTD
  \eTR
\bTR
    \bTD Druhý řádek \eTD
    \bTD DRUHYRADEK \eTD
    \bTD String (100) \eTD
    \bTD Nie \eTD
    \bTD F \eTD
  \eTR
  \bTR
    \bTD Adresní místo \eTD
    \bTD FR_ADRESNIMISTO \eTD
    \bTD as_adresniMisto_adresa \eTD
    \bTD Nie \eTD
    \bTD V \eTD
  \eTR
  \bTR
    \bTD Fyzická osoba \eTD
    \bTD FR_FYZICKAOSOBA \eTD
    \bTD as_fyzickaOsoba_adresa \eTD
    \bTD Nie \eTD
    \bTD V \eTD
  \eTR
  \bTR
    \bTD Právnická osoba \eTD
    \bTD FR_PRAVNICKAOSOBA \eTD
    \bTD as_pravnickaOsoba_adresa \eTD
    \bTD Nie \eTD
    \bTD V \eTD
  \eTR
  \bTR
    \bTD Platnost do \eTD
    \bTD PLATNOSTDO \eTD
    \bTD Date \eTD
    \bTD Nie \eTD
    \bTD F \eTD
  \eTR
  \bTR
    \bTD První řádek \eTD
    \bTD PRVNIRADEK \eTD
    \bTD String (100)  \eTD
    \bTD Nie \eTD
    \bTD F \eTD
  \eTR
  \bTR
    \bTD Úplná adresa \eTD
    \bTD UPLNAADRESA \eTD
    \bTD String (100) \eTD
    \bTD Nie \eTD
    \bTD D \eTD
  \eTR
\eTABLE
}


\setupTABLE[frame=on]
\setupTABLE[row][first][background=color, backgroundcolor=lightgray, style=bold]
\setupTABLE[column][1][width=15cc]
\setupTABLE[column][2][width=12cc]
\setupTABLE[column][3][width=5cc]
\setupTABLE[r][each][align={middle,lohi}]
\Tabulka{tab:FOPO}{Prehľad väzieb entity adresa (vlastné spracovanie)}{
\bTABLE
  \bTR
    \bTH ID \eTH
    \bTH Popis \eTH
    \bTH Násobnosť \eTH
  \eTR
  \bTR
    \bTD as_adresniMisto_adresa \eTD
    \bTD adresniMisto -> adresa \eTD
    \bTD 1\,:\,N \eTD
  \eTR
  \bTR
    \bTD as_pravnickaOsoba_adresa \eTD
    \bTD pravnickaOsoba -> adresa \eTD
    \bTD 1\,:\,N \eTD
  \eTR
   \bTR
    \bTD as_fyzickaOsoba_adresa \eTD
    \bTD fyzickaOsoba -> adresa \eTD
    \bTD 1\,:\,N \eTD
  \eTR  
\eTABLE
}


\setupTABLE[frame=on]
\setupTABLE[row][first][background=color, backgroundcolor=lightgray, style=bold]
\setupTABLE[column][1][width=9cc]
\setupTABLE[column][2][width=9cc]
\setupTABLE[column][3][width=6cc]
\setupTABLE[column][4][width=5cc]
\setupTABLE[column][5][width=3cc]
\setupTABLE[r][each][align={middle,lohi}]
\Tabulka{tab:zamestnanec1}{Prehľad atribútov entity zamestnanec (vlastné spracovanie)}{
\bTABLE
  \bTR
    \bTH Názov \eTH
    \bTH DB Name \eTH
    \bTH Dátový typ \eTH
    \bTH Povinný? \eTH
    \bTH Typ \eTH
  \eTR
  \bTR
    \bTD Zaměstnání malého rozsahu? \eTD
    \bTD AN_ZAMESTNANIMALEHOROZSAHU \eTD
    \bTD cl_anoNe \eTD
    \bTD Nie \eTD
    \bTD F \eTD
  \eTR
    \bTR
    \bTD Divize \eTD
    \bTD CL_DIVIZE \eTD
    \bTD cl_divize \eTD
    \bTD Nie \eTD
    \bTD F \eTD
  \eTR
  \bTR
    \bTD Druh pracovní čínnosti \eTD
    \bTD CL_DRUHPRACOVNICINNOSTI \eTD
    \bTD cl_druhPracovniCinnosti\eTD
    \bTD Áno \eTD
    \bTD F \eTD
  \eTR
  \bTR
    \bTD Pozice \eTD
    \bTD CL_POZICE \eTD
    \bTD cl_pozice \eTD
    \bTD Nie \eTD
    \bTD F \eTD
  \eTR
  \bTR
    \bTD Typ úvazku \eTD
    \bTD CL_TYPUVAZKU \eTD
    \bTD cl_typUvazku \eTD
    \bTD Áno \eTD
    \bTD F \eTD
  \eTR
  \bTR
    \bTD Zdravotní pojišťovna \eTD
    \bTD CL_ZDRAVOTNIPOJISTOVNA \eTD
    \bTD cl_zdravotniPojistovna \eTD
    \bTD Áno \eTD
    \bTD F \eTD
  \eTR
  \bTR
    \bTD Datum nástupu \eTD
    \bTD DATUMNASTUPU \eTD
    \bTD Date \eTD
    \bTD Áno \eTD
    \bTD F \eTD
  \eTR
  \bTR
    \bTD Fyzická osoba \eTD
    \bTD FR_FYZICKAOSOBA \eTD
    \bTD as_fyzickaOsoba_zamestnanec \eTD
    \bTD Nie \eTD
    \bTD V \eTD
  \eTR
  \bTR
    \bTD Právnická osoba \eTD
    \bTD FR_PRAVNICKAOSOBA \eTD
    \bTD as_pravnickaOsoba_zamestnanec \eTD
    \bTD Nie \eTD
    \bTD V \eTD
  \eTR
  \bTR
    \bTD Identifikační číslo \eTD
    \bTD IDENTIFIKACNICISLO \eTD
    \bTD Number (28, 0) \eTD
    \bTD Áno \eTD
    \bTD F \eTD
  \eTR
  \bTR
    \bTD Id pojistného \eTD
    \bTD IDPOJISTNEHO \eTD
    \bTD Number (28, 0) \eTD
    \bTD Áno \eTD
    \bTD F \eTD
  \eTR
  \bTR
    \bTD Místo výkonu \eTD
    \bTD MISTOVYKONU \eTD
    \bTD String (100) \eTD
    \bTD Áno \eTD
    \bTD F \eTD
  \eTR
  \bTR
    \bTD Úplné jméno osoby  \eTD
    \bTD UPLNEJMENOOSOBY \eTD
    \bTD String (150) \eTD
    \bTD Nie \eTD
    \bTD D \eTD
  \eTR
\eTABLE
}


\setupTABLE[frame=on]
\setupTABLE[row][first][background=color, backgroundcolor=lightgray, style=bold]
\setupTABLE[column][1][width=15cc]
\setupTABLE[column][2][width=12cc]
\setupTABLE[column][3][width=5cc]
\setupTABLE[r][each][align={middle,lohi}]
\Tabulka{tab:zamestnanec2}{Prehľad väzieb entity zamestnanec (vlastné spracovanie)}{
\bTABLE
  \bTR
    \bTH ID \eTH
    \bTH Popis \eTH
    \bTH Násobnosť \eTH
  \eTR
  \bTR
    \bTD as_zamestnavatel_zamestnanec \eTD
    \bTD zamestnavatel -> zamestnanec \eTD
    \bTD 1\,:\,N \eTD
  \eTR
   \bTR
    \bTD as_fyzickaOsoba_zamestnanec \eTD
    \bTD fyzickaOsoba -> zamestnanec \eTD
    \bTD 1\,:\,N \eTD
  \eTR  
\eTABLE
}


\setupTABLE[frame=on]
\setupTABLE[row][first][background=color, backgroundcolor=lightgray, style=bold]
\setupTABLE[column][1][width=9cc]
\setupTABLE[column][2][width=9cc]
\setupTABLE[column][3][width=6cc]
\setupTABLE[column][4][width=5cc]
\setupTABLE[column][5][width=3cc]
\setupTABLE[r][each][align={middle,lohi}]
\Tabulka{tab:zamestnavatel1}{Prehľad atribútov entity zamestnavatel (vlastné spracovanie)}{
\bTABLE
  \bTR
    \bTH Názov \eTH
    \bTH DB Name \eTH
    \bTH Dátový typ \eTH
    \bTH Povinný? \eTH
    \bTH Typ \eTH
  \eTR
  \bTR
    \bTD Právnická osoba \eTD
    \bTD FR_PRAVNICKAOSOBA \eTD
    \bTD as_pravnickaOsoba_zamestnanec \eTD
    \bTD Nie \eTD
    \bTD V \eTD
  \eTR
  \bTR
    \bTD Identifikační číslo \eTD
    \bTD IDENTIFIKACNICISLO \eTD
    \bTD Number (28, 0) \eTD
    \bTD Áno \eTD
    \bTD F \eTD
  \eTR
  \bTR
    \bTD Název zaměstnavatele  \eTD
    \bTD NAZEV \eTD
    \bTD String (150) \eTD
    \bTD Nie \eTD
    \bTD D \eTD
  \eTR
  \bTR
    \bTD Variabilní symbol \eTD
    \bTD VARIABILNISYMBOL \eTD
    \bTD String (100) \eTD
    \bTD Áno \eTD
    \bTD F \eTD
  \eTR
\eTABLE
}


\setupTABLE[frame=on]
\setupTABLE[row][first][background=color, backgroundcolor=lightgray, style=bold]
\setupTABLE[column][1][width=15cc]
\setupTABLE[column][2][width=12cc]
\setupTABLE[column][3][width=5cc]
\setupTABLE[r][each][align={middle,lohi}]
\Tabulka{tab:zamestnavatel2}{Prehľad väzieb entity zamestnavatel (vlastné spracovanie)}{
\bTABLE
  \bTR
    \bTH ID \eTH
    \bTH Popis \eTH
    \bTH Násobnosť \eTH
  \eTR
  \bTR
    \bTD as_zamestnavatel_zamestnanec \eTD
    \bTD zamestnavatel -> zamestnanec \eTD
    \bTD 1\,:\,N \eTD
  \eTR
   \bTR
    \bTD as_pravnickaOsoba_zamestnavatel \eTD
    \bTD pravnickaOsoba -> zamestnavatel \eTD
    \bTD 1\,:\,N \eTD
  \eTR  
\eTABLE
}


\kap{Drátené modely}

\obrazekH{obr:drat-dashboard}
{Drátený model dashboardu (vlastné spracovanie)}{images/wire/cockpit.png}{width=32cc}

\obrazekH{obr:drat-dashboard2}
{Drátený model druhej úrovne dlaždíc (vlastné spracovanie)}{images/wire/dlazdice2.png}{width=32cc}

\obrazekH{obr:drat-browse-detail}
{Drátený model zoznamu entít a ich detailu (vlastné spracovanie)}{images/wire/browse-detail.png}{width=32cc}

\obrazekH{obr:drat-spravacl}
{Drátený model správy číselníkov -- zoznam editovateľných číselníkov (vlastné spracovanie)}{images/wire/sprava-cl.png}{width=32cc}

\obrazekH{obr:drat-konkretnycl}
{Drátený model správy číselníkov -- konkrétny číselník (vlastné spracovanie)}{images/wire/konkretny-cl.png}{width=32cc}

\obrazekH{obr:drat-edit}
{Drátený model klasického editačného formulára (vľavo) a steppera (vpravo) (vlastné spracovanie)}{images/wire/edit.png}{width=32cc}


\kap{Testovacie scenáre}

% ...
\setupTABLE[frame=on]
\setupTABLE[row][first][background=color, backgroundcolor=lightgray, style=bold]
\setupTABLE[column][1][width=3cc]
\setupTABLE[column][2][width=15cc]
\setupTABLE[column][3][width=14cc]
\setupTABLE[r][each][align={middle,lohi}]
\Tabulka{tab:test1}{Testovací scenár pre UC Prihlásiť sa do systému (vlastné spracovanie)}{
\bTABLE
  \bTR
    \bTH Krok \eTH
    \bTH Akcia \eTH
    \bTH Očakávaný výsledok \eTH
  \eTR
  \bTR
    \bTD 1 \eTD
    \bTD Otvorte prihlasovaciu stránku. \eTD
    \bTD Zobrazí sa prihlasovacia stránka s políčkami login a heslo. \eTD
  \eTR
   \bTR
    \bTD 2a \eTD
    \bTD Zadajte správne meno a heslo. \eTD
    \bTD Systém užívateľa prihlási a zobrazí sa dashboard. \eTD
  \eTR  
   \bTR
    \bTD 2b \eTD
    \bTD Zadajte nesprávne meno alebo heslo. \eTD
    \bTD Systém zobrazí chybovú hlášku a užívateľa neprihlási. \eTD
  \eTR  
\eTABLE
}


\page
\setupTABLE[frame=on]
\setupTABLE[row][first][background=color, backgroundcolor=lightgray, style=bold]
\setupTABLE[column][1][width=3cc]
\setupTABLE[column][2][width=15cc]
\setupTABLE[column][3][width=14cc]
\setupTABLE[r][each][align={middle,lohi}]
\Tabulka{tab:test2}{Testovací scenár pre UC Vytvoriť žiadosť o registráciu povereného zamestnanca (vlastné spracovanie)}{
\bTABLE
  \bTR
    \bTH Krok \eTH
    \bTH Akcia \eTH
    \bTH Očakávaný výsledok \eTH
  \eTR
  \bTR
    \bTD 1 \eTD
    \bTD Prihláste sa do systému. \eTD
    \bTD Zobrazí sa úvodný rozcestník -- dashboard. \eTD
  \eTR
  \bTR
    \bTD 2 \eTD
    \bTD Kliknite na dlaždicu SPRÁVA OSOB. \eTD
    \bTD Zobrazí sa rozcestník pre modul "SPRÁVA OSOB". \eTD
  \eTR
   \bTR
    \bTD 3 \eTD
    \bTD Kliknite na "Fyzické osoby". \eTD
    \bTD Zobrazí sa zoznam FO. \eTD
  \eTR  
   \bTR
    \bTD 4a_1 \eTD
    \bTD Vyberte neregistrovanú osobu, ktorá má vyplnený primárny email. \eTD
    \bTD Zobrazí sa detail osoby s tlačidlami v hornej časti. \eTD
  \eTR  
 \bTR
    \bTD 4a_2 \eTD
    \bTD Kliknite na 3 bodky v pravom hornom rohu detailu a na tlačidlo "Žádost o vytvoření registrace". \eTD
    \bTD Spustí sa registrácia a registrovanej osobe príde email s aktivačným kľúčom a odkazom na dokončenie registrácie. \eTD
  \eTR  
   \bTR
    \bTD 4b_1 \eTD
    \bTD Vyberte neregistrovanú osobu, ktorá nemá vyplnený primárny email.  \eTD
    \bTD Zobrazí sa detail osoby s tlačidlami v hornej časti. \eTD
\eTR  
   \bTR
    \bTD 4b_2 \eTD
    \bTD Kliknite na 3 bodky v pravom hornom rohu detailu a na tlačidlo "Žádost o vytvoření registrace".  \eTD
    \bTD Zobrazí sa chybová hláška "U vybrané osoby není uveden email jako primární spojení". \eTD
\eTR
   \bTR
    \bTD 4c_1 \eTD
    \bTD Vyberte už zaregistrovanú osobu a kliknite na ňu. \eTD
    \bTD Zobrazí sa detail osoby s tlačidlami v hornej časti. \eTD
  \eTR  
   \bTR
    \bTD 4c_2 \eTD
    \bTD Kliknite na 3 bodky v pravom hornom rohu detailu a na tlačidlo "Žádost o vytvoření registrace". \eTD
    \bTD Zobrazí sa chybová hláška "U vybrané osoby již byla provedena aktivace účtu" \eTD
  \eTR
\eTABLE
}


\page
\setupTABLE[frame=on]
\setupTABLE[row][first][background=color, backgroundcolor=lightgray, style=bold]
\setupTABLE[column][1][width=3cc]
\setupTABLE[column][2][width=15cc]
\setupTABLE[column][3][width=14cc]
\setupTABLE[r][each][align={middle,lohi}]
\Tabulka{tab:test3}{Testovací scenár pre UC Registrovať sa do systému (vlastné spracovanie)}{
\bTABLE
  \bTR
    \bTH Krok \eTH
    \bTH Akcia \eTH
    \bTH Očakávaný výsledok \eTH
  \eTR
  \bTR
    \bTD 1 \eTD
    \bTD Kliknite na odkaz v registračnom emaili. \eTD
    \bTD Zobrazí sa registračná stránka. \eTD
  \eTR
  \bTR
    \bTD 2 \eTD
    \bTD Vložte správne meno a priezvisko. \eTD
    \bTD Umožní sa prechod na ďalší krok. \eTD
  \eTR
   \bTR
    \bTD 3a \eTD
    \bTD Vložte správny aktivačný kľúč z emailu. \eTD
    \bTD Overenie prebehne v poriadku a umožní sa prechod na ďalší krok. \eTD
  \eTR  
   \bTR
    \bTD 3b \eTD
    \bTD Vložte nesprávny aktivačný kľúč z emailu. \eTD
    \bTD Zobrazí sa chybová hláška "Zadaným údajům neodpovídá žádná aktivace. Prosím o kontrolu kombinace jméno, příjmení a aktivačního klíče". \eTD
  \eTR  
 \bTR
    \bTD 4a \eTD
    \bTD Zadajte jedinečný login, dostatočne silné heslo a heslo zopakujte dva krát. \eTD
    \bTD Umožní sa prechod na ďalší krok. \eTD
  \eTR  
   \bTR
    \bTD 4b \eTD
    \bTD Zadajte login, ktorý už v systéme existuje.  \eTD
    \bTD Zobrazí sa chybová hláška "Uživatelské jméno již existuje, zvolte si jiné". \eTD
\eTR  
 \bTR
    \bTD 4c \eTD
    \bTD Zadajte nedostatočné heslo.  \eTD
    \bTD Zobrazí sa chybová hláška "Heslo musí mít minimalně 8 znaků a musí obsahovat velká, malá písmena a číselný znak.  Diakritika a speciální znaky jsou zakázané". \eTD
\eTR  
   \bTR
    \bTD 4d \eTD
    \bTD Zadajte dva krát iné heslo.  \eTD
    \bTD Zobrazí sa chybová hláška "Hesla musí být totožná". \eTD
\eTR
   \bTR
    \bTD 5a \eTD
    \bTD Zaškrtnite súhlas s GDPR. \eTD
    \bTD Umožní sa dokončenie registrácie. \eTD
  \eTR  
   \bTR
    \bTD 5b \eTD
    \bTD Nezaškrtnite súhlas s GDPR. \eTD
    \bTD Dokončenie registrácie nie je možné. \eTD
  \eTR
   \bTR
    \bTD 6 \eTD
    \bTD Kliknite na tlačidlo "dokončit". \eTD
    \bTD Zobrazí sa hláška o úspešnej registrácii a osobe príde email s odkazom na prihlásenie. \eTD
  \eTR
\eTABLE
}


\page
\setupTABLE[frame=on]
\setupTABLE[row][first][background=color, backgroundcolor=lightgray, style=bold]
\setupTABLE[column][1][width=3cc]
\setupTABLE[column][2][width=15cc]
\setupTABLE[column][3][width=14cc]
\setupTABLE[r][each][align={middle,lohi}]
\Tabulka{tab:test4}{Testovací scenár pre UC Spravovať číselníky (vlastné spracovanie)}{
\bTABLE
  \bTR
    \bTH Krok \eTH
    \bTH Akcia \eTH
    \bTH Očakávaný výsledok \eTH
  \eTR
   \bTR
    \bTD 1 \eTD
    \bTD Prihláste sa do systému. \eTD
    \bTD Zobrazí sa úvodný rozcestník -- dashboard. \eTD
  \eTR
  \bTR
    \bTD 2 \eTD
    \bTD Kliknite na dlaždicu SPRÁVA ČÍSELNÍKŮ. \eTD
    \bTD Zobrazí sa rozcestník pre modul "SPRÁVA ČÍSELNÍKŮ". \eTD
  \eTR
   \bTR
    \bTD 3 \eTD
    \bTD Kliknite na oblasť, ktorú chcete spravovať. \eTD
    \bTD Zobrazí sa zoznam konfigurovateľných číselníkov. \eTD
  \eTR  
   \bTR
    \bTD 4 \eTD
    \bTD Kliknite na číselník, ktorý chcete editovať. \eTD
    \bTD Zobrazí sa zoznam hodnôt v číselníku. \eTD
  \eTR  
   \bTR
    \bTD 5a_1 \eTD
    \bTD Vytvorte nový záznam pomocou "+". \eTD
    \bTD Zobrazí sa editačný formulár na vytvorenie nového záznamu číselníka. \eTD
  \eTR   
 \bTR
    \bTD 5a_2a \eTD
    \bTD Vyplňte všetky povinné údaje. \eTD
    \bTD Umožní sa vytvorenie záznamu. \eTD
  \eTR  
 \bTR
    \bTD 5a_2b \eTD
    \bTD Nevyplňte všetky povinné údaje. \eTD
    \bTD Neumožní sa vytvorenie záznamu. \eTD
  \eTR  
   \bTR
    \bTD 5b_1 \eTD
    \bTD Editujte existujúcu hodnotu pomocou editačnej ceruzky pri každom stĺpci.  \eTD
    \bTD Zobrazí sa editačné políčko na editáciu konkrétnej hodnoty. \eTD
\eTR  
 \bTR
    \bTD 5b_2 \eTD
    \bTD Uložte hodnotu pomocou fajky.  \eTD
    \bTD Záznam v číselníku sa upraví. \eTD
\eTR    
   \bTR
    \bTD 6 \eTD
    \bTD Nájdite výskyt číselníka v aplikácii v nejakom editačnom formulári.  \eTD
    \bTD Ponúknuté sú len všetky existujúce a validné záznamy. \eTD
\eTR
\eTABLE
}


%\setupTABLE[frame=on]
%\setupTABLE[row][first][background=color, backgroundcolor=lightgray, style=bold]
%\setupTABLE[column][1][width=3cc]
%\setupTABLE[column][2][width=15cc]
%\setupTABLE[column][3][width=14cc]
%\setupTABLE[r][each][align={middle,lohi}]
%\Tabulka{tab:test5}{Testovací scenár pre UC Zadať nového zamestnanca do systému (vlastné spracovanie)}{
%\bTABLE
%  \bTR
%    \bTH Krok \eTH
%    \bTH Akcia \eTH
%    \bTH Očakávaný výsledok \eTH
%  \eTR
%   \bTR
%    \bTD 1 \eTD
%    \bTD Prihláste sa do systému. \eTD
%    \bTD Zobrazí sa úvodný rozcestník -- dashboard. \eTD
%  \eTR
%  \bTR
%    \bTD 2 \eTD
%    \bTD Kliknite na dlaždicu ZAMĚSTNANCI. \eTD
%    \bTD Zobrazí sa rozcestník pre modul "ZAMĚSTNANCI". \eTD
%  \eTR
%   \bTR
%    \bTD 3 \eTD
%    \bTD Kliknite na "Noví zaměstnanci". \eTD
%    \bTD Zobrazí sa zoznam zamestnancov v stave "nový". \eTD
%  \eTR  
%   \bTR
%    \bTD 4 \eTD
%    \bTD Vytvorte nového zamestnanca kliknutím na plus. \eTD
%    \bTD Zobrazí sa stepper na vytvorenie zamestnanca. \eTD
%  \eTR  
%   \bTR
%    \bTD 5a \eTD
%    \bTD Vyplňte správne a kompletne údaje prvého kroku. \eTD
%    \bTD Nezobrazuje sa žiadna chyba možno prejsť na druhý krok. \eTD
%  \eTR   
%   \bTR
%    \bTD 5b \eTD
%    \bTD Neyplňte nejaký povinný údaj. \eTD
%    \bTD Zobrazí sa hláška "Povinné pole" a neumožní sa prechod na druhý krok. \eTD
%  \eTR
%   \bTR
%    \bTD 5c \eTD
%    \bTD Vyplňte nesprávne rodné číslo - príliš krátke/dlhé alebo s použitím písmen. \eTD
%    \bTD Zobrazí sa hláška "Je nutno zadat platné rodné číslo". \eTD
%  \eTR
%\bTR
%    \bTD 5d \eTD
%    \bTD Vyplňte nesprávne rodné číslo - nedeliteľné 11. \eTD
%    \bTD Zobrazí sa hláška "Rodné číslo musí být dělitelné 11.". \eTD
%  \eTR
%\bTR
%    \bTD 5e \eTD
%    \bTD Vyplňte nesprávne rodné číslo - nesprávny dátum v rodnom čísle. \eTD
%    \bTD Zobrazí sa hláška "Datum v rodném čísle je nesprávný". \eTD
%  \eTR
% \bTR
%    \bTD 6a \eTD
%    \bTD Vyplňte email v správnom formáte. \eTD
%    \bTD Nezobrazuje sa žiadna chyba možno prejsť na tretí krok. \eTD
%  \eTR  
%\bTR
%    \bTD 6b \eTD
%    \bTD Vyplňte email v nesprávnom formáte. \eTD
%    \bTD Zobrazí sa hláška "Zadejte email ve správném formátu (např. jsmith@example.com)". \eTD
%  \eTR 
% \bTR
%    \bTD 7a \eTD
%    \bTD Vyplňte telefónne číslo v správnom formáte. \eTD
%    \bTD Nezobrazuje sa žiadna chyba možno prejsť na tretí krok. \eTD
%  \eTR  
%\bTR
%    \bTD 7b \eTD
%    \bTD Vyplňte  telefónne číslo v nesprávnom formáte. \eTD
%    \bTD Zobrazí sa hláška "Neplatné telefónne číslo. Zadajte číslo vo formáte +420XXXXXXXXX alebo +421XXXXXXXXX (nebo bez předvolby)". \eTD
%  \eTR 
%\eTABLE
%}
%
%
%\setupTABLE[frame=on]
%\setupTABLE[row][first][background=color, backgroundcolor=lightgray, style=bold]
%\setupTABLE[column][1][width=3cc]
%\setupTABLE[column][2][width=15cc]
%\setupTABLE[column][3][width=14cc]
%\setupTABLE[r][each][align={middle,lohi}]
%\Tabulka{tab:test5}{Testovací scenár pre UC Zadať nového zamestnanca do systému -- pokračovanie (vlastné spracovanie)}{
%\bTABLE
%  \bTR
%    \bTH Krok \eTH
%    \bTH Akcia \eTH
%    \bTH Očakávaný výsledok \eTH
%  \eTR
% \bTR
%    \bTD 8a \eTD
%    \bTD Vyplňte dátovú schránku v správnom formáte. \eTD
%    \bTD Nezobrazuje sa žiadna chyba možno prejsť na tretí krok. \eTD
%  \eTR  
%\bTR
%    \bTD 8b \eTD
%    \bTD Vyplňte dátovú schránku v nesprávnom formáte. \eTD
%    \bTD Zobrazí sa hláška "Zadejte platné ID datové schránky.". \eTD
%  \eTR  
%\bTR
%    \bTD 9a \eTD
%    \bTD Vyplňte všetky povinné údaje adresy.  \eTD
%    \bTD Nezobrazuje sa žiadna chyba možno prejsť na štvrtý krok. \eTD
%\eTR    
%   \bTR
%    \bTD 9b \eTD
%    \bTD Neyplňte nejaký povinný údaj. \eTD
%    \bTD Zobrazí sa hláška "Povinné pole" a neumožní sa prechod na štvrtý krok. \eTD
%  \eTR
%   \bTR
%    \bTD 10 \eTD
%    \bTD Vyplňte správne a kompletne údaje o zamestnaní  \eTD
%    \bTD Nezobrazuje sa žiadna chyba možno prejsť na piaty krok. \eTD
%\eTR
%\bTR
%    \bTD 11 \eTD
%    \bTD Skontrolujte úplnosť a správnosť predvyplnenia údajov piateho kroku.  \eTD
%    \bTD Zobrazia sa údaje žiadosti a sú vyplnené a needitovateľné. \eTD
%\eTR
%\bTR
%    \bTD 12 \eTD
%    \bTD Kliknite na tlačidlo "vytvoriť".  \eTD
%    \bTD Vytvorí sa zamestnanec, jeho spojenie, adresa a žiadosť. \eTD
%\eTR
%\eTABLE
%}


\page
\setupTABLE[frame=on]
\setupTABLE[row][first][background=color, backgroundcolor=lightgray, style=bold]
\setupTABLE[column][1][width=3cc]
\setupTABLE[column][2][width=15cc]
\setupTABLE[column][3][width=14cc]
\setupTABLE[r][each][align={middle,lohi}]
\setupfloatsplitting [conversion=continued]
\setupfloat[table][default={top,split}]
\Tabulka{tab:test5}{Testovací scenár pre UC Zadať nového zamestnanca do systému (vlastné spracovanie)}{
\bTABLE
  \bTR
    \bTH Krok \eTH
    \bTH Akcia \eTH
    \bTH Očakávaný výsledok \eTH
  \eTR
   \bTR
    \bTD 1 \eTD
    \bTD Prihláste sa do systému. \eTD
    \bTD Zobrazí sa úvodný rozcestník -- dashboard. \eTD
  \eTR
  \bTR
    \bTD 2 \eTD
    \bTD Kliknite na dlaždicu ZAMĚSTNANCI. \eTD
    \bTD Zobrazí sa rozcestník pre modul "ZAMĚSTNANCI". \eTD
  \eTR
   \bTR
    \bTD 3 \eTD
    \bTD Kliknite na "Noví zaměstnanci". \eTD
    \bTD Zobrazí sa zoznam zamestnancov v stave "nový". \eTD
  \eTR  
   \bTR
    \bTD 4 \eTD
    \bTD Vytvorte nového zamestnanca kliknutím na plus. \eTD
    \bTD Zobrazí sa stepper na vytvorenie zamestnanca. \eTD
  \eTR  
   \bTR
    \bTD 5a \eTD
    \bTD Vyplňte správne a kompletne údaje prvého kroku. \eTD
    \bTD Nezobrazuje sa žiadna chyba možno prejsť na druhý krok. \eTD
  \eTR   
   \bTR
    \bTD 5b \eTD
    \bTD Neyplňte nejaký povinný údaj. \eTD
    \bTD Zobrazí sa hláška "Povinné pole" a neumožní sa prechod na druhý krok. \eTD
  \eTR
   \bTR
    \bTD 5c \eTD
    \bTD Vyplňte nesprávne rodné číslo -- príliš krátke/dlhé alebo s použitím písmen. \eTD
    \bTD Zobrazí sa hláška "Je nutno zadat platné rodné číslo". \eTD
  \eTR
\bTR
    \bTD 5d \eTD
    \bTD Vyplňte nesprávne rodné číslo -- nedeliteľné 11. \eTD
    \bTD Zobrazí sa hláška "Rodné číslo musí být dělitelné 11.". \eTD
  \eTR
\bTR
    \bTD 5e \eTD
    \bTD Vyplňte nesprávne rodné číslo -- nesprávny dátum v rodnom čísle. \eTD
    \bTD Zobrazí sa hláška "Datum v rodném čísle je nesprávný". \eTD
  \eTR
 \bTR
    \bTD 6a \eTD
    \bTD Vyplňte email v správnom formáte. \eTD
    \bTD Nezobrazuje sa žiadna chyba možno prejsť na tretí krok. \eTD
  \eTR  
\bTR
    \bTD 6b \eTD
    \bTD Vyplňte email v nesprávnom formáte. \eTD
    \bTD Zobrazí sa hláška "Zadejte email ve správném formátu (např. jsmith@example.com)". \eTD
  \eTR 
 \bTR
    \bTD 7a \eTD
    \bTD Vyplňte telefónne číslo v správnom formáte. \eTD
    \bTD Nezobrazuje sa žiadna chyba možno prejsť na tretí krok. \eTD
  \eTR  
\bTR
    \bTD 7b \eTD
    \bTD Vyplňte  telefónne číslo v nesprávnom formáte. \eTD
    \bTD Zobrazí sa hláška "Neplatné telefónne číslo. Zadajte číslo vo formáte +420XXXXXXXXX alebo +421XXXXXXXXX (nebo bez předvolby)". \eTD
  \eTR 
 \bTR
    \bTD 8a \eTD
    \bTD Vyplňte dátovú schránku v správnom formáte. \eTD
    \bTD Nezobrazuje sa žiadna chyba možno prejsť na tretí krok. \eTD
  \eTR  
\bTR
    \bTD 8b \eTD
    \bTD Vyplňte dátovú schránku v nesprávnom formáte. \eTD
    \bTD Zobrazí sa hláška "Zadejte platné ID datové schránky.". \eTD
  \eTR  
\bTR
    \bTD 9a \eTD
    \bTD Vyplňte všetky povinné údaje adresy.  \eTD
    \bTD Nezobrazuje sa žiadna chyba možno prejsť na štvrtý krok. \eTD
\eTR    
   \bTR
    \bTD 9b \eTD
    \bTD Neyplňte nejaký povinný údaj. \eTD
    \bTD Zobrazí sa hláška "Povinné pole" a neumožní sa prechod na štvrtý krok. \eTD
  \eTR
   \bTR
    \bTD 10 \eTD
    \bTD Vyplňte správne a kompletne údaje o zamestnaní  \eTD
    \bTD Nezobrazuje sa žiadna chyba možno prejsť na piaty krok. \eTD
\eTR
\bTR
    \bTD 11 \eTD
    \bTD Skontrolujte úplnosť a správnosť predvyplnenia údajov piateho kroku.  \eTD
    \bTD Zobrazia sa údaje žiadosti a sú vyplnené a needitovateľné. \eTD
\eTR
\bTR
    \bTD 12 \eTD
    \bTD Kliknite na tlačidlo "vytvoriť".  \eTD
    \bTD Vytvorí sa zamestnanec, jeho spojenie, adresa a žiadosť. \eTD
\eTR
\eTABLE
}


\page
\setupTABLE[frame=on]
\setupTABLE[row][first][background=color, backgroundcolor=lightgray, style=bold]
\setupTABLE[column][1][width=3cc]
\setupTABLE[column][2][width=15cc]
\setupTABLE[column][3][width=14cc]
\setupTABLE[r][each][align={middle,lohi}]
\Tabulka{tab:test6}{Testovací scenár pre UC Registrovať nástup zamestnanca (vlastné spracovanie)}{
\bTABLE
  \bTR
    \bTH Krok \eTH
    \bTH Akcia \eTH
    \bTH Očakávaný výsledok \eTH
  \eTR
   \bTR
    \bTD 1 \eTD
    \bTD Po úspešnom založení zamestnanca kliknite na vytvorenú žiadosť na jeho detaile v sekcii Žiadosti. \eTD
    \bTD Zobrazí sa detail žiadosti s tlačidlom "Ověř stav žádosti". \eTD
  \eTR
\bTR
    \bTD 2a \eTD
    \bTD Kliknite na tlačidlo "Ověř stav žádosti" pre správne vytvoreného zamestnanca. \eTD
    \bTD Po chvíľke nastane zmena stavu žiadosti na OVĚŘENO vzhľadom k ZP aj ČSSZ. \eTD
  \eTR  
\bTR
    \bTD 2a \eTD
    \bTD Kliknite na tlačidlo "Ověř stav žádosti" pre zamestnanca s nesprávnymi údajmi. \eTD
    \bTD Po chvíľke nastane zmena stavu žiadosti na CHYBA vzhľadom k ZP a/alebo ČSSZ. \eTD
  \eTR    
\eTABLE
}


\stopappendices

\stopthesis

\endinput		

%%%% TODO %%%%%%%%%%%%%%%%%%%%%%%%%%%%%%
Tady si můžeš psát poznámky, které se neobjeví ve výstupu.
