%
\usemodule[ctx-thesis-v0.991]
\usemodule[bib.sty-v2.78]

\setupthesis[sk,mendelu,pef,none][ % jazyk,univerzita,fakulta,ústav/katedra/pracoviště ; language,university,faculty,department
  type={dp},                 % bp,dp,pp,zp,sp,pr,pt aj./etc.
  authorname={Zuzana},	     % jméno
  authorsurname={Lysová},        % příjmení
  authordegree={Bc.},	     % titul před jménem
  authorgender={F},          % pohlaví (holky mají F)
  supervisor={RNDr. Zuzana Špendel, Ph.D.},        % vedoucí práce
  title={Analýza, návrh a implementácia softwarovej platformy pre firmu Asseco Central Europe, a.s.},                           % název práce
  titleen={Analysis, design and implementation of software platform for Asseco Central Europe, a.s.}, 	           % název práce anglicky
  keywords={bla,blabla,bla,blabla},    %
  keywordsen={aa,xxx,vvv,aa,bbb}, %
%  acknowledgement={Děkuji své babičce, že mi napekla na cestu buchty.},	           % poděkování
  abstract={},		                           % český abstrakt
  abstracten={},		                   % anglický abstrakt
  location={Brno,Brne},	   % místo vydání (za čárkou 6. pád) ; location (second parameter is not necessary for English)
%  year={2021},		   % rok odevzdání práce (automaticky aktuální rok) ; year, the default is the current year
%  thesisassignmentform={img/file001.png,img/file002.png},  % seznam souborů se skenem zadání práce; file is thesis assignment
]

\startthesis
\startbodymatter

\kap{Úvod}
V súčasnosti je množstvo zamestnancov zahltených rutinnými administratívnymi úkonmi. Firma Asseco Central Europe, a.s. prišla s nápadom vytvoriť integračnú platformu s pracovným názvom EMMA, ktorá by mala tieto úkony minimalizovať. Projekt je v štádiu riešenia a aktuálne existuje zjednodušená implementácia služby "nástup zamestnanca do zamestnania", ktorá je v katalógu služieb. Tento katalóg služieb verejnej správy ČR obsahuje v dnešnej dobe takmer 8 tisíc služieb a vyše 34 tisíc úkonov.  Tieto služby a úkony sa týkajú bezmála 400 agend a ohlasuje ich približne 30 rôznych ohlasovateľov (ministerstvá, úrady a pod.). Väčšinu týchto úkonov je možné previesť online, prostredníctvom dátovej schránky. 

Projekt EMMA je riešením, ktoré umožni integráciu služieb VS do informačných systémov firiem a spojenie viacerých služieb verejnej správy. Už spomínaný príklad nástupu zamestnanca do zamestnania umožňuje pomocou jedného formuláru (poprípade tlačítka) nahlásiť túto udalosť príslušným úradom - Českej správe sociálneho zabezpečenia a zdravotnej poisťovni. Prvotným zámerom je integrácia služieb VS, no do budúcna sa plánuje rozšíriť to aj o služby komerčnej sféry, napr. poistenie auta. 

\TODO
TODO niekde ku koncu - VÝHODY A NEVÝHODY EMMA - napr. zamestnanci...

\kap{Cieľ}
Cieľom tejto diplomovej práce je analýza súčasných procesov verejnej správy a úrovne digitalizácie v Českej republike. Práca sa zameriava na rozbor existujúceho stavu projektu EMMA, ktorý zahŕňa služby verejnej správy, a identifikovať potenciálne oblasti na jeho rozšírenie. Kľúčovým prvkom je analyzovať a~následne rozšíriť stávajúce rozhranie projektu, vyvinuté na platforme SAMO, o~nový modul zamestnancov. 

\kap{Súčasný stav}
V dnešnej dobe sa digitalizácia verejnej správy stala častou témou rôznych diskusií. Je to najmä preto, že predstavuje cestu k efektívnejšiemu, účinnejšiemu a transparentnejšiemu poskytovaniu služieb naprieč rozličnými odvetviami. \zlom V~tomto dynamickom kontexte sľubuje digitalizácia transformáciu tradičných modelov služieb na modely, ktoré sú viac prispôsobené súčasným potrebám občanov a~inštitúcií. \scr(Andersson, 2022)

Existuje viacero spôsobov hodnotenia a merania úrovne rozvinutosti krajín v oblasti digitalizácie a rozvoja eGovernmentu. Patrí medzi ne napríklad \start \it Index digitálnej ekonomiky a spoločnosti, Index rozvoja eGovernmentu, eGovernment Benchmark \stop a iné. Tieto nástroje umožňujú detailnejšie pochopenie postupov, ktoré krajiny implementujú na podporu eGovernmentu a identifikáciu oblastí, v~ktorých je potrebné zlepšenie. V nasledujúcich kapitolách sú v~skratke popísané spomínané 3 prieskumy a ich posledné výsledky. Analýza výsledkov umožní lepšie porozumieť súčasný stav na národnej aj medzinárodnej úrovni.

\pkap{Index digitálnej ekonomiky a spoločnosti}
Európska komisia sleduje a monitoruje pokrok členských štátov v digitálnej oblasti od roku 2014 a každý rok zverejňuje informácie o indexe digitálnej ekonomiky a spoločnosti (Digital Economy and Society Index, DESI). Tento index zoraďuje štáty podľa úrovne digitalizácie a zároveň posudzuje ich relatívny pokrok za uplynulých päť rokov vzhľadom na ich počiatočnú situáciu.

Oblasti, ktoré skúma DESI sú:

\startitemize
\item{\start\bf ľudský kapitál\stop -- internetové znalosti používateľov, pokročilé znalosti ľudí v IT oblasti}
\item{\start\bf konektivita\stop -- využitie a pokrytie pevného a mobilného pripojenia a~ich ceny}
\item{\start\bf integrácia digitálnych technológií \stop -- digitálne technológie pre firmy (cloud, umelá inteligencia...), e-commerce \footnote{e-commerce -- obchodné činnosti prevádzané na internete a pomocou ďalších elektronických prostriedkov}}
\item{\start\bf digitálne verejné služby \stop -- e-government, otvorené dáta \footnote{otvorené data (open data, vládne dáta) -- informácie verejného sektoru, ktoré sú bezplatne dostupné na akékoľvek účely}}
\stopitemize 

Európska komisia spolu s Radou prejednávajú rozhodnutie o politickom programe "Cesta k digitálnej dekáde", ktorý stanovuje ciele na úrovni Európskej únie, dosiahnuteľné do roku 2030. Cieľom je zaistiť to, aby bola digitálna transformácia komplexná a udržateľná a aby prebehla vo všetkých odvetviach hospodárstva. Dosiahnutie cieľa programu závisí na všetkých členských krajinách a na ich spoločnom úsilí \scr(Európska komisia - Metodika, 2022).

Česká republika je podľa výsledkov DESI za rok 2022 na 19. mieste (z 27 členských štátov) (viď \in{obrázok}[obr:DESI]). V porovnaní s rokom 2021 sa Česká republika zlepšila v oblasti digitálnych verejných služieb a konektivite. Zhoršila sa v integrácii digitálnych technológií. Možným pozitívom a príležitosťou je, že za digitalizáciu verejnej správy v Českej republike je od roku 2007 prvýkrát zodpovedná konkrétna osoba - miestopredseda Ivan Bartoš. ČR pokračuje v implementácii stratégie "Digitálne Česko" z roku 2018 (aktualizovanej v roku 2020) \scr(Európska komisia - Česko, 2022).

\obrazekB{obr:DESI}
{Index digitálnej ekonomiky a spoločnosti 2022 (Európska komisia - Česko, 2022)}{images/DESI.png}{width=26cc}

\obrazek{obr:DESI}
{Index DESI 2022 - relatívne výsledky v jednotlivých oblastiach (Európska komisia - Česko, 2022)}{images/DESI2.png}{width=26cc}

Graf na \in{obrázku}[obr:DESI] ukazuje porovnanie jednotlivých oblastí indexu DESI s priemerom 27 členských štátov EÚ. Taktiež tam možno vidieť aj výsledok krajín s najvyšším skóre (Fínsko a v tesnom závese Dánsko) \scr(Európska komisia - Česko, 2022).

\pkap{Index rozvoja eGovernmentu}
Ďalším významným prieskumom je hodnotenie eGovernmentu vykonávané Organizáciou Spojených Národov, ktoré poskytuje hodnotenie eGovernmentu  naprieč všetkými 193 členskými štátmi. Tento prieskum hodnotí krajiny na základe Indexu rozvoja e-governmentu (E-Government Development Index, EGDI), ktorý je kombináciou primárnych dát (zbieraných a vlastnených OSN) a sekundárnych dát (získaných od iných agentúr) \scr(United Nations, 2024).

EGDI sa získava váženým priemerom troch indexov, ktoré sa týkajú týchto oblastí:

\startitemize
\item{\start\bf online služby \stop \footnote{Online Services Index (OSI)} -- hodnotenie verejných portálov na základe 5 kritérií (inštitucionálny rámec, poskytovanie služieb, poskytovanie obsahu, technológie a digitálna účasť občanov)}
\item{\start\bf telekomunikačná infraštruktúra  \stop \footnote{Telecommunications Infrastructure Index (TII)} -- hodnotí úroveň rozvoja infraštruktúry nevyhnutnej pre e-vládu, vrátane pripojenia na internet, infraštruktúry širokopásmového prístupu a mobilných sietí}
\item{\start\bf ľudský kapitál \stop \footnote{Human Capital Index (HCI)} -- hodnotí vzdelanie a úroveň zručností obyvateľstva krajiny, s dôrazom na faktory ako miera gramotnosti, zapojenie do vzdelávania a dostupnosť kvalifikovaných odborníkov v oblasti informačných a komunikačných technológií}
\stopitemize 

Na základe hodnôt indexov EGDI je možné členské štáty OSN rozčleniť do štyroch kategórií: krajiny s veľmi vysokým indexom, krajiny s vysokým indexom, krajiny so stredným indexom a krajiny s nízkym indexom. Podľa najnovšieho prieskumu z roku 2022 spadá do veľmi vysokého indexu 60 krajín (31~\%), do vysokého 73 (38~\%), do stredného 53 (27,5~\%) a 7 krajín (3,5~\%) má nízky index rozvoja eGovernmentu.

\obrazekB{obr:EGDI-mapa}
{Geografické rozloženie štyroch EGDI kategórií (United Nations, 2022)}{images/EGDI-mapa.png}{width=30cc}

Medzi najvyspelejšie krajiny v oblasti elektronického vládnutia podľa Indexu rozvoja e-governmentu (EGDI) sa, podobne ako v prípade DESI, radia Dánsko a Fínsko. Česká republika sa umiestňuje na 45. pozícii, avšak stále patrí do kategórie krajín s veľmi vysokým indexom EGDI. Porovnanie jednotlivých hodnôt je v \in{tabuľke}[EGDI]. Na \in{obrázku}[obr:EGDI-mapa] je zobrazené geografické rozloženie jednotlivých krajín a ich úrovní EGDI.

\setupTABLE[frame=on]
\setupTABLE[row][first][background=color, backgroundcolor=gray, style=bold]
\setupTABLE[column][1][width=10cc]
\setupTABLE[column][2][width=6cc]
\setupTABLE[column][3][width=4cc]
\setupTABLE[column][4][width=4cc]
\setupTABLE[column][5][width=4cc]
\setupTABLE[column][6][width=4cc]
\setupTABLE[r][each][align={middle,lohi}]

\Tabulka{EGDI}{Porovnanie EGDI vybraných krajín s ČR (United Nations, 2022)}{
\bTABLE
  \bTR
    \bTH Krajina \eTH
    \bTH EGDI poradie\eTH
    \bTH OSI \eTH
    \bTH HCI \eTH
    \bTH TII \eTH
    \bTH EGDI \eTH
  \eTR
  \bTR
    \bTD Dánsko \eTD
    \bTD 1 \eTD
    \bTD 0.9797 \eTD
    \bTD 0.9559 \eTD
    \bTD 0.9725 \eTD
    \bTD 0.9753 \eTD
  \eTR
  \bTR
    \bTD Fínsko \eTD
    \bTD 2 \eTD
    \bTD 0.9833 \eTD
    \bTD 0.9640 \eTD
    \bTD 0.9172 \eTD
    \bTD 0.9533 \eTD
  \eTR
  \bTR
    \bTD ... \eTD
    \bTD ... \eTD
    \bTD ... \eTD
    \bTD ... \eTD
    \bTD ... \eTD
    \bTD ... \eTD
  \eTR
  \bTR
    \bTD Česká republika \eTD
    \bTD 45 \eTD
    \bTD 0.6693 \eTD
    \bTD 0.9114 \eTD
    \bTD 0.8456 \eTD
    \bTD 0.8221 \eTD
  \eTR
  \bTR
    \bTD Ukrajina \eTD
    \bTD 46 \eTD
    \bTD 0.8148 \eTD
    \bTD 0.8669 \eTD
    \bTD 0.7270 \eTD
    \bTD 0.8029 \eTD
  \eTR
  \bTR
    \bTD Slovenská republika \eTD
    \bTD 47 \eTD
    \bTD 0.7260 \eTD
    \bTD 0.8436 \eTD
    \bTD 0.8328 \eTD
    \bTD 0.8008 \eTD
  \eTR
\eTABLE
}


\pkap{ eGovernment Benchmark}
Posledným zo spomínaných prieskumov je eGovernment Benchmark. eGovernment Benchmark monitoruje pokrok v digitalizácii verejných služieb 35-tich európskych krajín, známych ako EU27+ (27 členských štátov Európskej únie spolu s Islandom, Nórskom, Švajčiarskom, Albánskom, Čiernou horou, Severným Macedónskom, Srbskom a Tureckom) \scr(van der Linden, 2022).

Prieskum eGovernment Benchmark sa zameriava na tieto štyri kľúčové oblasti:

\startitemize
\item{\start\bf orientácia na užívateľa \stop -- miera poskytovania online služieb, mobile-friendly služby, online podpora a spätná väzba}
\item{\start\bf transparentnosť  \stop -- informácie o tom, ako sú poskytované služby VS, spracovaní osobných údajov a pod.}
\item{\start\bf kľúčové faktory \stop -- dostupnosť technologických faktorov v súvislosti so službami VS}
\item{\start\bf cezhraničné služby \stop -- jednoduchosť používania služieb VS pre občanov zo zahraničia a mechanizmy podpory a spätnej väzby pre takýchto občanov}
\stopitemize 

\obrazekB{obr:benchmark-mapa}
{Geografické rozloženie eGovernment vyspelosti podĺa prieskumu eGovernment Benchmark (van der Linden, 2022)}{images/benchmark-mapa.png}{width=30cc}

Na základe týchto štyroch oblastí získavajú krajiny tzv. "skóre eGovernment vyspelosti", ktorého škála sa pohybuje na stupnici od 0 do 100. Vedúcimi krajinami boli podľa posledného prieskumu z roku 2022 Malta a Estónsko. Česká republika dosiahla 22. miesto \scr(van der Linden, 2022). Porovnanie jednotlivých indexov vybraných krajín je v \in{tabuľke}[benchmark]. \in{Obrázok}[obr:benchmark-mapa] ilustruje geografické rozloženie krajín a ich príslušné skóre eGovernment vyspelosti.

\setupTABLE[frame=on]
\setupTABLE[row][first][background=color, backgroundcolor=lightgray, style=bold]
\setupTABLE[column][1][width=12cc]
\setupTABLE[column][2][width=6cc]
\setupTABLE[column][3][width=14cc]
\setupTABLE[r][each][align={middle,lohi}]
\Tabulka{benchmark}{Porovnanie skóre eGovernment Benchmark vybraných krajín s ČR (van der Linden, 2022)}{
\bTABLE
  \bTR
    \bTH Krajina \eTH
    \bTH poradie\eTH
    \bTH eGovernment maturity score \eTH
  \eTR
  \bTR
    \bTD Malta \eTD
    \bTD 1 \eTD
    \bTD 96 \eTD
  \eTR
  \bTR
    \bTD Estónsko \eTD
    \bTD 2 \eTD
    \bTD 90 \eTD
  \eTR
  \bTR
    \bTD ... \eTD
    \bTD ... \eTD
    \bTD ... \eTD
  \eTR
  \bTR
    \bTD Česká republika \eTD
    \bTD 22 \eTD
    \bTD 63 \eTD
  \eTR
  \bTR
    \bTD Bulharsko \eTD
    \bTD 23 \eTD
    \bTD 61 \eTD
  \eTR
  \bTR
    \bTD Bulharsko \eTD
    \bTD 23 \eTD
    \bTD 61 \eTD
  \eTR
  \bTR
    \bTD Taliansko \eTD
    \bTD 24 \eTD
    \bTD 61 \eTD
  \eTR
  \bTR
    \bTD Chorvátsko \eTD
    \bTD 25 \eTD
    \bTD 61 \eTD
  \eTR
  \bTR
    \bTD Slovenská republika \eTD
    \bTD 26 \eTD
    \bTD 60 \eTD
  \eTR
\eTABLE
}


\pkap{Zhrnutie výsledkov prieskumov}

Analýza Indexu digitálnej ekonomiky a spoločnosti (DESI), Indexu rozvoja eGovernmentu (EGDI) a eGovernment Benchmarku ukazuje, že Česká republika dosahuje pokrok v digitalizácii verejnej správy, no v porovnaní s medzinárodným kontextom, najmä so severskými krajinami a lídrami v EÚ, čelí výzvam v~konektivite, integrácii digitálnych technológií a poskytovaní digitálnych verejných služieb. Čo sa ale týka susedných krajín, ČR si vedie dobre. Severské krajiny a vyspelé členské štáty EÚ vynikajú v inováciách a ponúkaní efektívnych, užívateľsky prívetivých digitálnych služieb, ktoré ČR môže použiť ako model pre zlepšovanie svojich digitálnych služieb. Významná je tiež potreba zamerania sa na cezhraničné digitálne služby, kde ČR môže opäť čerpať z príkladov zo zahraničia.

V súčasnej dobe sa ČR nachádza v strednej časti hodnotiacich rebríčkov digitalizácie, avšak iniciatíva "Digitálne Česko" (viac popísaná nižšie) má potenciál posunúť ČR na prednejšie pozície v týchto prieskumoch. Predstavuje sľubný krok smerom k zlepšeniu výkonnosti Českej republiky v digitálnom prostredí, zvýšeniu jej konkurencieschopnosti a zlepšeniu poskytovania digitálnych služieb občanom.


%Posunom doby sa stále viac vecí spojených s verejnou správou dá riešiť online. Zamestnávatelia sa stretávajú na pravidelnej báze s rôznymi procesmi nutnými k chodu podniku a potrebujú mať všetko v súlade so zákonmi a pravidlami verejnej správy. Aktuálne to musia všetko riešiť buď osobne na úradoch alebo v tom lepšom prípade online na rôznych portáloch verejnej správy.
%
%Procesy, ktoré sa dajú v ČR vyriešiť online sú častokrát dostupné pomocou formulárov na portáloch ministerstiev. Príkladmi takýchto portálov sú portál Ministerstva práce a sociálnych vecí (https://www.mpsv.cz/), portál Českej správy sociálneho zabezpečenia (https://www.cssz.cz/), Portál občana (https://portal.gov.cz/) a podobne. Tieto portály poskytujú rôzne formuláre, ktoré umožnia občanom vybaviť rôzne požiadavky online. Ide pri tom o komunikáciu medzi klientom a orgánom verejnej moci (OVM). Cieľom platformy EMMA od firmy Asseco je, aby tieto služby boli zapojené do interných informačných systémov podnikov. To bude viesť k tomu, že zamestnávatelia všetko vybavia na jednom mieste a nebudú musieť vypĺňať množstvo formulárov na množstve rôznych portálov.
%
%Príkladom EMMA služby je nástup zamestnanca do zamestnania. Povinnosťou zamestnávateľa pri nástupe zamestnanca je oznámiť nástup príslušným úradom a poisťovniam. Existuje viac možností, ako to urobiť, no vždy to zahŕňa viacero oddelených činností. Jednou z možností je, že musí navštíviť ePortal ČSSZ (Česká správa sociálního zabezpečení), Portál zdravotních pojišťoven a v prípade zamestnávania cudzincov aj ePortal MPSV (Ministerstva práce a sociálních věcí). Služba EMMA umožní vyplniť údaje len raz v jednom formulári a spojí sa s atomickými službami úradov. 
%
%Táto služba spolu s ďalšími bude uvedená na trh po ukončení realizácie projektu, čo môže výrazne zjednodušiť biznis procesy podnikov. 

%\pkap{Súčasný stav digitálnej verejnej správy v zahraničí}
%
%\TODO TODO: zhrnutie predchádzajúcich kapitol - z pohľadu zahraničia

%V tejto kapitole a jej podkapitolách je zanalyzovaná a popísaná situácia v krajinách, ktoré sú buď blízko Českej republike (polohou, kultúrou, rozvinutosťou...) alebo krajiny, o ktorých je známe, že nejakým spôsobom v oblasti digitálnych služieb vynikajú.
%
%\ppkap{Slovensko}
%Prvou analyzovanou krajinou je Slovensko, ako najbližší sused Českej republiky. Na Slovensku existuje množstvo elektronických služieb verejnej správy, ktoré sú dostupné prostredníctvom Ústredného portálu verejnej správy (ÚPVS). Tento portál umožňuje centrálne vykonávať elektronickú úradnú komunikáciu s rôznymi orgánmi verejnej moci a pristupovať k spoločným modulom. %Jeho dizajn je ale pomerne zastaralý a neprehľadný (viď \in{obrázok}[obr:portal\_sk]).
%
%Na základe analýzy tohto portálu je možné tvrdiť, že Slovenská republika sa zatiaľ špecializuje viac na vzťah občan-štát, ako na vzťah občan-zamestnávateľ-štát. V popredí je oblasť e-financovania, hlavne situácie týkajúce sa daní. Ministerstvo financií Slovenskej republiky v spolupráci s daňovými úradmi zaviedlo štandardizovaný proces pre elektronické prenosy faktúr a to prostredníctvom Informačného systému elektronickej fakturácie (IS EFA), ktorý umožňuje odosielanie štruktúrovaných elektronických fakturačných údajov do slovenskej daňovej správy. Používanie tohto IS bude časom povinné pre inštitúcie štátnej správy, verejnej správy a taktiež pre podnikateľov, ktorí im dodávajú tovary a služby. Tento systém je v princípe podobný systému e-kasa, ktorý prepája všetky slovenské registračné pokladnice s online portálom slovenskej daňovej správy \scr(Kilinger, 2023; MFSR, 2024). Portály eKasa a eFaktúry už podliehajú Jednotným dizajn manuálom elektronických služieb (viac informácií v \in{kapitole}[dizajn-system]).
%
%\ppkap{Poľsko}
%
%
%\ppkap{Dánsko a Estónsko}
%Denmark’s outstanding performance in e-government is the UN’s \#1 ranking in the E-Government Development Index in 2018, 2020, and 2022, as well as the highest number of citizens using e-government services in the entire European Union, with 93\% of Danish internet users using digital public services in 2021. Meanwhile, according to the 2022 Digital Economy and Society Index by European Commission, Estonia is the best performing country in the sector of digital public services, and it is recognized as a leader in e-government, outperforming Central and Eastern European countries.
%The two countries have carefully and precisely defined their digitalization goals and have pursued them. To develop a wide range of efficient and well-integrated digital public services, they have supported digital innovation, enacted laws to promote the adoption of digital services, educated the public, and engaged in public-private partnerships. These developments have resulted in significant infrastructural solutions that benefit all citizens and businesses in their daily lives when interacting with the government. In the same way that Denmark has borger.dk, Estonia has eesti.ee as an integrated public service portal, both of which are available 24/7 and protected against high demand. This robust government service delivers a superior citizen experience by meeting the four core tenets of user needs by Aarron Walters: functional, reliable, usable, and pleasurable (Queue IT, 2023).

%\obrazekB{obr:portal\_sk}
%{Ústredný portál verejnej správy SR (Slovensko.sk, 2024)}{images/portal_sk.png}{width=30cc}

%\QUES Opravdu musím psát literární rešerši? 

\pkap{Architektura eGovernmentu ČR}
Pojem eGovernment bol už v tejto práci viackrát zmieňovaný. Ide o pojem popisujúci modernú digitálnu verejnú správu, ktorá využíva k výkonu svojich právomocí digitálnu infraštruktúru. Táto infraštruktúra realizuje sadu služieb informačných technológií (ICT služieb), ktoré sú zdieľané, dôveryhodné, prepojené, bezpečné, automatizované, efektívne a ľahko používateľné pre užívateľov. Služby eGovernmentu sú určené občanom, firmám, podnikateľom i úradníkom. Synonymami pojmu eGovernment sú "digitálny government" alebo "digitálna verejná správa" \scr(Digitální a informační agentura, 2023).

\ppkap{Poslanie a vízia eGovernmentu v Českej republike}
Digitálna verejná správa používa rôzne poskytnuté a dostupné informácie, ktoré automatizovane spracuváva s cieľom obmedziť, respektíve znížt množstvo podania a objemu informácií zo straný užívateľov služieb VS.

Hlavným poslaním eGovernmentu je: 

\citat{Poskytovať klientom verejnej správy jednoduché a efektívne služby, ktoré im uľahčia dosiahnutie ich práv a nárokov, ako aj plnenie ich povinností a záväzkov vo vzťahu k verejnej správe.} \scr(Digitální a informační agentura, 2023)

Vízia eGovernmentu v ČR do konca horizontu Informačnej koncepcie ČR (viac popísaná v kapitole 3.6 Informační koncepce ČR) je: 

\citat{Česká republika je jednou z popredných krajín v užívateľskej prívetivosti verejnej správy vďaka svojmu klientsky orientovanému prístupu, modernému dizajnu úradných procesov a efektívnemu využívaniu digitálnych a nedigitálnych technológií.} \scr(Digitální a informační agentura, 2023)

\pkap{Informační koncepce ČR}
Informační koncepce ČR (ďalej ako IKČR) rozpracováva vyššie spomenutú víziu do rôznych cieľov, ktoré realizujú jednotlivé orgány VS. Predstavuje komplexný plán na rozvoj informačných systémov verejnej správy, ktorý je prispôsobený potrebám a cieľom štátu. To, či ciele boli naplnené alebo nie ukazuje stav plnenia zadefinovaných cieľov a pozícia v rebríčkoch ako je napríklad DESI (rozobraté v kapitole 3.1 Index digitálnej ekonomiky a spoločnosti). Všetky povinné subjekty podľa zákona  č. 365/2000 Sb., o informačných systémoch, majú povinnosť viesť vlastné informačné koncepcie a vždy ich musia uviesť do súladu s Informačnou koncepciou ČR. Je to prakticky koncepcia rozvoja informačných systémov verejnej správy, ktorú spracováva Ministerstvo vnútra a schvaľuje vláda. Je vypracovaná na základe ustanovenia § 5a, Zákona č. 365/2000 Sb., o informačných systémoch verejnej správy. Týmto prístupom sa zabezpečuje jednotný rámec pre rozvoj a prevádzku informačných systémov a služieb eGovernmentu v celej krajine.

Medzi jej hlavné časti IKČR patria:

\startitemize
\item{\start\bf architektonické principy \stop eGovernmentu a elektronizácie verejnej správy,}
\item{\start \bf efektívny rozvoj \stop digitálnej verejnej správy a informačných systémov verejnej správy (ISVS),}
\item{\start\bf zásady \stop riadenia ICT vo verejnej správe,}
\item{základné koncepčné \start \bf povinnosti \stop pre budovanie, rozvoj a prevádzku ISVS a ich vzájomné prepojenie a pre budovanie spoločných služieb eGovernmentu.}
\stopitemize

IKČR je základný dokument, ktorý určuje dlhodobé ciele a strategické smerovanie ČR v oblasti informačných systémov a digitálnych služieb verejnej správy a všeobecné princípy obstarávania, tvorby, správy a prevádzky ISVS v ČR. Obsahuje predovšetkým:
\startitemize
\item{ciele a podporu oblasti eGovernmentu (zo strany informačných systémov verejnej správy),}
\item{zásady riadenia útvarov informatiky a riadenie životného cyklu ISVS,}
\item{architektonické principy pre návrh a rozvoj ISVS a ich služieb. \scr(Digitální a informační agentura, 2023)}
\stopitemize

\ppkap{Metódy riadenia ICT verejnej správy ČR}
Súčasťou a kľúčovým predpokladom naplnenia cieľov stanovených v IKČR je zavedenie efektívnej centrálnej koordinácie riadenia ICT. Zároveň je to aj podpora transformačných iniciatív, ktoré smerujú k digitalizácii VS a plnému digitálnemu governmentu. \uv{Metódy riadenia ICT verejnej správy ČR} (ďalej ako MRICT) je dokument, ktorý stanovuje pravidlá prevádzkovania ICT kapacít, kompetencií štátnych podnikov, riadenia útvarov informatiky, centrálneho koordinovaného riadenia ICT podpory eGovernmentu a podobne. MRICT nadväzuje na zásady riadenia ICT, ktoré sú súčasťou IKČR, a predstavuje kľúčový nástroj na zabezpečenie súladu a efektívnosti pri procesoch digitalizácie VS \scr(Digitální a informační agentura, 2023).

\pkap{Katalóg služieb verejnej správy}
Katalóg služieb VS je súčasťou registra práv a povinností (RPP) a obsahuje údaje o službách VS, úkonoch a dostupných kanáloch. Katalóg služieb VS sa dá vnímať z dvoch pohľadov:

\startitemize[a]
\item{ako klientskú aplikáciu, ktorá poskytuje údaje klientom}
\item{ako úradnícku aplikáciu, ktorá je určená na zber a úpravu údajov}
\stopitemize

Funkcie katalógu služieb VS možno ozdeliť do štyroch kategórií:

\startitemize
\item{\start \bf automatizačné \stop – zber dát potrebných na automatizáciu}
\item{\start \bf informačné \stop – poskytovanie prehľadu o existujícich službách VS a spôsobu ich spracovania}
\item{\start \bf publikačné \stop – poskytovanie informácií, ktoré sú potrebné na korektné zobrazovanie služieb VS na portáloch VS (kategórie, radenie...)}
\item{\start \bf riadiace \stop – riadenie poskytovania a dodávky služieb VS (tvorba plánu digitalizácie, zodpovednosť za služby...)}
\stopitemize

Časti katalógu služieb VS nie sú len služby vykonávané z úradnej moci, ale taktiež aj služby, ktoré iniciuje klient (subjekt práva). Na vyplnenie katalógu služieb je nutné urobiť nasledujúce kroky:

\startitemize[n]
\item{Identifikovať služby VS a popísať ich atribúty v agendách, ktoré ohlasujete.}
\item{Rozložit služby VS na jednotlivé úkony a popísať ich atribúty.}
\item{Definovať spôsob, akým dochádza k interakciou medzi OVM a klientom a určit obslužný kanál.}
\item{Určiť časové rámce a obslužné kanály pre vykonávanie digitálneho úko-nu a využívanie digitálnych služieb.}
\stopitemize

Vzhľadom na to, že údaje v katalogu služeb VS sú referenčné, je nutné ich udržiavať aktuálne \scr(Digitální a informační agentura, 2023).

V ďalších kapitolách budú objasnené vyššie spomenuté pojmy služba VS a~úkon.

\ppkap{Služba verejnej správy}
Služba VS reprezentuje funkciu (činnosť) úradu, ktorá je poskytovaná konkrétnym OVM (úradníkom) konkrétnemu príjemcovi služby podľa príslušného právneho predpisu. Prináša príjemcovi hodnotu - buď vo forme benefitu alebo splnenia zákonnej povinnosti. Ak ide o interakciu medzi OVM a OVM, nepokladá sa to za službu VS. Pri službe VS ide vždy o interakciu medzi OVM a klientom (a opačne). Každá služba sa skladá z minimálne jedného úkonu. \scr(Digitální a informační agentura, 2023).

\ppkap{Úkon}
Úkon je taktiež interakcia medzi klientom a OVM, no v tomto prípade ide len o jednu interakciu, ktorá vedie k ďalšiemu úkonu (resp. k naplneniu výstupu služby, ak sa jedná o koncový úkon). Úkon sa teda dá definovať ako jeden krok, jedna časť služby VS \scr(Digitální a informační agentura, 2023).

\pkap{Digitální Česko}
Táto kapitola je venovaná iniciatíve Digitálne Česko, keďže to je v tejto práci viackrát zmieňovaný pojem. Ide o ucelenú víziu, ktorá je realizovaná na základe niekoľých koncepcií, plánov a stratégií, ktoré sú v súlade s potrebami ČR a politikou EÚ. Tento projekt pokrýva 3 základné piliere:
\startitemize
\item{\start \bf Česko v digitální Evropě \stop -- vládna koncepcia zameriavajúca sa na jednotný digitálny trh v Európe} 
\item{\start \bf Digitální ekonomika a společnost \stop -- strategický dokument, ktorého cieľom je koordinácia agend z oblastí digitálnej ekonomiky a spoločnosti naprieč verejnou správou, hospodárstvom, sociálnou či akademickou sférou (súvisí aj s kap. 3.1)}
\item{\start \bf Informační koncepce České republiky \stop (kap. 3.6)}
\stopitemize

Vláda ČR považuje program Digitálne Česko za súbor stratégií, ktoré vytvárajú predpoklady pre dlhodobú prosperitu krajiny v ére digitálnej transformácie a revolúcie. (Úřad vlády ČR, 2024)

%\pkap[dizajn-system]{Jednotným dizajn manuálom elektronických služieb}
%haha

\pkap{Integračné platformy (toto ešte dokončím)}
Úlohou integračných platforiem (alebo integration platform-as-a-service, \zlom iPaaS) je prepojiť informácie z rôznych zdrojov (z aplikácií, procesov, služieb...) a pripraviť tak priestor pre rýchlejšie inovácie a automatizáciu. Množstvo podnikov sa prikláňa k riešeniam iPaaS, aby zjednotili a digitalizovali podnikové operácie a mohli používať pri procesoch moderné technológie a umelú inteligenciu. Táto služba môže pomocou konektorov a API rozhraní pomôcť spoločnostiam centralizovane a automatizovane vytvárať, spravovať a monitorovať integračné toky naprieč systémami. \scr(SAP, 2024)

Medzi existujúce "integračné platformy ako služby" patria napríklad SAP Integration Suite, IBM® App Connect, Workato, platforma EMMA a iné. EMMA je prvým a jedinečným riešením v Českej republike. Jej najväčšou výhodou oproti ostatným je, že je stavaná na integráciu služieb z katalógu služieb verejnej správy.

\pkap{Projekt EMMA}
EMMA predstavuje jedinečnú integračnú G2B2B platformu a nadväzujúce služby, ktoré sú efektívne, rýchle a \uv{zabudovateľné} do každodenných procesov klientov verejnej správy, hlavne podnikov. Tieto služby sú navrhnuté a poskytované tak, aby sa dali čo najjednoduchšie integrovať do ERP systémov podnikov\footnote{ERP (Enterprise Resource Planning) systém -- interný informačný systém podniku slúžiaci na správu rôznych činností podniku (účtovníctvo, zásobovanie, personalistika...)}. Zároveň by mali tieto služby podporovať podnikové procesy a zaisťovať prostredníctvom zakomponovaných služieb možnosť plniť svoje povinnosti a vymáhať si svoje práva vočí verejnej správe. \scr(Asseco, 2023)

Hlavným cieľom projektu je vybudovanie EMMA ako súhrn služieb a riešení v oblasti podpory komunikácie komerčného sektoru s verejnou správou a začlenenie do informačných systémov firiem (ERP/FM/CRM systémy).

Cieľovými skupinami sú najmä na skupiny, ktoré potrebujú informačne podporiť komunikáciu subjektov s verejnou správou hlavne v opakujúcich sa, rutinných činnostiach. Ide hlavne o činnosti spojené s vykazovaním, ohlasovaním (za zamestnancov alebo klietov). Medzi konkrétnych cieľových užívateľov patria napríklad personalisti, účtovní a daňoví pracovníci, banky, poisťovne a pod. Potenciálnych zákazníkov možno rozdeliť do dvoch skupín - zákazníci, ktorí nemajú žiaden plnohodnotný ERP systém a zákazníci, ktorí zvažujú zmenu/upgrade používaného ERP systému.

Projekt nadväzuje na Architektonický princíp č. 11: eGovernment jako platforma (Embedded eGovernment) uvedený v IKČR. Architektonické princípy IKČR sú spomenuté v kapitole 3.6. V skratke princíp č. 11 hovorí o tom, že procesy a služby verejnej správy aj s potrebnými technickými nástrojmi musia byť navrhnuté tak, aby organizácie mohli tieto služby jednoducho integrovať do svojich ICT systémov, čo im uľahčí plnenie povinností a využívanie práv voči verejnej správe \scr(Digitální a informační agentura, 2023).

Okrem tohto princípu sa EMMA riadi aj ďalšími procesnými zásadami ustanovenými v IKČR, napr.: 

\startitemize
\item \start\bf Z6 Riadenie výkonnosti a kvality \stop --  meranie výkonnosti a kvality, princípy merateľnosti a spätnej väzby, pravidelné audity
\item \start\bf Z7 Riadenie zodpovednosti za služby a systémy \stop -- každý proces a~služba musí mať svojho vlastníka a garanta
\item  \start\bf Z8 Riadenie ICT služieb \stop -- IT podpora riadená katalógom ICT služieb pre interné a externé procesy
\item \start\bf Z11 Riadenie prínosov a hodnoty \stop -- rozhodovanie založené na ekonomickej výhodnosti, zahŕňa analýzu nákladov, rizík a prínosov, nutnosť spracovania investičného zámeru
\item \start\bf Z16 Využívanie otvoreného software a štandardov \stop -- preferencia otvoreného softvéru a štandardov, podpora udržateľnosti, rozvoja a bezpečnosti  \scr(Digitální a informační agentura, 2023)
\stopitemize

Projekt EMMA môže okrem ekonomických prínosov priniesť aj neekonomické a to ako pre klientov, tak i pre ČR a EÚ. Dajú sa identifikovať napr. ako reputačný prínos ČR, prínos k riešeniu spoločenských výziev EÚ, rozvoj ľudského potenciálu a pod. Zámer projektu zároveň prispieva k naplneniu cieľov Národnej inovačnej stratégie.

\pkap{Platforma SAMO}

Platforma SAMO je v súčasnej dobe základom pre evidenciu služieb EMMA. Názov SAMO vznikol skrátením slov Strategic Asset Management \& Operations system, čo v preklade znamená systém pre strategickú správu majetku. Vývoj jednotlivých modulov začal v roku 1991. Počas posledných vyše 30-tich rokov vývoja sa firme Asseco podarilo do rôznych riešení zapojiť množstvo skúseností a best practices. Pôvodne bolo SAMO vyvinuté ako platforma zameraná hlavne na geografické informačné systémy, čo bol ideálny základ najmä pre spoločnosti spravujúce mestskú infraštruktúru a distribučné siete.

Platforma SAMO sa dá konceptuálne rozdeliť na dve hlavné zložky – evidenčnú a priestorovú. Evidenčná časť sa zaoberá evidenciou majetku a jeho charakteristík, ako sú napr. posledné kontroly či opravy. Priestorová zložka obsahuje geometrické údaje, mapové informácie a podobne. Postupne bol systém rozšírený o procesnú zložku, ktorá zahŕňa riadenie procesov ako sú napríklad hlásenie udalostí či plánovanie údržby. Vznikajú tak agendy na správu majetku, ktoré sa skladajú zo zoznamu entít, editačných formulárov, detailov, stavových diagramov a iných prvkov. 

Vzhľadom na to, že každý zákazník má špecifické potreby, SAMO sa neponúka ako finálny produkt, ale len ako flexibilná platforma zložená z rôznych komponent, ktoré sa skladajú podľa individuálnych požiadaviek zákazníka. Toto prispôsobenie a možnosť agilného vývoja projektov je konkurenčnou výhodou SAMO v oblasti verejnej správy. Vďaka modularite a možnosti znovupoužitia existujúcich metadát a komponentov je SAMO vhodné najmä pre unikátne agen-dy evidenčného charakteru, ktoré majú nejakú GIS zložku, pričom GIS zložka ale nie je podmienkou vhodnosti použitia SAMO. Pri implementácii SAMO aplikácie je nutné meniť hlavne business zložku jednotlivých systémov (logiku akcií a procesov).

Platforma SAMO má 3 základné moduly:

\startitemize
\item \start\bf SAMO EAM \stop (Enterprise Asset Management) --  správa podnikového majetku
\item \start\bf SAMO AIS \stop (Agendový IS) -- procesy verejnej správy
\item  \start\bf SAMO LIDS/GIS \stop -- geografický informačný systém
\stopitemize

Platforma SAMO je základom pre široké spektrum aplikácií používaných v~rozličných sektoroch – od priemyslu a energetiky až po verejnú správu a koncepty inteligentných miest. V Českej republike sa na nej zakladajú projekty pre významné inštitúcie, ako sú Český banský úrad, Český rybársky zväz, Agentúra ochrany prírody a krajiny ČR, čo potvrdzuje jej flexibilitu a široké využitie.


V kontexte tejto práce je cieľom rozšírenie agendového systému (SAMO AIS) na správu zamestnancov a služieb verejnej správy. SAMO AIS predstavuje špecifický modul na podporu procesov verejnej správy. Zahŕňa zadávanie požiadavkov, vyhodnocovanie workflow, notifikácie, analýzu dát, pridávanie priestorových informácií atď. Tento modul je navrhnutý tak, aby bol kompatibilný s inými systémami verejnej správy, využíval otvorené dáta z rôznych zdrojov (napr. registry, katastre) a zároveň poskytoval informácie podľa potrieb koncových užívateľov. Je vhodný pre miestne, ústredné, ale i federálne orgány akéhokoľvek druhu (od malých obcí až po ministerstvá). Cieľom je vybudovať a udržať efektívny e-government a prinášať hodnotu užívateľom a občanom. \scr(Asseco, 2024b)

\kap{Metodika}

\TODO TODO: všetko ostatné, zatiaľ len pár kapitol a poznámok

\TODO
Poznámka: 
For any type of work to be automated, or indeed digitalised, it must
at some point be represented visually in a way that is conducive to
translation into algorithmic instructions for a computer. Hence, the
digitalisation and automating of work requires a certain textualization
and abstraction of previously embodied and situated knowledge (Zuboff, 1988, ZDROJ Andersson strana 2 dole).

\pkap{Popis EMMA}
\TODO dokončiť poriadne

Platforma EMMA poskytuje služby VS pomocou štandardizovaného API
%\footnote{API (Application Programme Interface) -- webové rozhranie, ktoré umožňuje komunikáciu medzi dvomi rôznymi aplikáciami}
, ktoré je jednoducho integrovateľné do ERP systémov. Podniky môžu využívaním platformy EMMA dosiahnuť zníženie administrátorskej záťaže podnikov. Príkladom služby EMMA je \uv{oznámenie o nástupu zamestnanca}. Pri tejto životnej situácii je podnik povinný informovať viaceré subjekty VS, konkrétne ČSSZ, zdravotné poisťovne a MPSV (v prípade zahraničného zamestnanca). Podrobnejšie to bude popísané v kapitolách nižšie.

Obsahom platformy EMMA sú:

\startitemize
\item{interné služby na správu a prevádzku platformy,}
\item{nástroje na využívanie služby prostredníctvom Rozhrania na volanie služieb VS,}
\item{nástroje pre interoperabilitu VS ČR v legislativnom rámci Digital Service Act\footnote{Digital Service Act (Akt o digitálnych službách) -- súbor pravidiel platiacich v celej EÚ, ktorých cieľom je vytvoriť bezpečnejší digitálny priestor, v ktorom budú chránené základné práva všetkých užívateľov digitálnych služieb \scr(Európska komisia, 2022)} a Data Governance Act\footnote{Data Governance Act (Akt o správe dát) -- úsilie zvýšiť dôveru v zdieľanie dát a posilnenie mechanizmov pre zvýšenie dostupnosti dát \scr(Európska komisia, 2022)},}
\item{modul rozhrania pre G2B2B,}
\item{služby API pre integráciu.}
\stopitemize 

\TODO EBSI (https://ec.europa.eu/digital-building-blocks/sites/display/EBSI/Home) (https://assecoce.sharepoint.com/:w:/r/teams/EMMAPEGOV-Analza/Shared%20Documents/Anal%C3%BDza/80%20-%20V%C3%BDstupy/Etapa1%20-%20N%C3%A1vrh%20%C5%99e%C5%A1en%C3%AD/E1.4%20Enterprise%20architektura/EMMA_E1_Enterprise_architektura.docx?d=wa8731ee84cc84c9b9e26b2ac55ed06bf&csf=1&web=1&e=DRJrfT)

\pkap{Popis SAMO}
Ide o súbor integrovaných softvérových riešení, ktoré sú modulárne zložené do komplexného systému. Platforma SAMO slúži na strategickú správu aktív a činností. Platforma SAMO je založená na koncepte SOA \footnote{SOA -- Servisne orientovaná architektúra (Service Oriented Architecture) -- sada princípov a metodológií, ktorá odporúča stavbu aplikácií zo vzájomne nezávislých komponent}. %Považuje sa za platformu určenú na efektívne poskytovanie, nie za produkt samotný.


\ppkap{asi blbost}
Systém kladie dôraz na integritu dát a procesov. Architektúru tvoria základné vrstvy, ktoré sú na sebe technologicky nezávislé a ich komunikáciu zabezpečujú štandardy popísané API. Sú to tieto vrstvy:

\startitemize
\item \start\bf prezentačná vrstva \stop -- skupina webových serverov, ktoré poskytujú služby prezentačnej vrstvy pre interných i externých pracovníkov
\item \start\bf aplikačná vrstva \stop (Agendový IS) -- skupina aplikačných serverov založených na platforme J2EE, na ktorých je implementovaná biznis logika
\item  \start\bf databázová vrstva \stop -- skupina databázových serverov s požadovanou výkonnosťou zabezpečujúcich služby dátového úložiska pre aplikačnú vrstvu
\stopitemize

Viditeľná čast SAMO aplikácie sa nazýva SAMO Dynamic Application - klient, ktorý beží v prehliadači (ľahký klient, html5, js aplikácia) a komunikuje so serverovými komponentami. Medzi ne patrí SAMO Gateway (3 úlohy - web. server (poskytuje metadáta), session+auth, proxy), LIDS Application Server (core komponenta, ktorá obsahuje dátové modely, pôvodne GIS AS), Security Server (vnútorný identity management a správa oprávnení), User Service (drží profil prihláseného užívateľa), SAMO Liferay (Content Management System) (viď obrázok). Okrem komponent samotnej SAMO platformy sú do celkovej SAMO aplikácie zapojené aj ďalšie sw - Docker, ElasticSearch (objektová db pre rýchle fulltextové vyhľadávanie), Docker-compose/Kubernetes, databáza (PostgreSQL (+PostGIS), Oracle (+Oracle Spatial)), NGINX (proxy, SSL). Všetko to sú v Jave napísané aplikácia bežiace na serveri, ktoré sú ale púšťané pomocou Dockeru.

Dynamic app sa skladá z niekoľkých hlavných modulov - cockpit (úvodná obrazovka, ktorá zobrazuje úvodní rozcestník), tzv. pages, ktoré obsahujú browse (zoznam), po rozkliknutí sa zobrazí detail (ktorý môže obsahovať okrem hlavičkových dát aj sekcie obsahujúce ďalšie naviazané entity) a po kliku na editačnú tužku sa zobrazí editačný detail (na modifikovanie hlavičkových informácií o entite, avšak nie dát v spomínaných sekciách).

\TODO prepojenie SAMO a LIDS atd

SAMO používa mikroservice prístup k vývoju softvéru. Každá mikroslužba je zameraná na konkrétnu funkčnosť a môže byť vyvíjaná, nasadená a spravovaná nezávisle od ostatných častí aplikácie. To umožňuje flexibilnejšie škálovanie, rýchlejšie nasadzovanie nových funkcionalít a jednoduchšiu údržbu. SAMO je aplikácia poháňaná metadátami, ktoré obsahujú informácie o vzťahoch medzi časťami  dát, popis užívateľského rozhrania, pravidlá, formuláre a pod. \scr(Asseco, 2023)

\pkap{LIDS}

\TODO čo je vlastne LIDS atd (LIDS je metadátový systém, na ktorom je postavené SAMO.)
LIDS časť systému SAMO je oveľa väčšia ako SAMO Gateway. Riadi veškerú logiku systému - správa dát, riadi prístupy k aplikačnej logike a dátam, security, REST API a pod. SAMO Gateway slúži primárne na to, aby poskytovala metadáta pre Dynamic App. Preto je LIDS popísaný detailnejšie v samostatnej kapitole.

LIDS AS je kontrolovaný metadátami, s ktorými pracujú jednotlivé časti systému. Hlavnou časťou LIDS metadát sú tzv. feature types. Ide v podstate o nejaký typ objektu reálneho sveta (napr. ft\_osoba, ft\_adresa, ft\_zamestnanec...). Feature type definuje atribúty objektu, môže definovať aj geometriu, symboliku a iné vlastnosti. Tzv. feature je inštanciou feature typu, a teda je to reprezentácia objektu reálneho sveta. Feature nesie informácie o tom, aký je to feature type, sémantické atribúty (id, name, type), jeho miesto v databáze (tzv. databázový kontajner), poprípade symboliku a typ geometrie.

Tieto metadáta sú uložené, prenášané a spravované vo forme nasledujúcich XML dokumentov:

\startitemize
\item{\start \bf model.xml \stop - hlavný metadátový súbor, v ktorom sú uložené informácie o tzv. feature types (entita SAMO systému), ich atribútoch, číselníkoch a pod.}
\item{\start \bf presentation.xml \stop - definuje predvolenú symboliku projektu a pod.}
\item{\start \bf thematization.xml \stop - definuje dynamickú symboliku prvkov (na základe štandardu OpenGIS Symbology Encoding)}
\item{\start \bf tool.xml \stop - definuje panely nástrojov špecifických pre projekt}
\item{\start \bf resource.xml \stop - definuje napr. štýly čiar, symboly, fonty, ikony a pod.}
\item{\start \bf option.xml \stop - definuje voliteľné funkcie systému ako napr. kopírovanie prvkov, derivovanie atribútov, zobraziteľné atribúty...}
\stopitemize 

Okrem týchto hlavných XML súborov existuje aj množstvo ďalších. Všetky spomínané súbory majú pevne danú štruktúru popísanú v súboroch typu XSD (XML Schema Definiton). 

Základom pre budovanie aplikácie je vytvorenie dátového modelu, na ktorom sa celá aplikácia stavia a logika sa zapája až potom. Je to z toho dôvodu, že bez modelu sa nie je od čoho odpichnúť. LIDS dokáže ale aj na základe model.xml databázu modifikovať (vytvárať nové tabuľky, atribúty, meniť dátové typy a pod.). Spúšta sa to v administrátorskej konzole, kde sa porovnáva existujúca databáza s xml modelom a vygeneruje sa SQL skript s potrebnými príkazmi a po potvrdení sa to do db pustí. V administrátorskej konzoli existuje aj GUI, ktoré prehľadne zobrazuje všetky feature types, názov kontajneru (db tabuľky) daného ft, atribúty, väzby, stavový diagram (workflow, ak existuje), akcie (operácie, metódie, funkcie) a podobne. 

Okrem dátovej časti (model.xml) obsahuje LIDSová časť aj aplikačnú logiku - niekoľko javascript a json súborov definujúcich stavy, akcie nad entitov a celková potrebná logika správania danej entity.

Poslednou vrstvou je security vrstva, ktorá určuje prístupy k feature typom, atribútom, akciám alebo k riadkom browsu (napr. na základne lokality). Buď je to na základe security číselníku alebo druhá možnosť je nastavenie ownershipu (vlastníctva) na vybrané záznamy (napr. pri žiadostiach na základe funkcí a pod.). 

Vývoj prebieha na nainštalovanom lokálnom prostredí, pričom databáza je ale serverová. Na lokále bežia 2 veci - GTW a LIDS konzola (metadáta) a pripojí sa to na databázu, elastic search a user service na server. Po úpravách sa zmeny commitujú a pushujú a na gite beží CI/CD (na server sa to dostane až keď je CI/CD ok, bez failu). Významný nástroj, ktorý je dôležitý na rozbehnutie localu je utility localtron. 

Súborová štruktúra je rozdelená na 2 veľké celky - configuration a project. V časti project sú uložené rôzne parametre ako verzia, prístup do databáze, informácia o aktuálnom prostredí (vývojové, testovacie, produkčné a pod.). 

Správa uživateľov a skupín. Užívatelia zvyčajne prichádzajú pomocou LDAP od zákazníka. Skupiny určujú práva na feature types (zvyčajne ide o skupinu read, edit a admin, napr. ZAMESTNANCI-read, ZAMESTNANCI-edit, ZAMESTNANCI-admin, no je možné vytvárať aj špeciálne skupiny). Sú určené na to, aby boli užívatelia zaradení do skupiny oprávnení. Okrem toho existuje aj security rola, ktorá určuje napr. práva na dlaždice, tlačítka a pod. (pomocou hasAnyRole). Tieto role sa priradia vybraným skupinám a do skupín sa priradia užívatelia. To zabezpečí, že prihlásený užívateľ má umožnené v aplikácii vidieť a robiť len to, na čo má oprávnenie.

\pkap{EA2LIDS}

\pkap{AMK}

\pkap{Popis metodiky, analýzy a návrhu}

\TODO
TODO

\kap{Poznámky z konzultácie}
- kompetenčný model agendy VS

- centrálne zdielane služby

- porovnanie voči svetu - u nás (v čr) sú prenesená pusobnost a spravna pusobnost obcí
stavebné řízení - obec povoluje schvalovanie stavieb, ale obec môže byť dotknutá riadením (takže je vlastne schvalovatel aj účastník řízení)

- informačný koncepce, národný arch plán a rámec

- pôvodný zámer EMMA (vize a tak)

- právo na digitálne služby -> malo by priniest zlepšienie egov

- Jirka ohladom SAMO

\kap{Výsledky}

\pkap{Model požiadaviek}

\TODO
TODO

\pkap{Use case model}

\TODO
TODO

\pkap{Sekvenčný diagram}

\TODO
TODO

\pkap{Konceptuálny dátový model}

\TODO
TODO

\pkap{Logický dátový model pre SAMO}

\TODO
TODO

\pkap{Výber služby, ktorá bude implementovaná}
\pkap{Charakteristika vybraných služieb verejnej správy}
Na to, aby bolo možné vybrať vhodné služby VS k analýze a ďalším úkonom je potreba analyzovať rôzne životné situácie spojené s verejnou správou. V nasledujúcej tabuľke (viď \in{tabuľka}[situacie]) je zobrazený zoznam takýchto situácií spolu so službou a úradom, ktoré do riešenia danej situácie vstupujú.


  \setupTABLE[column][1][width=12cc]
  \setupTABLE[column][2][width=8cc]
  \setupTABLE[column][3][width=16cc]
  %\setupTABLE[column][4][width=0.25\textwidth]
  \setupTABLE[r][each][align={middle,lohi}]

  \Tabulka{situacie}{Životné situácie spojené so zamestnávateľmi a verejnou správou}{
    \bTABLE
      % Table header
      \bTR
        \bTH Životná situácia \eTH
        \bTH Úrad \eTH
        \bTH Služba VS \eTH
        %\bTH Služby v RPP \eTH
      \eTR

      % Table rows
	\bTR 
	    \bTD [nr=3] Nástup zamestnanca \eTD 
	    \bTD ČSSZ \eTD 
	    \bTD Oznámenie o nástupe do zamestnania (Přihlášky, odhlášky zamestnancov k nemocenskému poisteniu) \eTD  
	\eTR
	\bTR 
	    \bTD ZP \eTD 
	    \bTD Hromadné oznámenie zamestnávateľa (HOZ)  \eTD  
	\eTR
	\bTR 
	    \bTD MPSV/ÚP \eTD 
	    \bTD Informace o nástupe občana cudzinca, ktorý nepotrebuje/potrebuje pracovné oprávnenie do zamestnania  \eTD  
	\eTR
	\bTR 
	    \bTD [nr=4] Zamestnávanie osôb so zdravotným postihnutím\eTD 
	    \bTD MPSV  \eTD  
	    \bTD Evidencia náhradného plnenia \eTD 
	\eTR
	\bTR 
	    \bTD MPSV  \eTD  
	    \bTD Ohlásenie plnenia povinného podielu osôb so zdravotným postihnutím (OZP) \eTD 
	\eTR
	\bTR 
	    \bTD MPSV  \eTD  
	    \bTD Žiadosť o príspevok na zdriadenie pracovného miesta pre OZP \eTD 
	\eTR
	\bTR 
	    \bTD MPSV  \eTD  
	    \bTD Žiadosť o príspevok na Nový podnikateľský program\eTD 
	\eTR
	\bTR 
	    \bTD [nr=2] Voľné miesta\eTD 
	    \bTD MPSV \eTD 
	    \bTD Oznámenie voľných pracovných miest ÚP ČR \eTD  
	\eTR
	\bTR 
	    \bTD MPSV \eTD 
	    \bTD Oznámenie popisu pracovnej pozície pre Jobmatch do evidencie ÚP ČR neregistrovaným uživateľom  \eTD  
	\eTR
	\bTR 
	    \bTD Row 1, Col 1 \eTD 
	    \bTD Row 1, Col 2 \eTD 
	    \bTD Row 1, Col 3 \eTD  
	\eTR
    \eTABLE
  }



\TODO
TODO

\pkap{Implementácia vybranej služby}

\TODO
TODO

\pkap{Návrh testovacích scenárov}

\TODO
TODO

\pkap{Dokumentácia prevedených testov}

\TODO
TODO


\kap{Diskusia}

\TODO
TODO

\kap{Závěr}

\TODO
TODO

%%%%%%%%%%%%%%%%%%%%%%%%% \def\refname{}

\bbib

\publE{
\autor{Anderrsson, C.} \autor{Hallin, A.} \autor{Ivory, C.}
\nazev{Unpacking the digitalisation of public services: Configuring work during automation in local government}
\rok{2022}
\issn{0740624X}
\doi{10.1016/j.giq.2021.101662}
\www{https://www.sciencedirect.com/science/article/pii/S0740624X21000988}
\online{2023-11-19}
\nazevdok{Government Information Quarterly}
\cast{roč. \,39}
}

\publW{
\autor{Ardhaninggar, N.}
\nazev{E-Government Success Stories: Learning from Denmark and Estonia}
\rok{2023}
\www{https://moderndiplomacy.eu/author/nurulardhaninggar/}
\online{2024-01-31}
\nazevdok{moderndiplomacy.eu}
%\podnazev{Informační koncepce ČR}
}

\publX{
\autorkorp{Asseco Central Europe, a.s.}
\online{2023-11-27}
\www{interný SharePoint}
\nazev{SAMO conceptual application architecture}
\rok{2023}
}

\publX{
\autorkorp{Asseco Central Europe, a.s.}
\online{2024-02-09}
\www{interný dokument}
\nazev{SAMO Implementation Guide Version 9.4}
\rok{2024a}
}

\publX{
\autorkorp{Asseco Central Europe, a.s.}
\online{2024-03-03}
\www{https://www.samo-asseco.com/}
\nazev{SAMO – Platform for asset management solutions}
\rok{2024b}
}

\publX{
\autorkorp{Asseco Central Europe, a.s.}
\online{2023-11-26}
\www{interný dokument}
\nazev{Závěrečná zpráva o realizaci výsledků výzkumu a vývoje: VaV softwarové platformy embedded government (EMMA)}
\rok{2023}
}

\publE{
\autor{Barone, L. a kol.}
\nazev{State-of-play report on digital public administration and interoperability}
\rok{2023}
\isbn{978-92-68-08101-3}
\doi{10.2799/686251}
\www{https://op.europa.eu/en/publication-detail/-/publication/e2cf65a7-6719-11ee-9220-01aa75ed71a1/language-en}
\online{2024-1-12}
\nazevdok{Directorate-General for Informatics}
\cast{NO-04-23-973-EN-N}
}

\publW{
\autorkorp{Digitální a informační agentura}
\nazev{Architektura eGovernmentu ČR}
\rok{2023}
\www{https://archi.gov.cz/start}
\online{2023-11-26}
\nazevdok{Národní architektonický plán}
\podnazev{Informační koncepce ČR}
}

\publW{
\autorkorp{Digitální a informační agentura}
\nazev{Architektura eGovernmentu ČR}
\rok{2023}
\www{https://archi.gov.cz/start}
\online{2023-11-26}
\nazevdok{Národní architektonický plán}
\podnazev{Katalog služeb veřejné správy}
}

\publW{
\autorkorp{Digitální a informační agentura}
\nazev{Architektura eGovernmentu ČR}
\rok{2023}
\www{https://archi.gov.cz/start}
\online{2023-11-26}
\nazevdok{Národní architektonický plán}
\podnazev{Slovník pojmů eGovernmentu}
}

\publW{
 \online{2023-11-19}
 \autorkorp{Evropská komise.}
 \nazev{Balíček aktu o digitálních službách}
%\podnazev{Metodika}
 \www{https://digital-strategy.ec.europa.eu/cs/policies/digital-services-act-package}
 \nazevdok{Shaping Europe’s digital future}
 \rok{2022}
}

\publW{
 \online{2023-11-19}
 \autorkorp{Evropská komise.}
 \nazev{Index digitální ekonomiky a společnosti (DESI) 2022}
\podnazev{Česko}
 \www{https://digital-strategy.ec.europa.eu/en/policies/desi-czech-republic}
 \nazevdok{Shaping Europe’s digital future}
 \rok{2022}
}

\publW{
 \online{2023-11-19}
 \autorkorp{Evropská komise.}
 \nazev{Index digitální ekonomiky a společnosti (DESI) 2022}
\podnazev{Metodika}
 \www{https://digital-strategy.ec.europa.eu/cs/policies/desi}
 \nazevdok{Shaping Europe’s digital future}
 \rok{2022}
}

%\publA{
%\autor{OECD}
%\nazev{ Government at a Glance}
%\nakl{OECD Publishing}
%\vyd{2023}
%\rok{2023}
%\isbn{978-92-64-85180-1}
%\rozsah{60}
%}

%\publE{
% \autor{Karunia, L. a kol.}
% \nazev{Analysis of the Factors that Affect the Implementation of EGovernment in Indonesia}
% \nazevdok{International Journal of Membrane Science and Technology}
% \cast{Vol.\,10, No.\,3}
% \rok{2023}
% \umist{46}{54}
% %\issn{0896-3207}
%\doi{https://doi.org/10.1063/5.0118820}
%}

\publW{
\autor{Kilinger, A.}
\nazev{Obligatory Slovakian Information System (IS EFA) for exchanging B2G and B2B E-Invoice}
 \online{2024-01-29}
 \www{https://blog.seeburger.com/new-obligatory-slovakian-information-system-is-efa-for-b2g-and-b2b-e-invoicing/}
\issn{978-92-64-85180-1}
 \nazevdok{SEEBURGER}
 \rok{2023}
}

\publW{
 \autorkorp{Ministerstvo financií Slovenskej republiky}
 \nazev{Informačný systém elektronickej fakturácie - BETA}
 \online{2024-01-29}
 \www{https://web-einvoice-demo.mypaas.vnet.sk/}
 \nazevdok{e-Faktúra}
 \rok{©~2024}
}

\publA{
 \autor{OECD}
 \nazev{Government at a Glance}
 \nakl{Paris}{OECD Publishing}
 %\vyd{2023}
 \rok{2023}
 \xisbn{978-92-64-85180-1}
 \rozsah{234\stran}
}

\publX{
 \autorkorp{SAP}
 \nazev{What is integration platform as a service (iPaaS)?}
 \online{2024-03-03}
 \www{https://www.sap.com/products/technology-platform/integration-suite/what-is-ipaas.html}
 \rok{©~2024}
}

\publD{%
 \autorkorp{United Nations}
 \nazev{E-Government Survey 2022}
 \nazevdok{The Future of Digital Government}
\nakl{UN}{New York}
\rok{2022}
\isbn{978-92-1-123213-4}
 \umist{32}{51}
}

%https://desapublications.un.org/sites/default/files/publications/2022-09/Web%20version%20E-Government%202022.pdf

\publW{
 \autorkorp{Úřad vlády ČR}
 \nazev{Tři pilíře Digitálního Česka}
 \online{2024-03-01}
 \www{https://digitalnicesko.gov.cz/vize/}
 \nazevdok{Digitální Česko}
 \rok{©~2024}
}

%\publE{
%\autor{van der Linden, N. a kol.}
%\nazev{eGovernment Benchmark 2023: Insight Report}
%\rok{2023}
%\isbn{978-92-68-05653-0}
%\doi{10.2759/474056}
%\www{https://espanadigital.gob.es/sites/espanadigital/files/2023-10/1_eGovernment_Benchmark_2023__Insight_Report_tmnnsE9rmVDxpAZ8IJECpnUZGLA_98708.pdf}
%\online{2024-2-7}
%\nazevdok{Connecting Digital Governments}
%\cast{KK-BH-23-001-EN-N}
%}

\publE{
\autor{van der Linden, N. a kol.}
\nazev{eGovernment Benchmark 2022: Insight Report}
\rok{2022}
\isbn{ 978-92-76-49793-6}
\doi{10.2759/488218}
\www{https://prod.ucwe.capgemini.com/wp-content/uploads/2022/07/eGovernment-Benchmark-2022-1.-Insight-Report.pdf}
\online{2024-2-7}
\nazevdok{Connecting Digital Governments}
\cast{KK-08-22-084-EN-N}
}



\ebib

\stopbodymatter

%%%%%%%%%%%%%%%%%%%%%%%% Varianta, kdy seznamy jsou součástí práce a nejsou uvedeny v přílohách

\setupsectionblock[backmatter][before={\setuplist[kap][before={}]}]

\startbackmatter

\THESIScompletelistof{tables}
\THESIScompletelistof{figures}
\THESIScompletelistof{abbreviations}
%\THESIScompletelistof{codes}

\stopbackmatter

%%%%%%%%%%%%%%%%%%%%%%%% Varianta, kdy seznamy nejsou součástí práce, ale jsou zařazeny do příloh.
%%%%%%%%%%%%%%%%%%%%%%%% Níže uvedeným čtyřem příkazům postačí odstranit znak procenta.
%%%%%%%%%%%%%%%%%%%%%%%% Naopak před výše uvedené čtyři příkazy je potřeba znak procenta vložit.

\startappendices

\cast{Přílohy}
%\THESIScompletelistof{tables}
%\THESIScompletelistof{figures}
%\THESIScompletelistof{abbreviations}
%\THESIScompletelistof{codes}

\stopappendices

\stopthesis

\endinput		

%%%% TODO %%%%%%%%%%%%%%%%%%%%%%%%%%%%%%
Tady si můžeš psát poznámky, které se neobjeví ve výstupu.
