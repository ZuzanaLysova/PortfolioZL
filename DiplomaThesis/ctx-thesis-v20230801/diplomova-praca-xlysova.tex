%
\usemodule[ctx-thesis-v0.991]
\usemodule[bib.sty-v2.78]

\setupthesis[sk,mendelu,pef,none][ % jazyk,univerzita,fakulta,ústav/katedra/pracoviště ; language,university,faculty,department
  type={dp},                 % bp,dp,pp,zp,sp,pr,pt aj./etc.
  authorname={Zuzana},	     % jméno
  authorsurname={Lysová},        % příjmení
  authordegree={Bc.},	     % titul před jménem
  authorgender={F},          % pohlaví (holky mají F)
  supervisor={RNDr. Zuzana Špendel, Ph.D.},        % vedoucí práce
  title={Analýza, návrh a implementácia softwarovej platformy pre firmu Asseco Central Europe, a.s.},                           % název práce
  titleen={Analysis, design and implementation of software platform for Asseco Central Europe, a.s.}, 	           % název práce anglicky
  keywords={bla,blabla,bla,blabla},    %
  keywordsen={aa,xxx,vvv,aa,bbb}, %
%  acknowledgement={Děkuji své babičce, že mi napekla na cestu buchty.},	           % poděkování
  abstract={},		                           % český abstrakt
  abstracten={},		                   % anglický abstrakt
  location={Brno,Brne},	   % místo vydání (za čárkou 6. pád) ; location (second parameter is not necessary for English)
%  year={2021},		   % rok odevzdání práce (automaticky aktuální rok) ; year, the default is the current year
%  thesisassignmentform={img/file001.png,img/file002.png},  % seznam souborů se skenem zadání práce; file is thesis assignment
]

\startthesis
\startbodymatter

\kap{Úvod}

\TODO
TODO niekde ku koncu - VÝHODY A NEVÝHODY EMMA - napr. zamestnanci...

\kap{Cieľ}

\abbreviation{G2B2B}{Government to Business to Business}

Digitalizácia je veľmi dôležitá oblasť spojená s rýchlym rozvojom technológií. Či už ide o súkromný alebo verejný sektor, existuje mnoho procesov, ktoré by sa mohli zjednodušiť. Cieľom tejto práce je analýza a návrh vybraných G2B2B (\infull{G2B2B}) služieb EMMA. Po analýze bude vybraná služba implementovaná. Súčasťou bude aj začlenenie vybraných služieb do katalógu služieb EMMA.


\kap{Súčasný stav}

V dnešnej dobe je digitalizácia verejnej správy častou témou rôznych diskusií. Digitalizácia sľubuje využívanie efektívnejších, účinnejších a transparentnejších služieb v rôznych sektoroch \scr(Andersson, 2022).

Existuje viacero spôsobov "merania" úrovne rozvinutosti krajín v oblasti digitalizácie a e-governmentu. Patrí medzi ne napríklad Index digitálnej ekonomiky a spoločnosti, Index rozvoja e-Governmentu, E-Government Benchmark a iné.

\pkap{Index digitálnej ekonomiky a spoločnosti}
Európska komisia sleduje pokrok členských štátov v digitálnej oblasti od roku 2014 a každý rok zverejňuje informácie o indexe digitálnej ekonomiky a spoločnosti (Digital Economy and Society Index, DESI). DESI zoraďuje štáty podľa úrovne digitalizácie a zároveň analyzuje ich relatívny pokrok za posledných 5 rokov vzhľadom na ich počiatočnú situáciu.

Oblasti, ktoré skúma DESI sú:

\startitemize
\item{\start\bf ľudský kapitál\stop - internetové znalosti používateľov, pokročilé znalosti ľudí v IT oblasti}
\item{\start\bf konektivita\stop - využitie a pokrytie pevného a mobilného pripojenia a ich ceny}
\item{\start\bf integrácia digitálnych technológií \stop - digitálne technológie pre firmy (cloud, umelá inteligencia...), e-commerce \footnote{e-commerce -- obchodné činnosti prevádzané na internete a pomocou ďalších elektronických prostriedkov}}
\item{\start\bf digitálne verejné služby \stop - e-government, otvorené dáta \footnote{otvorené data (open data, vládne dáta) -- informácie verejného sektoru, ktoré sú bezplatne dostupné na akékoľvek účely}}
\stopitemize 

Európska komisia a Rada prejednávajú rozdhodnutie o politickom programe "Cesta k digitálnej dekáde", ktorý určuje ciele na úrovni EÚ, ktoré majú byť splnené do roku 2030. Cieľom je zaistiť to, aby bola digitálna transformácia komplexná a udržateľná a aby prebehla vo všetkých odvetviach hospodárstva. Dosiahnutie cieľa programu závisí na všetkých členských krajinách a na ich spoločnom úsilí \scr(Európska komisia - Metodika, 2022).

Česká republika je podľa výsledkov DESI za rok 2022 na 19. mieste (z 27 členských štátov) (viď \in{obrázok}[obr:DESI]). V porovnaní s rokom 2021 sa Česká republika zlepšila v oblasti digitálnych verejných služieb a konektivite. Zhoršila sa v integrácii digitálnych technológií. Česká vláda má po prvýkrát od roku 2007 osobu zodpovednú za digitalizáciu verejnej správy - miestopredsedu Ivana Bartoša  a pokračuje v prevádzaní stratégie "Digitálne Česko" z roku 2018 (aktualizovanej v roku 2020) \scr(Európska komisia - Česko, 2022).
\obrazekB{obr:DESI}
{Index digitálnej ekonomiky a spoločnosti 2022 (Európska komisia - Česko, 2022)}{images/DESI.png}{width=30cc}

\obrazekB{obr:DESI}
{Index DESI 2022 - relatívne výsledky v jednotlivých oblastiach (Európska komisia - Česko, 2022)}{images/DESI2.png}{width=30cc}

Graf na \in{obrázku}[obr:DESI] ukazuje porovnanie jednotlivých oblastí indexu DESI s priemerom 27 členských štátov EÚ. Taktiež je tam zobrazený aj výsledok krajín s najvyšším skóre (Fínsko a v tesnom závese Dánsko) \scr(Európska komisia - Česko, 2022).

\pkap{Index rozvoja e-governmentu}
Prieskum Organizácie Spojených Národov slúži na hodnotie e-governmentu naprieč všetkými 193 členskými štátmi. Tento prieskum hodnotí krajiny na základe Indexu rozvoja e-governmentu (E-Government Development Index, EGDI), ktorý je kombináciou primárnych dát (zbieraných a vlastnených OSN) a sekundárnych dát (od iných agentúr) \scr(United Nations, 2024)

EGDI sa získava váženým priemerom troch indexov z nasledovných oblastí:

\startitemize
\item{\start\bf online služby \stop \footnote{Online Services Index (OSI)} - hodnotenie verejných portálov na základe 5 kritérií (inštitucionálny rámec, poskytovanie služieb, poskytovanie obsahu, technológie a digitálna účasť občanov)}
\item{\start\bf telekomunikačná infraštruktúra  \stop \footnote{Telecommunications Infrastructure Index (TII)} - hodnotí úroveň rozvoja infraštruktúry nevyhnutnej pre e-vládu, vrátane pripojenia na internet, infraštruktúry širokopásmového prístupu a mobilných sietí}
\item{\start\bf ľudský kapitál \stop \footnote{Human Capital Index (HCI)} - hodnotí vzdelanie a úroveň zručností obyvateľstva krajiny, s dôrazom na faktory ako miera gramotnosti, zapojenie do vzdelávania a dostupnosť kvalifikovaných odborníkov v oblasti informačných a komunikačných technológií}
\stopitemize 

Podľa výšky indexov sa dokážu krajiny rozdeliť do 4 skupín - krajiny s veľmi vysokým, vysokým, stredným a nízkym EGDI. Podľa prieskumu z roku 2022 patrí do veľmi vysokého indexu 60 krajín (31~\%) , do vysokého 73 (38~\%), do stredného 53 (27,5~\%) a 7 krajín (3,5~\%) má nízky index rozvoja e-governmentu.

\obrazekB{obr:EGDI-mapa}
{Geografické rozloženie štyroch EGDI kategórií (United Nations, 2022)}{images/EGDI-mapa.png}{width=30cc}

Medzi najviac rozvinuté krajiny podľa EGDI patria ako aj pri DESI Dánsko a Fínsko. Česko sa nachádza na 45. mieste, no stále má index v kategórii veľmi vysoký. Porovnanie jednotlivých hodnôt je v \in{tabuľke}[EGDI]

\setupTABLE[frame=on]
\setupTABLE[row][first][background=color, backgroundcolor=gray, style=bold]
\setupTABLE[column][1][width=10cc]
\setupTABLE[column][2][width=6cc]
\setupTABLE[column][3][width=4cc]
\setupTABLE[column][4][width=4cc]
\setupTABLE[column][5][width=4cc]
\setupTABLE[column][6][width=4cc]
\setupTABLE[r][each][align={middle,lohi}]

\Tabulka{EGDI}{Porovnanie EGDI najrozvinutejších krajín s ČR (United Nations, 2022)}{
\bTABLE
  \bTR
    \bTH Krajina \eTH
    \bTH EGDI poradie\eTH
    \bTH OSI \eTH
    \bTH HCI \eTH
    \bTH TII \eTH
    \bTH EGDI \eTH
  \eTR
  \bTR
    \bTD Dánsko \eTD
    \bTD 1 \eTD
    \bTD 0.9797 \eTD
    \bTD 0.9559 \eTD
    \bTD 0.9725 \eTD
    \bTD 0.9753 \eTD
  \eTR
  \bTR
    \bTD Fínsko \eTD
    \bTD 2 \eTD
    \bTD 0.9833 \eTD
    \bTD 0.9640 \eTD
    \bTD 0.9172 \eTD
    \bTD 0.9533 \eTD
  \eTR
  \bTR
    \bTD ... \eTD
    \bTD ... \eTD
    \bTD ... \eTD
    \bTD ... \eTD
    \bTD ... \eTD
    \bTD ... \eTD
  \eTR
  \bTR
    \bTD Česká republika \eTD
    \bTD 45 \eTD
    \bTD 0.6693 \eTD
    \bTD 0.9114 \eTD
    \bTD 0.8456 \eTD
    \bTD 0.8221 \eTD
  \eTR
  \bTR
    \bTD Ukrajina \eTD
    \bTD 46 \eTD
    \bTD 0.8148 \eTD
    \bTD 0.8669 \eTD
    \bTD 0.7270 \eTD
    \bTD 0.8029 \eTD
  \eTR
  \bTR
    \bTD Slovenská republika \eTD
    \bTD 47 \eTD
    \bTD 0.7260 \eTD
    \bTD 0.8436 \eTD
    \bTD 0.8328 \eTD
    \bTD 0.8008 \eTD
  \eTR
\eTABLE
}


\pkap{ eGovernment Benchmark}
eGovernment Benchmark monitoruje digitalizáciu verejných služieb 35 európskych krajín (tzv. EU27+, ktoré pozostávajú z 27 členov Európskej únie + Island, Nórsko, Švajčiarsko, Albánsko, Čierna hora, Severné Macedónsko, Srbsko a Turecko) \scr(van der Linden, 2022).

Prieskum eGovernment Benchmark skúma tieto 4 oblasti:

\startitemize
\item{\start\bf orientácia na užívateľa \stop - miera poskytovania online služieb, mobile-friendly služby, online podpora a spätná väzba}
\item{\start\bf transparentnosť  \stop - informácie o tom, ako sú poskytované služby VS, spracovaní osobných údajov a pod.}
\item{\start\bf kľúčové faktory \stop - dostupnosť technologických faktorov v súvislosti so službami VS}
\item{\start\bf cezhraničné služby \stop - jednoduchosť používania služieb VS pre občanov zo zahraničia a mechanizmy podpory a spätnej väzby pre takýchto občanov}
\stopitemize 

\obrazekB{obr:benchmark-mapa}
{Geografické rozloženie eGovernment vyspelosti podĺa prieskumu eGovernment Benchmark (van der Linden, 2022)}{images/benchmark-mapa.png}{width=30cc}

Na základe týchto 4  oblastí získavajú krajiny tzv. "skóre eGovernment vyspelosti", ktorého škála je od 0 do 100. Vedúcimi krajinami boli podľa najnovšieho prieskumu (z roku 2022) Malta a Estónsko. Česká republika získala je na 22. mieste \scr(van der Linden, 2022). 

\setupTABLE[frame=on]
\setupTABLE[row][first][background=color, backgroundcolor=lightgray, style=bold]
\setupTABLE[column][1][width=12cc]
\setupTABLE[column][2][width=6cc]
\setupTABLE[column][3][width=14cc]
\setupTABLE[r][each][align={middle,lohi}]
\Tabulka{benchmark}{Porovnanie skóre eGovernment Benchmark najrozvinutejších krajín s ČR (van der Linden, 2022)}{
\bTABLE
  \bTR
    \bTH Krajina \eTH
    \bTH poradie\eTH
    \bTH eGovernment maturity score \eTH
  \eTR
  \bTR
    \bTD Malta \eTD
    \bTD 1 \eTD
    \bTD 96 \eTD
  \eTR
  \bTR
    \bTD Estónsko \eTD
    \bTD 2 \eTD
    \bTD 90 \eTD
  \eTR
  \bTR
    \bTD ... \eTD
    \bTD ... \eTD
    \bTD ... \eTD
  \eTR
  \bTR
    \bTD Česká republika \eTD
    \bTD 22 \eTD
    \bTD 63 \eTD
  \eTR
  \bTR
    \bTD Bulharsko \eTD
    \bTD 23 \eTD
    \bTD 61 \eTD
  \eTR
  \bTR
    \bTD Bulharsko \eTD
    \bTD 23 \eTD
    \bTD 61 \eTD
  \eTR
  \bTR
    \bTD Taliansko \eTD
    \bTD 24 \eTD
    \bTD 61 \eTD
  \eTR
  \bTR
    \bTD Chorvátsko \eTD
    \bTD 25 \eTD
    \bTD 61 \eTD
  \eTR
  \bTR
    \bTD Slovenská republika \eTD
    \bTD 26 \eTD
    \bTD 60 \eTD
  \eTR
\eTABLE
}


\pkap{Súčasný stav digitálnej verejnej správy v Česku}

Posunom doby sa stále viac vecí spojených s verejnou správou dá riešiť online. Zamestnávatelia sa stretávajú na pravidelnej báze s rôznymi procesmi nutnými k chodu podniku a potrebujú mať všetko v súlade so zákonmi a pravidlami verejnej správy. Aktuálne to musia všetko riešiť buď osobne na úradoch alebo v tom lepšom prípade online na rôznych portáloch verejnej správy.

Procesy, ktoré sa dajú v ČR vyriešiť online sú častokrát dostupné pomocou formulárov na portáloch ministerstiev. Príkladmi takýchto portálov sú portál Ministerstva práce a sociálnych vecí (https://www.mpsv.cz/), portál Českej správy sociálneho zabezpečenia (https://www.cssz.cz/), Portál občana (https://portal.gov.cz/) a podobne. Tieto portály poskytujú rôzne formuláre, ktoré umožnia občanom vybaviť rôzne požiadavky online. Ide pri tom o komunikáciu medzi klientom a orgánom verejnej moci (OVM). Cieľom platformy EMMA od firmy Asseco je, aby tieto služby boli zapojené do interných informačných systémov podnikov. To bude viesť k tomu, že zamestnávatelia všetko vybavia na jednom mieste a nebudú musieť vypĺňať množstvo formulárov na množstve rôznych portálov.

Príkladom EMMA služby je nástup zamestnanca do zamestnania. Povinnosťou zamestnávateľa pri nástupe zamestnanca je oznámiť nástup príslušným úradom a poisťovniam. Existuje viac možností, ako to urobiť, no vždy to zahŕňa viacero oddelených činností. Jednou z možností je, že musí navštíviť ePortal ČSSZ (Česká správa sociálního zabezpečení), Portál zdravotních pojišťoven a v prípade zamestnávania cudzincov aj ePortal MPSV (Ministerstva práce a sociálních věcí). Služba EMMA umožní vyplniť údaje len raz v jednom formulári a spojí sa s atomickými službami úradov. 

Táto služba spolu s ďalšími bude uvedená na trh po ukončení realizácie projektu, čo môže výrazne zjednodušiť biznis procesy podnikov. 

\pkap{Súčasný stav digitálnej verejnej správy v zahraničí}

\TODO {na základe tých indexov, aj zahranicie aj cesko}

V tejto kapitole a jej podkapitolách je zanalyzovaná a popísaná situácia v krajinách, ktoré sú buď blízko Českej republike (polohou, kultúrou, rozvinutosťou...) alebo krajiny, o ktorých je známe, že nejakým spôsobom v oblasti digitálnych služieb vynikajú.

\ppkap{Slovensko}
Prvou analyzovanou krajinou je Slovensko, ako najbližší sused Českej republiky. Na Slovensku existuje množstvo elektronických služieb verejnej správy, ktoré sú dostupné prostredníctvom Ústredného portálu verejnej správy (ÚPVS). Tento portál umožňuje centrálne vykonávať elektronickú úradnú komunikáciu s rôznymi orgánmi verejnej moci a pristupovať k spoločným modulom. %Jeho dizajn je ale pomerne zastaralý a neprehľadný (viď \in{obrázok}[obr:portal\_sk]).

Na základe analýzy tohto portálu je možné tvrdiť, že Slovenská republika sa zatiaľ špecializuje viac na vzťah občan-štát, ako na vzťah občan-zamestnávateľ-štát. V popredí je oblasť e-financovania, hlavne situácie týkajúce sa daní. Ministerstvo financií Slovenskej republiky v spolupráci s daňovými úradmi zaviedlo štandardizovaný proces pre elektronické prenosy faktúr a to prostredníctvom Informačného systému elektronickej fakturácie (IS EFA), ktorý umožňuje odosielanie štruktúrovaných elektronických fakturačných údajov do slovenskej daňovej správy. Používanie tohto IS bude časom povinné pre inštitúcie štátnej správy, verejnej správy a taktiež pre podnikateľov, ktorí im dodávajú tovary a služby. Tento systém je v princípe podobný systému e-kasa, ktorý prepája všetky slovenské registračné pokladnice s online portálom slovenskej daňovej správy \scr(Kilinger, 2023; MFSR, 2024). Portály eKasa a eFaktúry už podliehajú Jednotným dizajn manuálom elektronických služieb (viac informácií v \in{kapitole}[dizajn-system]).

\ppkap{Poľsko}


\ppkap{Dánsko a Estónsko}
Denmark’s outstanding performance in e-government is the UN’s \#1 ranking in the E-Government Development Index in 2018, 2020, and 2022, as well as the highest number of citizens using e-government services in the entire European Union, with 93\% of Danish internet users using digital public services in 2021. Meanwhile, according to the 2022 Digital Economy and Society Index by European Commission, Estonia is the best performing country in the sector of digital public services, and it is recognized as a leader in e-government, outperforming Central and Eastern European countries.
The two countries have carefully and precisely defined their digitalization goals and have pursued them. To develop a wide range of efficient and well-integrated digital public services, they have supported digital innovation, enacted laws to promote the adoption of digital services, educated the public, and engaged in public-private partnerships. These developments have resulted in significant infrastructural solutions that benefit all citizens and businesses in their daily lives when interacting with the government. In the same way that Denmark has borger.dk, Estonia has eesti.ee as an integrated public service portal, both of which are available 24/7 and protected against high demand. This robust government service delivers a superior citizen experience by meeting the four core tenets of user needs by Aarron Walters: functional, reliable, usable, and pleasurable (Queue IT, 2023).

%\obrazekB{obr:portal\_sk}
%{Ústredný portál verejnej správy SR (Slovensko.sk, 2024)}{images/portal_sk.png}{width=30cc}

%\QUES Opravdu musím psát literární rešerši? 

\kap{Metodika}

\TODO
Poznámka: 
For any type of work to be automated, or indeed digitalised, it must
at some point be represented visually in a way that is conducive to
translation into algorithmic instructions for a computer. Hence, the
digitalisation and automating of work requires a certain textualization
and abstraction of previously embodied and situated knowledge (Zuboff, 1988, ZDROJ Andersson strana 2 dole).

\pkap{Popis EMMA}

EMMA predstavuje jedinečnú G2B2B platformu a nadväzujúce služby, ktoré sú efektívne, rýchle a \uv{zabudovateľné} do každodenných procesov klientov verejnej správy, hlavne podnikov. Tieto služby sú navrhnuté a poskytované tak, aby sa dali čo najjednoduchšie integrovať do ERP systémov podnikov\footnote{ERP (Enterprise Resource Planning) systém -- interný informačný systém podniku slúžiaci na správu rôznych činností podniku (účtovníctvo, zásobovanie, personalistika...)}. Zároveň by mali tieto služby podporovať podnikové procesy a zaisťovať prostredníctvom zakomponovaných služieb možnosť plniť svoje povinnosti a vymáhať si svoje práva vočí verejnej správe \scr(Asseco, 2023) .

Platforma EMMA poskytuje služby VS pomocou štandardizovaného API\footnote{API (Application Programme Interface) -- webové rozhranie, ktoré umožňuje komunikáciu medzi dvomi rôznymi aplikáciami}, ktoré je jednoducho integrovateľné do ERP systémov. Podniky môžu využívaním platformy EMMA dosiahnuť zníženie administrátorskej záťaže podnikov. Príkladom služby EMMA je \uv{oznámenie o nástupu zamestnanca}. Pri tejto životnej situácii je podnik povinný informovať viaceré subjekty VS, konkrétne ČSSZ, zdravotné poisťovne a MPSV (v prípade zahraničného zamestnanca). Podrobnejšie to bude popísané v kapitolách nižšie.

Obsahom platformy EMMA sú:

\startitemize
\item{interné služby na správu a prevádzku platformy,}
\item{nástroje na využívanie služby prostredníctvom Rozhrania na volanie služieb VS,}
\item{nástroje pre interoperabilitu VS ČR v legislativnom rámci Digital Service Act\footnote{Digital Service Act (Akt o digitálnych službách) -- súbor pravidiel platiacich v celej EÚ, ktorých cieľom je vytvoriť bezpečnejší digitálny priestor, v ktorom budú chránené základné práva všetkých užívateľov digitálnych služieb \scr(Európska komisia, 2022)} a Data Governance Act\footnote{Data Governance Act (Akt o správe dát) -- úsilie zvýšiť dôveru v zdieľanie dát a posilnenie mechanizmov pre zvýšenie dostupnosti dát \scr(Európska komisia, 2022)},}
\item{modul rozhrania pre G2B2B,}
\item{služby API pre integráciu.}
\stopitemize 

\TODO EBSI (https://ec.europa.eu/digital-building-blocks/sites/display/EBSI/Home) (https://assecoce.sharepoint.com/:w:/r/teams/EMMAPEGOV-Analza/Shared%20Documents/Anal%C3%BDza/80%20-%20V%C3%BDstupy/Etapa1%20-%20N%C3%A1vrh%20%C5%99e%C5%A1en%C3%AD/E1.4%20Enterprise%20architektura/EMMA_E1_Enterprise_architektura.docx?d=wa8731ee84cc84c9b9e26b2ac55ed06bf&csf=1&web=1&e=DRJrfT)

\pkap{Popis SAMO}

Názov SAMO vznikol skrátením slov Strategic Asset Management \& Operations system. Ide o súbor integrovaných softvérových riešení, ktoré sú modulárne zložené do komplexného systému. Platforma SAMO slúži na strategickú správu aktív a činností.

Vývoj jednotlivých modulov začal v roku 1991. Počas posledných vyše 30-tich rokov vývoja sa firme Asseco podarilo do rôznych riešení zapojiť množstvo skúseností a best practices.

Platforma SAMO je založená na koncepte SOA \footnote{SOA -- Servisne orientovaná architektúra (Service Oriented Architecture) -- sada princípov a metodológií, ktorá odporúča stavbu aplikácií zo vzájomne nezávislých komponent}. Považuje sa za platformu určenú na efektívne poskytovanie, nie za produkt samotný.

SAMO používa mikroservice prístup k vývoju softvéru. Každá mikroslužba je zameraná na konkrétnu funkčnosť a môže byť vyvíjaná, nasadená a spravovaná nezávisle od ostatných častí aplikácie. To umožňuje flexibilnejšie škálovanie, rýchlejšie nasadzovanie nových funkcionalít a jednoduchšiu údržbu. SAMO je aplikácia poháňaná metadátami, ktoré obsahujú informácie o vzťahoch medzi časťami  dát, popis užívateľského rozhrania, pravidlá, formuláre a pod. \scr(Asseco, 2023)

\pkap{Architektura eGovernmentu}

eGovernment je pojem popisujúci modernú digitálnu verejnú správu, ktorá využíva k výkonu svojich právomocí digitálnu infraštruktúru. Táto infraštruktúra realizuje sadu služieb informačných technológií (ICT služieb), ktoré sú zdieľané, dôveryhodné, prepojené, bezpečné, automatizované, efektívne a ľahko používateľné pre užívateľov. Služby eGovernmentu sú určené občanom, firmám, podnikateľom i úradníkom. Synonymami pojmu eGovernment sú "digitálny government" alebo "digitálna verejná správa" \scr(Digitální a informační agentura, 2023).

\ppkap{Poslanie a vízia eGovernmentu v Českej republike}
Digitálna verejná správa používa rôzne poskytnuté a dostupné informácie, ktoré automatizovane spracuváva s cieľom obmedziť, respektíve znížt množstvo podania a objemu infomrácií zo straný užívateľov služieb VS.

Poslaním eGovernmentu je: 

\citat{Poskytovať klientom verejnej správy jednoduché a efektívne služby, ktoré im uľahčia dosiahnutie ich práv a nárokov, ako aj plnenie ich povinností a záväzkov vo vzťahu k verejnej správe.} \scr(Digitální a informační agentura, 2023)

Vízia eGovernmentu v ČR do konca horizontu Informačnej koncepcie ČR (viac popísaná v kapitole 4.3.2 Informační koncepce ČR) je: 

\citat{Česká republika je jednou z popredných krajín v užívateľskej prívetivosti verejnej správy vďaka svojmu klientsky orientovanému prístupu, modernému dizajnu úradných procesov a efektívnemu využívaniu digitálnych a nedigitálnych technológií.} \scr(Digitální a informační agentura, 2023)

\ppkap{Informační koncepce ČR}
Informační koncepce ČR (ďalej ako IKČR) rozpracováva vyššie spomenutú víziu do rôznych cieľov, ktoré realizujú jednotlivé orgány VS. To, či ciele boli naplnené alebo nie ukazuje stav plnenia zadefinovaných cieľov a pozícia v rebríčku podľa DESI (rozobraté v kapitole 3.1 Index digitálnej ekonomiky a spoločnosti). Všetky povinné subjekty podľa zákona  č. 365/2000 Sb., o informačných systémoch majú povinnosť viesť vlastné informačné koncepcie a vždy ich musia uviesť do súladu s Informačnou koncepciou ČR. Je to prakticky koncepcia rozvoja informačných systémov verejnej správy, ktorú spracováva Ministerstvo vnútra a schvaľuje vláda. Je vypracovaná na základe ustanovenia § 5a, Zákona č. 365/2000 Sb., o informačných systémoch verejnej správy. Jej časti sú najmä:

\startitemize
\item{architektonické principy eGovernmentu a elektronizácie verejnej správy,}
\item{efektívny rozvoj digitálnej verejnej správy a informačných systémov verejnej správy (ISVS),}
\item{zásady riadenia ICT vo verejnej správe,}
\item{základné koncepčné povinnosti pre budovanie, rozvoj a prevádzku ISVS a ich vzájomné prepojenie a pre budovanie spoločných služieb eGovernmentu.}
\stopitemize

IKČR je základný dokument, ktorý stanovuje ciele ČR v oblasti ISVS a všeobecné princípy obstarávania, tvorby, správy a prevádzky ISVS v ČR. Obsahuje predovšetkým:
\startitemize
\item{ciele a podpora oblasti eGovernmentu (zo strany informačných systémov verejnej správy),}
\item{zásady riadenia útvarov informatiky a riadenie životného cyklu ISVS,}
\item{architektonické principy pre návrh a rozvoj ISVS a ich služieb.\scr(Digitální a informační agentura, 2023)}
\stopitemize

\ppkap{Metódy riadenia ICT verejnej správy ČR}
Súčasťou a kľúčovým predpokladom naplnenia cieľov IKČR je zavedenie efektívnej centrálnej koordinácie riadenia ICT. Zároveň je to aj podpora transformačných iniciatív, ktoré smerujú k digitalizácii VS a plnému digitálnemu governmentu. \uv{Metódy riadenia ICT verejnej správy ČR} (ďalej ako MRICT) je dokument, ktorý stanovuje pravidlá prevádzkovania ICT kapacít, kompetencií štátnych podnikov, riadenia útvarov informatiky, centrálneho koordinovaného riadenia ICT podpory eGovernmentu a podobne. MRICT nadväzuje na zásady riadenia ICT, ktoré sú súčasťou IKČR \scr(Digitální a informační agentura, 2023).

\pkap{Katalóg služieb verejnej správy}
Katalóg služieb VS je súčasťou registru práv a povinností (RPP) a obsahuje údaje o službách VS, úkonoch a dostupných kanáloch. Katalóg služieb VS sa dá vnímať z dvoch pohĺadov:

\startitemize[a]
\item{ako na klientskú aplikáciu, ktorá poskytuje údaje klientom}
\item{ako na úradnícku aplikáciu, ktorá je určená na zber a úpravu údajov}
\stopitemize

Funkcie katalógu služieb VS sa dajú rozdeliť do 4 kategórií:

\startitemize
\item{automatizačné – zber dát potrebných na automatizáciu,}
\item{informačné – poskytovanie prehľadu o existujícich službách VS a spôsobu ich spracovania,}
\item{publikačné – poskytovanie informácií, ktoré sú nezbytne potrebné na korektné zobrazovanie služieb VS na portáloch VS (kategórie, radenie...),}
\item{riadiace – riadenie poskytovania a dodávky služieb VS (tvorba plánu digitalizácie, zodpovednosť za služby...).}
\stopitemize

Časti katalógu služieb VS nie sú len služby vykonávané z úradnej moci, ale taktiež aj služby, ktoré iniciuje klient (subjekt práva). Na vyplnenie katalógu služieb je nutné urobiť nasledujúce kroky:

\startitemize[n]
\item{Identifikovať služby VS a popísať ich atribúty v agendách, ktoré ohlasujete.}
\item{Rozložit služby VS na jednotlivé úkony a popísať ich atribúty.}
\item{Definovať spôsob, akým dochádza k interakciou medzi OVM a klientom a určit obslužný kanál.}
\item{Určit, časové rámce a obslužné kanály pre vykovanie digitálneho úkonu a využívanie digitálnych služieb.}
\stopitemize

Vzhľadom k tomu, že údaje v katalogu služeb VS sú referenčné, je nutné ich udržiavať aktuálne \scr(Digitální a informační agentura, 2023).

V ďalších kapitolách budú objasnené vyššie spomenuté pojmy služba VS a úkon.

\ppkap{Služba VS}
Služba VS reprezentuje funkciu (činnosť) úradu, ktorá je poskytovaná konkrétnym OVM (úradníkom) konkrétnemu príjemcovi služby podľa príslušného právneho predpisu. Prináša príjemcovi hodnotu - buď vo forme benefitu alebo splnenia zákonnej povinnosti. Ak ide o interakciu medzi OVM a OVM, nepokladá sa to za službu VS. Pri službe VS ide vždy o interakciu medzi OVM a klientom (a opačne). Každá služba sa skladá z minimálne jedného úkonu. \scr(Digitální a informační agentura, 2023).

\ppkap{Úkon}
Úkon je taktiež interakcia medzi klientom a OVM, no v tomto prípade ide len o jednu interakciu, ktorá vedie k ďalšiemu úkonu (resp. k naplneniu výstupu služby, ak sa jedná o koncový úkon). Úkon sa teda dá definovať ako jeden krok, jedna časť služby VS \scr(Digitální a informační agentura, 2023).

\pkap[dizajn-system]{Jednotným dizajn manuálom elektronických služieb}
haha

\pkap{Popis metodiky, analýzy a návrhu}

\TODO
TODO

\pkap{Charakteristika vybraných služieb verejnej správy}
Na to, aby bolo možné vybrať vhodné služby VS k analýze a ďalším úkonom je potreba analyzovať rôzne životné situácie spojené s verejnou správou. V nasledujúcej tabuľke (viď \in{tabuľka}[situacie]) je zobrazený zoznam takýchto situácií spolu so službou a úradom, ktoré do riešenia danej situácie vstupujú.


  \setupTABLE[column][1][width=12cc]
  \setupTABLE[column][2][width=8cc]
  \setupTABLE[column][3][width=16cc]
  %\setupTABLE[column][4][width=0.25\textwidth]
  \setupTABLE[r][each][align={middle,lohi}]

  \Tabulka{situacie}{Životné situácie spojené so zamestnávateľmi a verejnou správou}{
    \bTABLE
      % Table header
      \bTR
        \bTH Životná situácia \eTH
        \bTH Úrad \eTH
        \bTH Služba VS \eTH
        %\bTH Služby v RPP \eTH
      \eTR

      % Table rows
	\bTR 
	    \bTD [nr=3] Nástup zamestnanca \eTD 
	    \bTD ČSSZ \eTD 
	    \bTD Oznámenie o nástupe do zamestnania (Přihlášky, odhlášky zamestnancov k nemocenskému poisteniu) \eTD  
	\eTR
	\bTR 
	    \bTD ZP \eTD 
	    \bTD Hromadné oznámenie zamestnávateľa (HOZ)  \eTD  
	\eTR
	\bTR 
	    \bTD MPSV/ÚP \eTD 
	    \bTD Informace o nástupe občana cudzinca, ktorý nepotrebuje/potrebuje pracovné oprávnenie do zamestnania  \eTD  
	\eTR
	\bTR 
	    \bTD [nr=4] Zamestnávanie osôb so zdravotným postihnutím\eTD 
	    \bTD MPSV  \eTD  
	    \bTD Evidencia náhradného plnenia \eTD 
	\eTR
	\bTR 
	    \bTD MPSV  \eTD  
	    \bTD Ohlásenie plnenia povinného podielu osôb so zdravotným postihnutím (OZP) \eTD 
	\eTR
	\bTR 
	    \bTD MPSV  \eTD  
	    \bTD Žiadosť o príspevok na zdriadenie pracovného miesta pre OZP \eTD 
	\eTR
	\bTR 
	    \bTD MPSV  \eTD  
	    \bTD Žiadosť o príspevok na Nový podnikateľský program\eTD 
	\eTR
	\bTR 
	    \bTD [nr=2] Voľné miesta\eTD 
	    \bTD MPSV \eTD 
	    \bTD Oznámenie voľných pracovných miest ÚP ČR \eTD  
	\eTR
	\bTR 
	    \bTD MPSV \eTD 
	    \bTD Oznámenie popisu pracovnej pozície pre Jobmatch do evidencie ÚP ČR neregistrovaným uživateľom  \eTD  
	\eTR
	\bTR 
	    \bTD Row 1, Col 1 \eTD 
	    \bTD Row 1, Col 2 \eTD 
	    \bTD Row 1, Col 3 \eTD  
	\eTR
    \eTABLE
  }




\TODO
TODO

\kap{Poznámky z konzultácie}
- kompetenčný model agendy VS

- centrálne zdielane služby

- porovnanie voči svetu - u nás (v čr) sú prenesená pusobnost a spravna pusobnost obcí
stavebné řízení - obec povoluje schvalovanie stavieb, ale obec môže byť dotknutá riadením (takže je vlastne schvalovatel aj účastník řízení)

- informačný koncepce, národný arch plán a rámec

- pôvodný zámer EMMA (vize a tak)

- právo na digitálne služby -> malo by priniest zlepšienie egov

- Jirka ohladom SAMO

\kap{Výsledky}

\pkap{Model požiadaviek}

\TODO
TODO

\pkap{Use case model}

\TODO
TODO

\pkap{Sekvenčný diagram}

\TODO
TODO

\pkap{Konceptuálny dátový model}

\TODO
TODO

\pkap{Logický dátový model pre SAMO}

\TODO
TODO

\pkap{Výber služby, ktorá bude implementovaná}

\TODO
TODO

\pkap{Implementácia vybranej služby}

\TODO
TODO

\pkap{Návrh testovacích scenárov}

\TODO
TODO

\pkap{Dokumentácia prevedených testov}

\TODO
TODO


\kap{Diskusia}

\TODO
TODO

\kap{Závěr}

\TODO
TODO

%%%%%%%%%%%%%%%%%%%%%%%%% \def\refname{}

\bbib

\publE{
\autor{Anderrsson, C.} \autor{Hallin, A.} \autor{Ivory, C.}
\nazev{Unpacking the digitalisation of public services: Configuring work during automation in local government}
\rok{2022}
\issn{0740624X}
\doi{10.1016/j.giq.2021.101662}
\www{https://www.sciencedirect.com/science/article/pii/S0740624X21000988}
\online{2023-11-19}
\nazevdok{Government Information Quarterly}
\cast{roč. \,39}
}

\publW{
\autor{Ardhaninggar, N.}
\nazev{E-Government Success Stories: Learning from Denmark and Estonia}
\rok{2023}
\www{https://moderndiplomacy.eu/author/nurulardhaninggar/}
\online{2024-01-31}
\nazevdok{moderndiplomacy.eu}
%\podnazev{Informační koncepce ČR}
}

\publX{
\autorkorp{Asseco Central Europe, a.s.}
\online{2023-11-27}
\www{interný SharePoint}
\nazev{SAMO conceptual application architecture}
\rok{2023}
}

\publX{
\autorkorp{Asseco Central Europe, a.s.}
\online{2023-11-26}
\www{interný dokument}
\nazev{Závěrečná zpráva o realizaci výsledků výzkumu a vývoje: VaV softwarové platformy embedded government (EMMA)}
\rok{2023}
}

\publE{
\autor{Barone, L. a kol.}
\nazev{State-of-play report on digital public administration and interoperability}
\rok{2023}
\isbn{978-92-68-08101-3}
\doi{10.2799/686251}
\www{https://op.europa.eu/en/publication-detail/-/publication/e2cf65a7-6719-11ee-9220-01aa75ed71a1/language-en}
\online{2024-1-12}
\nazevdok{Directorate-General for Informatics}
\cast{NO-04-23-973-EN-N}
}

\publW{
\autorkorp{Digitální a informační agentura}
\nazev{Architektura eGovernmentu ČR}
\rok{2023}
\www{https://archi.gov.cz/start}
\online{2023-11-26}
\nazevdok{Národní architektonický plán}
\podnazev{Informační koncepce ČR}
}

\publW{
\autorkorp{Digitální a informační agentura}
\nazev{Architektura eGovernmentu ČR}
\rok{2023}
\www{https://archi.gov.cz/start}
\online{2023-11-26}
\nazevdok{Národní architektonický plán}
\podnazev{Katalog služeb veřejné správy}
}

\publW{
\autorkorp{Digitální a informační agentura}
\nazev{Architektura eGovernmentu ČR}
\rok{2023}
\www{https://archi.gov.cz/start}
\online{2023-11-26}
\nazevdok{Národní architektonický plán}
\podnazev{Slovník pojmů eGovernmentu}
}

\publW{
 \online{2023-11-19}
 \autorkorp{Evropská komise.}
 \nazev{Balíček aktu o digitálních službách}
%\podnazev{Metodika}
 \www{https://digital-strategy.ec.europa.eu/cs/policies/digital-services-act-package}
 \nazevdok{Shaping Europe’s digital future}
 \rok{2022}
}

\publW{
 \online{2023-11-19}
 \autorkorp{Evropská komise.}
 \nazev{Index digitální ekonomiky a společnosti (DESI) 2022}
\podnazev{Česko}
 \www{https://digital-strategy.ec.europa.eu/en/policies/desi-czech-republic}
 \nazevdok{Shaping Europe’s digital future}
 \rok{2022}
}

\publW{
 \online{2023-11-19}
 \autorkorp{Evropská komise.}
 \nazev{Index digitální ekonomiky a společnosti (DESI) 2022}
\podnazev{Metodika}
 \www{https://digital-strategy.ec.europa.eu/cs/policies/desi}
 \nazevdok{Shaping Europe’s digital future}
 \rok{2022}
}

%\publA{
%\autor{OECD}
%\nazev{ Government at a Glance}
%\nakl{OECD Publishing}
%\vyd{2023}
%\rok{2023}
%\isbn{978-92-64-85180-1}
%\rozsah{60}
%}

%\publE{
% \autor{Karunia, L. a kol.}
% \nazev{Analysis of the Factors that Affect the Implementation of EGovernment in Indonesia}
% \nazevdok{International Journal of Membrane Science and Technology}
% \cast{Vol.\,10, No.\,3}
% \rok{2023}
% \umist{46}{54}
% %\issn{0896-3207}
%\doi{https://doi.org/10.1063/5.0118820}
%}

\publW{
\autor{Kilinger, A.}
\nazev{Obligatory Slovakian Information System (IS EFA) for exchanging B2G and B2B E-Invoice}
 \online{2024-01-29}
 \www{https://blog.seeburger.com/new-obligatory-slovakian-information-system-is-efa-for-b2g-and-b2b-e-invoicing/}
\issn{978-92-64-85180-1}
 \nazevdok{SEEBURGER}
 \rok{2023}
}

\publW{
 \autorkorp{Ministerstvo financií Slovenskej republiky}
 \nazev{Informačný systém elektronickej fakturácie - BETA}
 \online{2024-01-29}
 \www{https://web-einvoice-demo.mypaas.vnet.sk/}
 \nazevdok{e-Faktúra}
 \rok{©~2024}
}

\publA{
 \autor{OECD}
 \nazev{Government at a Glance}
 \nakl{Paris}{OECD Publishing}
 %\vyd{2023}
 \rok{2023}
 \xisbn{978-92-64-85180-1}
 \rozsah{234\stran}
}

\publD{%
 \autorkorp{United Nations}
 \nazev{E-Government Survey 2022}
 \nazevdok{The Future of Digital Government}
\nakl{UN}{New York}
\rok{2022}
\isbn{978-92-1-123213-4}
 \umist{32}{51}
}

%https://desapublications.un.org/sites/default/files/publications/2022-09/Web%20version%20E-Government%202022.pdf

\publW{
 \autorkorp{Slovensko.sk}
 \nazev{Ústredný portál verejnej správy}
 \online{2024-01-29}
 \www{https://www.slovensko.sk/sk/titulna-stranka}
 \nazevdok{Slovensko.sk}
 \rok{©~2024}
}

%\publE{
%\autor{van der Linden, N. a kol.}
%\nazev{eGovernment Benchmark 2023: Insight Report}
%\rok{2023}
%\isbn{978-92-68-05653-0}
%\doi{10.2759/474056}
%\www{https://espanadigital.gob.es/sites/espanadigital/files/2023-10/1_eGovernment_Benchmark_2023__Insight_Report_tmnnsE9rmVDxpAZ8IJECpnUZGLA_98708.pdf}
%\online{2024-2-7}
%\nazevdok{Connecting Digital Governments}
%\cast{KK-BH-23-001-EN-N}
%}

\publE{
\autor{van der Linden, N. a kol.}
\nazev{eGovernment Benchmark 2022: Insight Report}
\rok{2022}
\isbn{ 978-92-76-49793-6}
\doi{10.2759/488218}
\www{https://prod.ucwe.capgemini.com/wp-content/uploads/2022/07/eGovernment-Benchmark-2022-1.-Insight-Report.pdf}
\online{2024-2-7}
\nazevdok{Connecting Digital Governments}
\cast{KK-08-22-084-EN-N}
}



\ebib

\stopbodymatter

%%%%%%%%%%%%%%%%%%%%%%%% Varianta, kdy seznamy jsou součástí práce a nejsou uvedeny v přílohách

\setupsectionblock[backmatter][before={\setuplist[kap][before={}]}]

\startbackmatter

\THESIScompletelistof{tables}
\THESIScompletelistof{figures}
\THESIScompletelistof{abbreviations}
%\THESIScompletelistof{codes}

\stopbackmatter

%%%%%%%%%%%%%%%%%%%%%%%% Varianta, kdy seznamy nejsou součástí práce, ale jsou zařazeny do příloh.
%%%%%%%%%%%%%%%%%%%%%%%% Níže uvedeným čtyřem příkazům postačí odstranit znak procenta.
%%%%%%%%%%%%%%%%%%%%%%%% Naopak před výše uvedené čtyři příkazy je potřeba znak procenta vložit.

\startappendices

\cast{Přílohy}
%\THESIScompletelistof{tables}
%\THESIScompletelistof{figures}
%\THESIScompletelistof{abbreviations}
%\THESIScompletelistof{codes}

\stopappendices

\stopthesis

\endinput		

%%%% TODO %%%%%%%%%%%%%%%%%%%%%%%%%%%%%%
Tady si můžeš psát poznámky, které se neobjeví ve výstupu.
